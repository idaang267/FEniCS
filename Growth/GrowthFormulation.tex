\documentclass[12pt,3p]{article}
\usepackage[utf8]{inputenc}
\usepackage[english]{babel}
 \usepackage[margin=0.5in]{geometry}
 \usepackage{amsmath}
\usepackage{mathtools}
\usepackage{enumitem}
\usepackage{physics}
\usepackage[round,numbers]{natbib}
\usepackage[colorlinks = false]{hyperref}

\setcounter{tocdepth}{4} 
\setcounter{secnumdepth}{4}

\numberwithin{equation}{section}
\begin{document}

\title{Notes on Research \\
	\large{Updated on \_}}
\author{Ida Ang}
\date{\vspace{-5ex}}
\maketitle

\tableofcontents
\newpage

%====================================================================================
%====================================================================================
%====================================================================================
\section{neo-Hookean Model}
The deformation can be decomposed into elastic and plastic (growth) terms 
\begin{equation}\label{DefMulti}
\mathbf{F} = \mathbf{F_e} \mathbf{F_g}
\end{equation}
where $\mathbf{F_e}$ is connected to the elastic stress response of the material and $\mathbf{F_g}$ is the growth tensor 
\begin{align}\label{JMulti}
\begin{split}
J &= \det \mathbf{F} \\
   &= \det (\mathbf{F_e} \mathbf{F_g}) \quad \text{property of determinant} \\
   &= \det \mathbf{F_e} \det \mathbf{F_g} \\
J &= J_e J_g 
\end{split}
\end{align}
where 
\begin{align}\label{ThirdInvariant}
I_e &= \tr \mathbf{C_e} \\
J_e &= \det \mathbf{F_e} \quad \text{and} \quad J_g = \det \mathbf{F_g}
\end{align}
where the right Cauchy Green Tensor is $\mathbf{C_e} = \mathbf{F_e}^T \mathbf{F_e}$ \\
\noindent\rule{19cm}{0.5pt} \\ % Straight line across
Proof of the derivative $\pdv{\trace(\mathbf{F^T} \mathbf{F})}{\mathbf{F}}$ in indicial: 
\begin{align*}
\pdv{\trace(\mathbf{F^T} \mathbf{F})}{\mathbf{F}} &= \pdv{F_{kI} F_{kI}}{F_{pQ}} \quad \text{Product Rule}\\ 
									&= \pdv{F_{kI}}{F_{pQ}} F_{kI} + F_{kI} \pdv{F_{kI}}{F_{pQ}} \\
									&= \delta_{kp} \delta_{IQ} F_{kI} + F_{kI} \delta_{kp} \delta_{IQ} \\
									&= F_{pQ} + F_{pQ} = 2 F_{pQ}
\end{align*}
Therefore, we have the following relationship: 
\begin{equation}\label{proof1}
\pdv{\trace(\mathbf{F^T} \mathbf{F})}{\mathbf{F}} = 2 \mathbf{F}
\end{equation}
Second proof: 
\begin{align*}
\pdv{\det \mathbf{F} }{\mathbf{F}} = \det \mathbf{F} \mathbf{F^{-T}} = J \mathbf{F}^{-T}
\end{align*}
\noindent\rule{19cm}{0.5pt} % Straight line across

%====================================================================================
%====================================================================================
\subsection{Compressible}
If the compressible neo-Hookean model is as follows: 
\begin{align}\label{CompNH}
W_e = \frac{\mu}{2} (I_e - 3 - 2 \ln J_e) + \frac{\lambda}{2} (\ln J_e)^2
\end{align}
The nominal stress tensor is defined as: 
\begin{equation}\label{PW}
\mathbf{P} = \pdv{W_e}{\mathbf{F_e}} \\
\end{equation}
substituting in Eq. \ref{CompNH} and using the chain rule: 
\begin{align*}
\mathbf{P} &= \pdv{W_e}{I_e} \pdv{I_e}{\mathbf{F_e}} + \pdv{W_e}{J_e} \pdv{J_e}{\mathbf{F_e}} \\
&= \frac{\mu}{2} 2 \mathbf{F_e} + \bigg[ \frac{\mu}{2} \bigg( -2 \pdv{\ln J_e}{J_e}\bigg) + \frac{\lambda}{2} \pdv{(ln J_e)^2}{J_e} \bigg] \det \mathbf{F_e} \mathbf{F_e}^{-T} \\
&= \mu \mathbf{F_e} + \bigg[ - \mu \pdv{ln J_e}{J_e} + \frac{\lambda}{2} \pdv{(\ln J_e)^2}{J_e} \bigg] J_e \mathbf{F_e}^{-T} \\
&= \mu \mathbf{F_e} + \bigg[ - \mu \frac{1}{J_e} + \lambda \ln J_e \frac{1}{J_e} \bigg] J_e \mathbf{F_e}^{-T} \\
\mathbf{P} &= \mu \mathbf{F_e} + ( \lambda \ln J_e - \mu ) \mathbf{F_e}^{-T} 
\end{align*}

%====================================================================================
%====================================================================================
\subsection{Incompressible}
For an incompressible material, $J_e = \det \mathbf{F_e} = 1$ ; therefore, Eq. \ref{CompNH} can be modified to: 
\begin{align*}
W_e &= \frac{\mu}{2} (I_e - 3 - 2 \ln (1)) + \frac{\lambda}{2} (\ln (1))^2 \\
W_e &= \frac{\mu}{2} (I_e - 3) 
\end{align*}
To enforce incompressibility, introduce a lagrange multiplier, p, which acts like a stress term. 
\begin{align}\label{EnergyDensity}
\begin{split}
W_e(\mathbf{F_e}) &= \frac{\mu}{2} (I_e - 3) + \text{p} (J_e -1) \\
W_e(\mathbf{F_e}) &= \frac{\mu}{2} (\tr (\mathbf{F_e}^T \mathbf{F_e}) - 3) + \text{p} (\det \mathbf{F_e} -1)
\end{split}
\end{align}
Use the chain rule, just as in the last case. 
\begin{align*}
\mathbf{P} &= \pdv{W_e}{I_e} \pdv{I_e}{\mathbf{F_e}} + \pdv{W_e}{J_e} \pdv{J_e}{\mathbf{F_e}} \\
		&= \pdv{W_e}{\tr (\mathbf{F_e}^T \mathbf{F_e})} \pdv{\tr (\mathbf{F_e}^T \mathbf{F_e})}{\mathbf{F_e}} + \pdv{W_e}{\det \mathbf{F_e}} \pdv{\det \mathbf{F_e}}{\mathbf{F_e}} \\
		&= \frac{\mu}{2} 2 \mathbf{F_e} + p J_e \mathbf{F_e}^{-T} \\
		&= \mu \mathbf{F_e} + p J_e \mathbf{F_e}^{-T}
\end{align*}

%====================================================================================
%====================================================================================
%====================================================================================
\section{Incompressible formulation based on Alawiye paper}
The deformation still has a multiplicative decomposition. Eq. \ref{DefMulti}
\begin{equation*}
\mathbf{F} = \mathbf{F_e} \mathbf{F_g}
\end{equation*}
but we assume that the material is incompressible and the only local change in volume of the material comes from the growth process. 
\begin{align}
\det \mathbf{F_e} = 1
\end{align}
For a hyperelastic material, we can define an augmented energy density functional 
\begin{align}
\overline{W} (\mathbf{F}, \mathbf{F_g}) &= (\det \mathbf{F_g}) W_e (\mathbf{F_e}) - p (\det \mathbf{F_e} - 1)
\end{align}
where $\mathbf{F_e} = \mathbf{F} \mathbf{F_g}^{-1}$. 
\begin{align}
W_e (\mathbf{F_e}) = \frac{\mu}{2} (\tr \mathbf{F_e} \mathbf{F_e}^T - 3)
\end{align}
We can calculate 
\begin{align*}
\pdv{W_e}{\mathbf{F_e}} &= \frac{\mu}{2} \pdv{}{\mathbf{F_e}} \big( \tr \mathbf{F_e} \mathbf{F_e}^T - 3 \big) \\
	&= \frac{\mu}{2} \pdv{(\tr \mathbf{F_e} \mathbf{F_e}^T)}{\mathbf{F_e}} \\
\pdv{W_e}{\mathbf{F_e}} &= \mu \mathbf{F_e}
\end{align*}
The nominal stress tensor can be found as 
\begin{align*}
\mathbf{P} = \pdv{\overline{W}}{\mathbf{F}}
\end{align*}
Using the chain rule
\begin{align*}
\pdv{\overline{W}}{\mathbf{F}} &= \pdv{\overline{W}}{\mathbf{F_e}} \pdv{\mathbf{F_e}}{\mathbf{F}} \\
						&= \pdv{}{\mathbf{F_e}} \bigg[ (\det \mathbf{F_g}) W_e (\mathbf{F_e}) - p (\det \mathbf{F_e} - 1) \bigg] \pdv{\mathbf{F_e}}{\mathbf{F}} \\
						&= \bigg[ \det \mathbf{F_g} \pdv{W_e (\mathbf{F_e})}{\mathbf{F_e}} - p \pdv{(\det \mathbf{F_e} - 1)}{\mathbf{F_e}} \bigg] \pdv{(\mathbf{F} \mathbf{F_g}^{-1})}{\mathbf{F}} \\
						&= \bigg[ \det \mathbf{F_g} \pdv{W_e (\mathbf{F_e})}{\mathbf{F_e}} - p \pdv{\det \mathbf{F_e}}{\mathbf{F_e}} \bigg] \mathbf{F_g}^{-1} \quad \text{where} \, \det \mathbf{F} = \det \mathbf{F_g}\\
						&= \bigg[ \det \mathbf{F} \pdv{W_e (\mathbf{F_e})}{\mathbf{F_e}} - p \det \mathbf{F_e} \mathbf{F_e}^{-T} \bigg] \mathbf{F_g}^{-1} \quad \text{substitute} \\
\pdv{\overline{W}}{\mathbf{F}} &= \bigg[ \mu J \mathbf{F_e} - p J_e \mathbf{F_e}^{-T} \bigg] \mathbf{F_g}^{-1}
\end{align*}
This disagrees with Eq. 5 in Ciarletta paper 
\begin{align*}
\text{Ours:}\, 
\mathbf{P} = \pdv{\overline{W}}{\mathbf{F}} &= \mathbf{F_g}^{-1} \bigg[  J \pdv{W_e}{\mathbf{F_e}} - p J_e \mathbf{F_e}^{-T} \bigg]\\
\text{Ciarletta et. al.:} \, 
\mathbf{P} = \pdv{\overline{W}}{\mathbf{F}} &= J \mathbf{F_g}^{-1} \bigg[ \pdv{W_e}{\mathbf{F_e}} - p \mathbf{F_e}^{-1} \bigg] 
\end{align*}

\subsection{Weak Form}
Finally, in order to enforce incompressibility we use Eq. \ref{invariants} where:
\begin{equation*}
\det F_e = 1 \rightarrow \det F_e - 1 = 0 \rightarrow J_e - 1 = 0
\end{equation*}
This can be multiplied by a test function and integrated over the domain $\Omega_o$
\begin{equation}\label{wFormIn}
\int_{\Omega_o} (J_e - 1) \tau dV = 0
\end{equation}
We have the bilinear form and linear form where we have two test functions
\begin{equation*}
a\big((\sigma,u),(\tau, v) \big) = L\big((\tau, v) \big)  \indent \forall (\tau, v) \in \Sigma_0 \times V 
\end{equation*}
Therefore, we have the bilinear and linear forms: 
\begin{equation}\label{bilinear}
a\big((\sigma,u),(\tau, v) \big) = \int_{\Omega_o} \big( P : \text{Grad}(v) + (J-1) \cdot \tau \big) dV
\end{equation}
\begin{equation}\label{linear}
L\big((\tau, v) \big) = \int_{\Omega_o} b v dV + \int_{\partial \Omega_o} T v dS
\end{equation}
The bilinear and linear form (Eq. \ref{bilinear} and \ref{linear})can also be combined into one statement: 
\begin{equation}\label{F}
a\big((\sigma,u),(\tau, v) \big) - L\big((\tau, v) \big) = \int_{\Omega_o} \big( P : \text{Grad}(v) + (J_e-1) \cdot \tau \big) dV - \int_{\Omega_o} b v dV - \int_{\partial \Omega_o} T v dS = 0 
\end{equation}
Solving for two unknowns, displacement (u) and hydrostatic pressure (p), which will allow us to obtain the nominal stress field (P)
\newpage 
%====================================================================================
%====================================================================================
%====================================================================================
\section{Formulation based on Ambrosi and Mollika Paper}
Deformation gradient: 
\begin{equation}\label{Deformation}
\mathbf{F}=\frac{\partial \chi}{\partial \mathbf{X}}
\end{equation}
The deformation can be decomposed into elastic and plastic (growth) terms 
\begin{equation}\label{DefMulti}
\mathbf{F} = \mathbf{F_e} \mathbf{F_g}
\end{equation}
where $\mathbf{F_e}$ is connected to the stress response of the material and $\mathbf{F_g}$ is the growth tensor 
\begin{equation}\label{JMulti}
J = J_e J_g
\end{equation}
The third invariant of the elastic deformation tensor where $\varrho$ is the density field 
\begin{equation}\label{ThirdInElastic}
J_e = \det \mathbf{F_e} = \frac{\varrho_o}{\varrho}
\end{equation}
The third invariant of the growth tensor shows whether a particle is growing ($J_g > 1$)or resorbing ($J_g < 1$)
\begin{equation}\label{ThirdInGrowth}
J_g = \det \mathbf{F_g}
\end{equation}

%====================================================================================
%====================================================================================
\subsection{Strain Energy Density and Stress Equations}
Strain Energy Density for General Blatz-Ko Material 
\begin{align}\label{BlatzKo}
\begin{split}
W_{e} &= \frac{v f}{2} \left[ \left(\mathrm{I}_{e}-3\right) -\frac{2}{q} \left(\mathrm{III}_{e}^{q / 2}-1\right) \right] + \frac{v(1-f)}{2} \left[ \left( \frac{\mathrm{II}_{e}}{\mathrm{III}_{e}}-3 \right) - \frac{2}{q} \left( \mathrm{III}_{e}^{-(q / 2)}-1 \right) \right] \quad \text{where } f = 1 \\
W_{e} &= \frac{v}{2} \left[ \left(\mathrm{I}_{e}-3\right) -\frac{2}{q} \left(\mathrm{III}_{e}^{q / 2}-1\right) \right] 
\end{split}
\end{align}
The first and third invariant are calculated as follows:
\begin{align*}
\mathrm{I}_e = \tr \mathbf{C_e} = \tr \mathbf{F_e^T} \mathbf{F_e} \quad \quad \mathrm{III}_e = \det \mathbf{F_e}
\end{align*}
The Cauchy Stress Tensor 
\begin{equation}\label{CauchyStressTensor}
\mathbf{T}=\varrho \mathbf{F}_{e}\left(\frac{\partial W_{e}}{\partial \mathbf{F}_{e}}\right)^{\mathrm{T}}
\end{equation}
First Piola Stress Tensor
\begin{equation}\label{FirstPiolaStressTensor}
\mathbf{P}=J \mathbf{T} \mathbf{F}^{-\mathrm{T}}
\end{equation}
Second Piola Stress Tensor
\begin{equation}\label{SecondPiolaStressTensor}
\mathbf{S}=J \mathbf{F}^{-1} \mathbf{T} \mathbf{F}^{-\mathbf{T}}
\end{equation}

%====================================================================================
%====================================================================================
\subsection{Stress Calculation}
\begin{align}\label{StressPartial}
\pdv{W_e}{\mathbf{F_{e}}} = \pdv{W_e}{\mathrm{I}_e} \pdv{\mathrm{I}_e}{\mathbf{F_e}} + \pdv{W_e}{\mathrm{III}_e} \pdv{\mathrm{III}_e}{\mathbf{F_e}} 
\end{align}
Invariant derivative relationships
\begin{align*}
\pdv{\mathrm{I}_e}{\mathbf{F_e}} &= 2 \mathbf{F_e} \\ 
\pdv{\mathrm{I}_e}{\mathbf{F_e}} &= \det \mathbf{F_e} \mathbf{F_e^{-T}} = \mathrm{III}_e \mathbf{F_e^{-T}} 
\end{align*}
Calculate partials of strain energy density with respect to the invariants:
\begin{align*}
\pdv{W_{e}}{\mathrm{I}_e} &= \frac{v}{2} \pdv{( \mathrm{I}_{e}-3 )}{\mathrm{I}_e} \\
					 &= \frac{v}{2} 
\end{align*}
\begin{align*}
\pdv{W_{e}}{\mathrm{III}_e} &= \frac{v}{2} \bigg( - \frac{2}{q} \bigg) \pdv{(\mathrm{III}_e^{q/2} - 1)}{\mathrm{III}_e} \\
					   &= - \frac{v}{q} \frac{q}{2} \mathrm{III}_e^{q/2 - 1} \\
					   &= - \frac{v}{2} \mathrm{III}_e^{q/2 - 1} 
\end{align*}
Therefore, substituting this into Eq. \ref{StressPartial}
\begin{align}
\begin{split}
\pdv{W_e}{\mathbf{F_{e}}} &= \pdv{W_e}{\mathrm{I}_e} \pdv{\mathrm{I}_e}{\mathbf{F_e}} + \pdv{W_e}{\mathrm{III}_e} \pdv{\mathrm{III}_e}{\mathbf{F_e}} \\
					&= \frac{v}{2} 2 \mathbf{F_e} - \frac{v}{2} \mathrm{III}_e^{q/2 - 1} \mathrm{III}_e \mathbf{F_e^{-T}} \\
\pdv{W_e}{\mathbf{F_{e}}} &= v \mathbf{F_e} - \frac{v}{2} \mathrm{III}_e^{q/2} \mathbf{F_e^{-T}}
\end{split}			
\end{align}

%====================================================================================
\subsubsection{Cauchy Stress Tensor}
The Cauchy Stress Tensor can now be calculated using Eq. \ref{CauchyStressTensor}
\begin{align*}
\mathbf{T} &= \varrho \mathbf{F}_{e} \bigg( \frac{\partial W_{e}}{\partial \mathbf{F}_{e}} \bigg)^{\mathrm{T}} \\
		 &= \varrho \mathbf{F}_{e} \big( v \mathbf{F_e} - \frac{v}{2} \mathrm{III}_e^{q/2} \mathbf{F_e^{-T}} \big)^{\mathrm{T}} \\
		 &= \varrho \mathbf{F}_{e} \big( v \mathbf{F_e}^T - \frac{v}{2} \mathrm{III}_e^{q/2} \mathbf{F_e^{-1}} \big) \\
		 &= \varrho v \big( \mathbf{F}_{e} \mathbf{F_e}^T - \frac{1}{2} \mathrm{III}_e^{q/2} \mathbf{F}_{e} \mathbf{F_e^{-1}} \big) \\
\mathbf{T} &= \varrho v \big( \mathbf{F}_{e} \mathbf{F_e}^T - \frac{1}{2} \mathrm{III}_e^{q/2} \big) 
\end{align*}
Rewrite this term where 
\begin{align}\label{CauchyStressBK}
\begin{split}
\mathbf{T} &= \varrho v \big( \mathbf{F}_{e} \mathbf{F_e}^T - \frac{1}{2} \mathrm{III}_e^{q/2} \big) \quad \text{where } \varrho = \frac{\varrho_o}{J_e}\\
		 &= \frac{\varrho_o v}{J_e} \big( \mathbf{F}_{e} \mathbf{F_e}^T - \frac{1}{2} \mathrm{III}_e^{q/2} \big) \quad \text{where } \mu = \varrho_o v \\
\mathbf{T} &= \frac{\mu}{J_e} \big( \mathbf{F}_{e} \mathbf{F_e}^T - \frac{1}{2} \mathrm{III}_e^{q/2} \big) 
\end{split}
\end{align}

%====================================================================================
\subsubsection{First Piola-Kirchoff Stress Tensor}
The first PK stress tensor can be calculated using Eq. \ref{FirstPiolaStressTensor}
First Piola Stress Tensor
\begin{align*}
\mathbf{P} &= J \mathbf{T} \mathbf{F}^{-\mathrm{T}} \quad \text{Substitute Eq. \ref{CauchyStressBK}} \\
		 &= J \frac{\mu}{J_e} \big( \mathbf{F}_{e} \mathbf{F_e}^T - \frac{1}{2} \mathrm{III}_e^{q/2} \big)  \mathbf{F}^{-\mathrm{T}} \quad \text{Use equation \ref{DefMulti} and \ref{JMulti}} \\
		 &= J_e J_g \frac{\mu}{J_e} \big( \mathbf{F}_{e} \mathbf{F_e}^T - \frac{1}{2} \mathrm{III}_e^{q/2} \big)  (\mathbf{F_e} \mathbf{F_g})^{-\mathrm{T}} \\
		 &= \mu J_g \big[ \mathbf{F}_{e} \mathbf{F_e}^T \mathbf{F_e}^{-T} - \frac{1}{2} \mathrm{III}_e^{q/2} \mathbf{F_e}^{-T} \big] \mathbf{F_g}^{-\mathrm{T}} \\
		 &= \mu J_g \big[ \mathbf{F}_{e} (\mathbf{F_e}^{-1} \mathbf{F_e})^{T} - \frac{1}{2} \mathrm{III}_e^{q/2} \mathbf{F_e}^{-T} \big] \mathbf{F_g}^{-\mathrm{T}} \quad \text{Recognize identity} \\
\mathbf{P} &= \mu J_g \big[ \mathbf{F}_{e} - \frac{1}{2} J_e^{q/2} \mathbf{F_e}^{-T} \big] \mathbf{F_g}^{-\mathrm{T}}
\end{align*}

%====================================================================================
%====================================================================================
\subsection{Summary}
\begin{align}\label{AMFinalEquations}
\begin{split}
W_{e} = \frac{v}{2} \left[ \left(\mathrm{I}_{e}-3\right) -\frac{2}{q} \left(\mathrm{III}_{e}^{q / 2}-1\right) \right] 
\end{split}
\end{align}
Discrepancy shown below:
\begin{align}
\begin{split}
\mathbf{P} &= \mu J_g \big[ \mathbf{F}_{e} - \frac{1}{2} J_e^{q/2} \mathbf{F_e}^{-T} \big] \mathbf{F_g}^{-\mathrm{T}} \\
\mathbf{P} &= \mu J_g \big[ \mathbf{F}_{e} - J_e^{q} \mathbf{F_e}^{-T} \big] \mathbf{F_g}^{-\mathrm{T}}
\end{split}
\end{align}
where 
\begin{align*}
\alpha &= \frac{\lambda}{2 \mu} = \frac{\nu}{1 - 2 \nu} \\
- \alpha &= \frac{q}{2}
\end{align*}
Therefore: 
\begin{equation}
\mathbf{P} = \mu J_g \big[ \mathbf{F}_{e} - \frac{1}{2} J_e^{- \alpha} \mathbf{F_e}^{-T} \big] \mathbf{F_g}^{-\mathrm{T}} 
\end{equation}

%====================================================================================
%====================================================================================
\subsection{Weak Form}
\begin{align}\label{EqmEq}
\pdv{P_{iK}}{X_K} = - B_i
\end{align}
Rearrange and multiply Eq. \ref{EqmEq} with a test function $\xi_i$
\begin{align*}
\pdv{P_{iK}}{X_K} \xi_i &= - B_i \xi_i \quad \text{Integrate over domain} \\
\int \pdv{P_{iK}}{X_K} \xi_i dV &= - \int  B_i \xi_i dV \\
\int (P_{iK} \xi_i)_{,K} dV - \int P_{iK} \pdv{\xi_i}{X_K} dV &= - \int  B_i \xi_i dV \\
\int (P_{iK} \xi_i)_{,K} dV - \int P_{iK} \pdv{\xi_i}{X_K} dV &= - \int  B_i \xi_i dV \quad \text{Use divergence theorem} \\
\int P_{iK} N_K \xi_i dA - \int P_{iK} \pdv{\xi_i}{X_K} dV &= - \int  B_i \xi_i dV \quad \text{Recognize } P_{iK} N_K = T_i \\
\int T_i \xi_i dA - \int P_{iK} \pdv{\xi_i}{X_K} dV &= - \int B_i \xi_i dV 
\end{align*}
Rearrange 
\begin{equation}\label{2}
\int P_{iK} \pdv{\xi_i}{X_K} dV - \int T_i \xi_i dA - \int  B_i \xi_i dV = 0
\end{equation}

\end{document}