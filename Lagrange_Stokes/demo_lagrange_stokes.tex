\documentclass[12pt,3p]{article}
\usepackage[T1]{fontenc}
\usepackage[utf8]{inputenc}
\usepackage[english]{babel}
\usepackage[margin=0.7in]{geometry}
\usepackage{amsmath}
\usepackage{mathtools}
\usepackage{enumitem}
\usepackage{physics}

\usepackage[round,numbers]{natbib}
\usepackage[colorlinks = false]{hyperref}

\begin{document}

\title{FEniCS: Lagrange formulation for Stokes Equation \\
	\large{Based on DOLFIN 1.4.0 Demo 21: Stokes equations}}
	\vspace{-5ex}
\author{Ida Ang \\
	\small{Made: 10/4/19} \quad \small{Edited: 10/4/19}}
\date{\vspace{-5ex}}
\maketitle

\section{Problem Definition}
The Stokes equations represent a considerable simplification of the full Navier–Stokes equations, especially in the incompressible Newtonian case. There is no longer a time dependency. 

\subsection{Strong Form}
 The following equations solve for velocity, u, and pressure, p, with the second equation enforcing incompressibility. 
\begin{align}\label{ProbDef}
\begin{split}
- \nabla \cdot (\nabla \mathbf{u} + p \mathbf{I}) &= \mathbf{f} \\
\nabla \cdot \mathbf{u} &= 0 
\end{split}
\end{align}
Boundary conditions:
\begin{align}
\begin{split}
u &= u_o \quad \text{on } \Gamma_D \\
\nabla u \cdot n + p n &= g \quad \text{on } \Gamma_N
\end{split}
\end{align}
Note that consulting Wikipedia, the classical definition is written in the following form ($\mu$ = dynamic viscosity). Note that the sign of pressure is changed from the classical definition (body forces can be defined as negative or positive).
\begin{align*}
\mu \nabla^2 u -\nabla p + f &= 0 \\
\nabla \cdot u &= 0 
\end{align*}
The sign difference for pressure is done in order to have a symmetric (but not positive-definite) system of equations rather than a non-symmetric (but positive-definite) system of equations.

\subsection{Weak Form}
We have two partial differential equations, leading to the introduction of two different test functions, v and q. 
\subsubsection{Equation 1} 
\begin{align*} 
- \nabla \cdot \bigg[ \pdv{}{x_i} \mathbf{e_i} \otimes u_k \mathbf{e_k} + p \mathbf{I} \bigg] &= f \\
- \nabla \cdot \bigg[ \pdv{u_k}{x_i} \mathbf{e_i} \otimes \mathbf{e_k} + p \bigg] &= f \\
- \nabla \cdot \pdv{u_k}{x_i} \mathbf{e_i} \otimes \mathbf{e_k} - \nabla \cdot p &= f \\
- \pdv{}{x_l} \mathbf{e_l} \cdot \pdv{u_k}{x_i} \mathbf{e_i} \otimes \mathbf{e_k} - \pdv{}{x_l} \mathbf{e_l} \cdot p &= f \\
- \pdv{^2 u_k}{x_l x_i} \mathbf{e_l} \cdot \mathbf{e_i} \otimes \mathbf{e_k} - \pdv{p}{x_l} \mathbf{e_l} &= f \\
- \pdv{^2 u_k}{x_l x_i} \delta_{li} \mathbf{e_k} - \pdv{p}{x_l} \mathbf{e_l} &= f _q \mathbf{e_q} \\
- \pdv[2]{u_k}{x_l} \mathbf{e_k} - \pdv{p}{x_l} \mathbf{e_l} &= f _q \mathbf{e_q} \quad \text{Multiply by test function} \\
- \pdv[2]{u_k}{x_l} \mathbf{e_k} \cdot v_p \mathbf{e_p} - \pdv{p}{x_l} \mathbf{e_l} \cdot v_p \mathbf{e_p} &= f _q \mathbf{e_q} \cdot v_p \mathbf{e_p} \\
- \pdv[2]{u_k}{x_l} v_k - \pdv{p}{x_l} v_l &= f _q v_q \quad \text{Integrate over domain} \\
- \int_{\Omega} \pdv[2]{u_k}{x_l} v_k dx - \int_{\Omega} \pdv{p}{x_l} v_l dx &= \int_{\Omega} f _q v_q dx 
\end{align*}
Use integration by parts on the left hand side terms 
\begin{align}\label{intParts}
\begin{split}
(fg)' = f'g + fg' \rightarrow f'g &= (fg)' - fg' \\
					f''g &= (f'g)' - f'g'
\end{split}
\end{align}
Where, we can use integration by parts twice 
\begin{align*}
\pdv[2]{u_k}{x_l} v_k &= \bigg(\pdv{u_k}{x_l} v_k \bigg)_{,l} - \pdv{u_k}{x_l} \pdv{v_k}{x_l} \\
\pdv{p}{x_l} v_l &= (p v_l)_{,l} - p \pdv{v_l}{x_l}
\end{align*}
Substitute: 
\begin{align*}
- \int_{\Omega} \pdv[2]{u_k}{x_l} v_k dx - \int_{\Omega} \pdv{p}{x_l} v_l dx &= \int_{\Omega} f _q v_q dx \\
- \int_{\Omega} \bigg(\pdv{u_k}{x_l} v_k \bigg)_{,l} dx + \int_{\Omega} \pdv{u_k}{x_l} \pdv{v_k}{x_l} dx - \int_{\Omega} (p v_l)_{,l} dx + \int_{\Omega} p \pdv{v_l}{x_l} dx &= \int_{\Omega} f _q v_q dx \quad \text{Use divergence theorem} \\
- \int_{\delta \Omega} \pdv{u_k}{x_l} v_k n_l ds + \int_{\Omega} \pdv{u_k}{x_l} \pdv{v_k}{x_l} dx - \int_{\partial \Omega} p v_l n_l ds + \int_{\Omega} p \pdv{v_l}{x_l} dx &= \int_{\Omega} f _q v_q dx 
\end{align*}
Rearrange and change from indicial to direct notation: 
\begin{align*}
\begin{split}
\int_{\Omega} \pdv{u_k}{x_l} \pdv{v_k}{x_l} dx + \int_{\Omega} p \pdv{v_l}{x_l} dx - \int_{\partial \Omega} \pdv{u_k}{x_l} v_k n_l ds - \int_{\partial \Omega} p v_l n_l ds&= \int_{\Omega} f _q v_q dx   \\ 
\int_{\Omega} \pdv{u_k}{x_l} \pdv{v_k}{x_l} dx + \int_{\Omega} p \pdv{v_l}{x_l} dx - \int_{\partial \Omega} \bigg( \pdv{u_k}{x_l} v_k n_l + p v_l n_l \bigg) ds &= \int_{\Omega} f _q v_q dx  \\
\int_{\Omega} \nabla u \cdot \nabla v dx +  \int_{\Omega} p (\nabla \cdot v) dx - \int_{\partial \Omega} \big( \nabla u \cdot n  + p n \big) v ds&=  \int_{\Omega} f v dx 
\end{split}
\end{align*}

\subsubsection{Equation 2} 
Multiply the second equation of Eq. \ref{ProbDef} with test function q:
\begin{align*}
\begin{split}
\nabla \cdot u &= 0 \quad \text{Multiply by test function q}\\
(\nabla \cdot u) q &= 0 \quad \text{Integrate over domain} \\
\end{split}
\end{align*}
\begin{equation}\label{WeakForm2}
\int_{\Omega} q (\nabla \cdot u) dx = 0 
\end{equation}

\subsubsection{Boundary Conditions}
 Relabeling to match with FEniCS implementation, (v $\rightarrow v_u$ and q $\rightarrow v_p$). Utilize the Neumann boundary condition on the surface integral term.
\begin{align*}
\int_{\Omega} \nabla u \cdot \nabla v_u dx +  \int_{\Omega} p (\nabla \cdot v_u) dx - \int_{\partial \Omega} g v ds &= f v dx \\
\int_{\Omega} v_p (\nabla \cdot u) dx &= 0 
\end{align*}

\section{Lagrange Multiplier}
Pressure and lagrange multiplier space are equal order. Displacement is a higher order space. 
\begin{align*}
\int_{\Omega} \nabla u \cdot \nabla v_u dx +  \int_{\Omega} p (\nabla \cdot v_u) dx + \int_{\partial \Omega} \lambda v_u[1] ds &= 0 \\
\int_{\Omega} v_p (\nabla \cdot u) dx &= 0 \\ 
\int_{\partial \Omega} v_l (u[1] - u_in) ds &= 0 
\end{align*}
%Save the solution in VTK format: \\ 
%{\fontfamily{qcr}\selectfont
%file
%}
\end{document}
