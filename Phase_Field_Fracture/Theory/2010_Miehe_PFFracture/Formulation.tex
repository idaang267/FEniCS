\documentclass[12pt,3p]{article}
\usepackage[utf8]{inputenc}
\usepackage[english]{babel}
 \usepackage[margin=0.5in]{geometry}
 \usepackage{amsmath}
\usepackage{mathtools}
\usepackage{enumitem}
\usepackage{physics}
\usepackage[round,numbers]{natbib}
\usepackage[colorlinks = false]{hyperref}

\setcounter{tocdepth}{4} 
\setcounter{secnumdepth}{4}

\numberwithin{equation}{section}
\begin{document}

\title{Notes on Miehe 2010 \\
	\large{Updated on \_}}
\author{Ida Ang}
\date{\vspace{-5ex}}
\maketitle

\tableofcontents
\newpage

%====================
%====================
%====================
\section{Definitions}
Young's Modulus: E \\
Poisson's Ratio: $\nu$ \\
Shear Modulus or first Lamé Parameter: $\mu$ 
\begin{equation}
\mu = \frac{E}{2 (1 + \nu)} \quad 
\end{equation}
Second Lamé Parameter: $\lambda$ 
\begin{equation}\label{EqLame}
\lambda = \frac{E \nu}{(1+ \nu) (1 - 2 \nu)} 
\end{equation}
Bulk Modulus: $\kappa$ \\ 
\begin{equation}\label{EqBulk}
\kappa = \lambda + \frac{2}{3} \mu 
\end{equation}
Invariants of strain tensor $\boldsymbol{\varepsilon}$
\begin{equation}
\begin{array}{l}
E_{1}=\varepsilon_{i i}=\operatorname{tr} \boldsymbol{\varepsilon} \\
E_{2}=\varepsilon_{i j} \varepsilon_{i j}=\operatorname{tr} \boldsymbol{\varepsilon}^{2} \\
E_{3}=\varepsilon_{i j} \varepsilon_{j k} \varepsilon_{k i}=\operatorname{tr} \varepsilon^{3}
\end{array}
\end{equation}
The derivative of the invariants with respect to the strain tensor: 
\begin{equation}
\begin{array}{ll}
\partial E_{1} / \partial \varepsilon_{i j}=\delta_{i j} & \partial E_{1} / \partial \boldsymbol{\varepsilon}=\mathrm{I} \\
\partial E_{2} / \partial \varepsilon_{i j}=2 \varepsilon_{i j} & \partial E_{2} / \partial \boldsymbol{\varepsilon}=2 \boldsymbol{\varepsilon} \\
\partial E_{3} / \partial \varepsilon_{i j}=3 \varepsilon_{i k} \varepsilon_{k j} & \partial E_{3} / \partial \boldsymbol{\varepsilon}=3 \boldsymbol{\varepsilon}^{2}
\end{array}
\end{equation}
Deformation 
\begin{align}
\mathbf{F} &= \nabla \mathbf{u} + \mathbf{I} \\
\mathbf{F} &= \begin{bmatrix}
			\lambda_1 & 0 & 0 \\
			0 & \lambda_2 & 0 \\
			0 & 0 & \lambda_3 
		\end{bmatrix}
\end{align}
Invariants of deformation gradient 
\begin{align}
\begin{split}
I_{1} &= \operatorname{tr} \mathbf{F} = \lambda_1 + \lambda_2 + \lambda_3 \\
I_{2} &= \frac{1}{2} \big[ ( \operatorname{tr} \mathbf{F} )^2 - \operatorname{tr} (\mathbf{F}^2) \big] \\
I_{3} &= \operatorname{det} \mathbf{F} = \lambda_1 \lambda_2 \lambda_3 
\end{split}
\end{align}
Strain 
\begin{align}
\boldsymbol{\varepsilon} &= \frac{1}{2} (\nabla \mathbf{u} + \nabla^T \mathbf{u}) \\
				      &= \frac{1}{2} \big[ \mathbf{F} - \mathbf{I} + \mathbf{F}^T - \mathbf{I} \big] \\
\boldsymbol{\varepsilon} &= \frac{1}{2} (\mathbf{F} + \mathbf{F}^T ) - \mathbf{I} 
\end{align}
which can also be written as: 
\begin{align}
\boldsymbol{\varepsilon} &= \frac{1}{2} (\mathbf{F} + \mathbf{F}^T ) - \mathbf{I} \\
\boldsymbol{\varepsilon} &= \begin{bmatrix}
				      	    \lambda_1 - 1& 0 & 0 \\
					    0 & \lambda_2 - 1& 0 \\
					    0 & 0 & \lambda_3 -1
				      	   \end{bmatrix}
\end{align}
Therefore we can write certain invariant in terms of stretches 
\begin{align}
\begin{split}
E_{1} &=\operatorname{tr} \boldsymbol{\varepsilon} \\
	 &= \lambda_1 + \lambda_2 + \lambda_3 -3 \\
E_{1} &= \lambda_i - 3
\end{split}
\end{align}
\begin{align}
\begin{split}
E_{2} &= \operatorname{tr} \boldsymbol{\varepsilon}^{2} \\
	 &= (\lambda_1 - 1)^2 + (\lambda_2 - 1)^2 + (\lambda_3 - 1)^2 \\
	 &= \lambda_1^2 + \lambda_2^2 + \lambda_3^2 - 2 (\lambda_1 + \lambda_2 + \lambda_3) + 3 \\
E_{2} &= \lambda_i^2 - 2 \lambda_i + 3
\end{split}
\end{align}
This one is an invariant relationship 
\begin{align}
\begin{split}
E_1^2 &= \big( \operatorname{tr} \boldsymbol{\varepsilon} \big)^2 \\
	   &= ( \lambda_1 + \lambda_2 + \lambda_3 -3 ) ( \lambda_1 + \lambda_2 + \lambda_3 -3 ) \\
	   &= \lambda_1^2 + \lambda_2^2 + \lambda_3^2 
	   + 2 (\lambda_1 \lambda_2 + \lambda_1 \lambda_3 + \lambda_2 \lambda_3 ) 
	   - 3 (\lambda_1 + \lambda_2 + \lambda_3 ) + 9 \\
E_1^2 &= \lambda_i^2 + \frac{J}{\lambda_i} - 3 \lambda_i + 9
\end{split}
\end{align}
Right Cauchy-Green (CG) definition
\begin{align}
\mathbf{C} = \mathbf{F}^T \mathbf{F}
\end{align}
Invariants of right CG
\begin{align}
\begin{split}
I^\mathbf{C}_{1} &= \operatorname{tr} \mathbf{C} = \lambda_1^2 + \lambda_2^2 + \lambda_3^2\\
I^\mathbf{C}_{2} &= \frac{1}{2} \big[ ( \operatorname{tr} \mathbf{C} )^2 - \operatorname{tr} (\mathbf{C}^2) \big] \\
I^\mathbf{C}_{3} &= \operatorname{det} \mathbf{C} = \lambda_1^2 \lambda_2^2 \lambda_3^2
\end{split}
\end{align}
Lastly we have the following useful identites
\begin{align}\label{EqProof1}
\begin{split}
\frac{\partial I_1^{\mathbf{C}}}{\partial \mathbf{F}} &= 2 \mathbf{F} \\
\frac{\partial I_3}{\partial \mathbf{F}} &= I_3 \mathbf{F}^{-T}
\end{split}
\end{align}
% \noindent\rule{19cm}{0.5pt} % Straight line across

%====================
%====================
\subsection{Stresses}
Pull-back operation
\begin{align}
\begin{split}
\boldsymbol{\sigma} &= \frac{1}{J} \frac{\partial W}{\partial \mathbf{F}} \mathbf{F}^T  \\
\boldsymbol{\sigma} &= \frac{1}{J} \mathbf{P} \mathbf{F}^T \rightarrow \mathbf{P} = J \boldsymbol{\sigma} \mathbf{F}^{T}
\end{split}
\end{align}
where for convenience we are using W instead of $\psi_0$, which is Miehe's notation
\begin{align}
\begin{split}
\boldsymbol{\sigma} &= \frac{\partial \psi_{0}}{\partial \boldsymbol{\varepsilon}} \\
				&= \frac{\partial \psi_{0}}{\partial E_1} \frac{\partial E_1}{\partial \boldsymbol{\varepsilon}}
				+ \frac{\partial \psi_{0}}{\partial E_2 } \frac{\partial E_2 }{\partial \boldsymbol{\varepsilon}}
				+ \frac{\partial \psi_{0}}{\partial E_3 } \frac{\partial E_3 }{\partial \boldsymbol{\varepsilon}} \\
\boldsymbol{\sigma} &= \frac{\partial \psi_{0}}{\partial E_1} \mathrm{I} 
				+ 2 \frac{\partial \psi_{0}}{\partial E_2 } \boldsymbol{\varepsilon}
				+ 3 \frac{\partial \psi_{0}}{\partial E_3 } \boldsymbol{\varepsilon}^{2}
\end{split}
\end{align}
and the nominal stress is defined as 
\begin{align}
\begin{split}
\mathbf{P} &= \frac{\partial W}{\partial \mathbf{F}} \\
		 &= \frac{\partial W}{\partial I_1^\mathbf{C}} \frac{\partial I_1^\mathbf{C}}{\partial \mathbf{F}} + \frac{\partial W}{\partial I_3} \frac{\partial I_3}{\partial \mathbf{F}} \\
\mathbf{P} &= 2 \frac{\partial W}{\partial I_1^\mathbf{C}} \mathbf{F} + I_3 \frac{\partial W}{\partial I_3} \mathbf{F}^{-T}
\end{split}
\end{align}


%====================
%====================
%====================
\section{Isotropic elasticity}
Standard free energy
\begin{align}
\begin{split}
\psi_{0}(\boldsymbol{\varepsilon}) &= \frac{\lambda}{2} \operatorname{tr}^{2}[\boldsymbol{\varepsilon}] +\mu \operatorname{tr}\left[\boldsymbol{\varepsilon}^{2}\right] \\
\psi_{0}(\boldsymbol{\varepsilon}) &= \frac{\lambda}{2} E_1^2 + \mu E_2
\end{split}
\end{align}
Obtain stress 
\begin{align}
\begin{split}
\boldsymbol{\sigma_0} &= \frac{\partial \psi_{0}}{\partial E_1} \mathrm{I} 
				+ 2 \frac{\partial \psi_{0}}{\partial E_2 } \boldsymbol{\varepsilon} \\
				&= \lambda E_1 \mathrm{I} + 2 \mu \boldsymbol{\varepsilon} \\
\boldsymbol{\sigma_0} &= \lambda \big( \operatorname{tr} \boldsymbol{\varepsilon} \big) \mathrm{I} + 2 \mu \boldsymbol{\varepsilon}
\end{split}
\end{align}
Additive decomposition form of stored energy 
\begin{equation}
\psi(\boldsymbol{\varepsilon}, d)=[g(d)+k] \psi_{0}^{+}(\boldsymbol{\varepsilon})+\psi_{0}^{-}(\boldsymbol{\varepsilon})
\end{equation}
The stress associated with this form 
\begin{align}
\begin{split}
\boldsymbol{\sigma} &=\partial_{\boldsymbol{\varepsilon}} \psi \\
				&=\left[(1-d)^{2}+k\right] \frac{\partial \Psi_{0}^{+}(\boldsymbol{\varepsilon})}{\partial \boldsymbol{\varepsilon}}+\frac{\partial \Psi_{0}^{-}(\boldsymbol{\varepsilon})}{\partial \boldsymbol{\varepsilon}} \\
\boldsymbol{\sigma} &=\left[(1-d)^{2}+k\right] \boldsymbol{\sigma}_{0}^{+}-\boldsymbol{\sigma}_{0}^{-}
\end{split}
\end{align}
where 
\begin{align}
g(d) = (1-d)^2
\end{align}


%====================
%====================
%====================
\section{Hyperelasticity}
\subsection{Compressible neo-Hookean: Case I}
Strain energy function 
\begin{align}
W(\mathbf{F}) &= \frac{\mu}{2} (I_1^{\mathbf{C}} - 3 - 2 \ln I_3) + \frac{\kappa}{2} (I_3 - 1)^2 
\end{align}
Calculation of the 1st PK
\begin{align}
\begin{split}
\mathbf{P} &= 2 \frac{\partial W}{\partial I_1^\mathbf{C}} \mathbf{F} + I_3 \frac{\partial W}{\partial I_3} \mathbf{F}^{-T} \\
		&= 2 \frac{\mu}{2} \mathbf{F} + I_3 \bigg[ \mu \frac{1}{I_3} + \frac{\kappa}{2} 2 (I_3 -1 ) \bigg]  \mathbf{F}^{-T} \\
		&= \mu \mathbf{F} + \mu \mathbf{F}^{-T} + \kappa I_3 (I_3 - 1) \mathbf{F}^{-T} \\
\mathbf{P} &= \mu (\mathbf{F} + \mathbf{F}^{-T}) + \kappa I_3 (I_3 - 1) \mathbf{F}^{-T} 
\end{split}
\end{align}
Therefore, we can also calculate the Cauchy Stress
\begin{align}
\begin{split}
\boldsymbol{\sigma} &= \frac{1}{J} \mathbf{P} \mathbf{F}^T\\
				&= \frac{1}{I_3} \bigg[ \mu (\mathbf{F} + \mathbf{F}^{-T}) + \kappa I_3 (I_3 - 1) \mathbf{F}^{-T} \bigg] \mathbf{F}^T \\
				&= \frac{1}{I_3} \mu (\mathbf{F} \mathbf{F}^T + \mathbf{F}^{-T} \mathbf{F}^T ) + \kappa (I_3 - 1) \mathbf{F}^{-T} \mathbf{F}^T \\
\boldsymbol{\sigma} &= \frac{\mu}{I_3} (\mathbf{b} + \mathbf{I}) + \kappa (I_3 - 1)\mathbf{I} 
\end{split}
\end{align}

%====================
%===================
\subsection{Incompressible neo-Hookean: Case I }
\begin{align}
W(\mathbf{F}) = \frac{\mu}{2} (I_1^{\mathbf{C}}  - 3)  + \text{p} (I_3-1)
\end{align}
Calculation of the 1st PK
\begin{align}
\begin{split}
\mathbf{P} &= 2 \frac{\partial W}{\partial I_1^\mathbf{C}} \mathbf{F} + I_3 \frac{\partial W}{\partial I_3} \mathbf{F}^{-T} \\
		&= 2 \frac{\mu}{2} \mathbf{F} + I_3 p \mathbf{F}^{-T} \\
\mathbf{P} &= \mu \mathbf{F} + p I_3 \mathbf{F}^{-T}
\end{split}
\end{align}
Therefore, we can also calculate the Cauchy Stress
\begin{align}
\begin{split}
\boldsymbol{\sigma} &= \frac{1}{I_3} \big[  \mu \mathbf{F} + p I_3 \mathbf{F}^{-T} \big] \mathbf{F}^T \\
				&= \frac{1}{I_3} \big[  \mu \mathbf{b} + p I_3 \mathbf{I} \big] \\
\boldsymbol{\sigma} &= \frac{\mu}{I_3} \mathbf{b} + p \mathbf{I} 
\end{split}
\end{align}
If $ \det \mathbf{F} = 1$, then 
\begin{align}
\boldsymbol{\sigma} &= \mu\mathbf{b} + p \mathbf{I} 
\end{align}
%====================
%====================
%====================
\subsection{Summary}
Compressible neo-Hookean, case 1
\begin{align}
\begin{split}
W(\mathbf{F}) &= \frac{\mu}{2} (I_1^{\mathbf{C}} - 3 - 2 \ln I_3) + \frac{\kappa}{2} (I_3 - 1)^2 \\
\mathbf{P} &= \mu ( \mathbf{F} - \mathbf{F}^{-T}) + \kappa I_3 (I_3 - 1) \mathbf{F}^{-T} 
\end{split}
\end{align}
Compressible neo-Hookean, case 2
\begin{align}
\begin{split}
W(\mathbf{F}) &= \frac{\mu}{2} (I_1^{\mathbf{C}} - 3 - 2 \ln I_3) + \frac{\lambda}{2} (\ln I_3)^2 \\
\mathbf{P} &= \mu ( \mathbf{F} - \mathbf{F}^{-T}) + \lambda I_3 (\ln I_3) \mathbf{F}^{-T} 
\end{split}
\end{align}
Incompressible neo-Hookean, case 1
\begin{align}
\begin{split}
W(\mathbf{F}) &= \frac{\mu}{2} (I_1^{\mathbf{C}}  - 3)  + \text{p} (I_3-1) \\
\mathbf{P} &= \mu \mathbf{F} + p I_3 \mathbf{F}^{-T} 
\end{split}
\end{align}
Incompressible neo-Hookean, case 2 
\begin{align}
\begin{split}
W(\mathbf{F}) &= \frac{\mu}{2} (I_1^{\mathbf{C}}  - 3 - 2 \ln I_3 )  + \text{p} (I_3-1) \\
\mathbf{P} &= \mu ( \mathbf{F} - \mathbf{F}^{-T}) + p I_3 \mathbf{F}^{-T} 
\end{split}
\end{align}


\end{document}