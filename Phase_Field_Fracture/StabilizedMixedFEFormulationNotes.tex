\documentclass[12pt,3p]{article}
\usepackage[utf8]{inputenc}
\usepackage[english]{babel}
 \usepackage[margin=0.75in]{geometry}
 \usepackage{amsmath}
 \usepackage{bm}
\usepackage{mathtools}
\usepackage{enumitem}
\usepackage[round,numbers]{natbib}
\usepackage[colorlinks = false]{hyperref}
\usepackage{xcolor}


\numberwithin{equation}{section}
\begin{document}

\title{Stabilized Mixed Finite Element Formulation}
\author{Personal Notes by Ida Ang}
\maketitle

%===============================
%===============================
%===============================
\section*{Definitions}
$\mathbf{F}$: Deformation gradient \\
$\mathbf{I}$: second-order unit tensor \\
$\bm{u}$: Displacement \\
$J$: determinant of the deformation gradient \\
$\mathbf{C}$: Right Cauchy-Green Strain Tensor \\
$\mathcal{W}(\mathbf{F})$: strain energy function \\
$\mathbf{P}$: first Piola-Kirchhoff stress tensor \\
$\mathbf{S}$: second PK stress tensor \\ \\
$\alpha$: cracks are represented by a scalar phase-field variable \\
$p$: Lagrange multiplier, hydrostatic pressure field \\
$\kappa$: bulk modulus 
\begin{equation}
\kappa = \frac{E}{3 (1 - 2 \nu)}
\end{equation}
$\mu$: shear modulus 
\begin{equation}
\mu = \frac{E}{2 (1 + \nu)}
\end{equation}
$\lambda$: Lamé modulus 
\begin{equation}
\lambda = \frac{\nu E}{(1+ \nu) (1 - 2 \nu)}
\end{equation}
$\mathcal{E}_\ell$: potential energy functional 
$w(\alpha)$ is an increasing function representing the specific energy dissipation per unit of volume \\
$c_w$ is a normalization constant 

%===============================
%===============================
%===============================
\section{Hyperelastic Phase-Field Fracture Models}
Deformation Gradient 
\begin{equation}\label{EqDefGrad}
\mathbf{F}=\mathbf{I}+\nabla\otimes\bm{u}
\end{equation}
where $J=\det \mathbf{F}$ and $\mathbf{C}=\mathbf{F}^T\mathbf{F}$.

The strain energy function $\mathcal{W}(\mathbf{F})$ is defined per unit reference volume such that the first PK and second PK
\begin{subequations}\label{EqPK}
\begin{align}
\mathbf{P} &= \frac{\partial \mathcal{W}(\mathbf{F})}{\partial \mathbf{F}} \\
\mathbf{S} &= 2 \frac{\partial \mathcal{W}(\mathbf{F})}{\partial \mathbf{C}}
\end{align}
\end{subequations}
where $\mathbf{P}=\mathbf{F}\mathbf{S}$. 

%===============================
%===============================
\subsection{Phase-Field Fracture Model}
Non-modified strain energy function is the compressible Neo-Hookean:
\begin{align}\label{StrainEnergyNH}
\mathcal{W}(\mathbf{F}) = \frac{\mu}{2} (I_1 - 3 - 2 \ln J)
\end{align}
For incompressible hyperelastic materials, the strain energy function  is defined as
\begin{equation}\label{EqEnergyIncompressible}
\widetilde{\mathcal{W}}(\mathbf{F}) = \mathcal{W}(\mathbf{F}) + p\left(J-1\right),
\end{equation}
We instead consider a damage dependent relaxation of the incompressibility constraint and the phase-field couples to the strain energy through the modified function 
\begin{equation}\label{EqEnergyRelaxed}
\widetilde{\mathcal{W}}(\mathbf{F},\alpha) = a(\alpha)\mathcal{W}(\mathbf{F}) + a^3(\alpha)\frac{1}{2}\kappa\left(J-1\right)^2
\end{equation}
where the decreasing stiffness modulation function and $w(\alpha)$ is an increasing function representing the specific energy dissipation per unit of volume
\begin{equation}\label{EqStiffMod}
a (\alpha) = (1 - \alpha)^2 \quad w(\alpha) = \alpha
\end{equation}
In the code we have the following definition
\begin{align*}
b (\alpha) = (1 - \alpha)^3 = \sqrt{a^3(\alpha)}
\end{align*}
To circumvent numerical difficulties, we resort to the classical mixed formulation and introduce the pressure-like field 
\begin{equation}\label{eq_pressure}
p = -\sqrt{a^3(\alpha)}\kappa\left(J-1\right),
\end{equation}
as an independent variable along with the displacement field. \\ \\
Lastly, the normalization constant is defined as: 
\begin{align}\label{NormConst} % normalization constant.
c_w=\int_0^1\sqrt{w(\alpha)}d\alpha
\end{align}
The first Piola-Kirchhoff stress tensor is given: 
\begin{align}\label{eq_firstPK}
\begin{split}
\mathbf{P} &= \frac{\partial \tilde{\mathcal{W}}(\mathbf{F},\alpha)}{\partial \mathbf{F}} \\
		 &= \frac{\partial}{\partial \mathbf{F}} \bigg[ a(\alpha)\mathcal{W}(\mathbf{F}) + a^3(\alpha) \frac{1}{2}\kappa\left(J-1\right)^2 \bigg] \\
		 &= a(\alpha)\frac{\mathcal{W}(\mathbf{F})}{\partial \mathbf{F}} + a^3(\alpha) \frac{1}{2} \kappa \frac{\partial (J-1)^2}{\partial \mathbf{F}}  \\
\mathbf{P} &= a(\alpha)\frac{\mathcal{W}(\mathbf{F})}{\partial \mathbf{F}} + a^3(\alpha)\kappa\left(J-1\right)\frac{\partial J}{\partial \mathbf{F}} \quad \text{where } \frac{\partial J}{\partial \mathbf{F}} = J \mathbf{F}^{-T} \\
\mathbf{P} &= a(\alpha)\frac{\mathcal{W}(\mathbf{F})}{\partial \mathbf{F}} + a^3(\alpha)\kappa (J-1) J \mathbf{F}^{-T} 
\end{split}
\end{align}
Note that we DO NOT consider this form of the modified functional; therefore, we do not consider this PK stress. 

%===============================
\subsubsection{According to the rough draft}
Therefore our modified function:
\begin{align*}
\widetilde{\mathcal{W}}(\mathbf{F},\alpha) &= a(\alpha)\mathcal{W}(\mathbf{F}) + a^3(\alpha)\frac{1}{2} \kappa (J-1)^2 \\
	&= a(\alpha)\mathcal{W}(\mathbf{F}) + a^3(\alpha)\frac{1}{2} \frac{-p}{\sqrt{a^{3}(\alpha)}} (J-1) \\
	&= a(\alpha)\mathcal{W}(\mathbf{F}) - \sqrt{a^3(\alpha)} \frac{p}{2} (J-1) \\
 \widetilde{\mathcal{W}}(\mathbf{F},\alpha) &= a(\alpha)\mathcal{W}(\mathbf{F}) - b(\alpha) \frac{p}{2} (J-1)
\end{align*}
In the rough draft of the paper we have Eq. 5: energy functional of a possibly fractured elastic body with isotropic surface energy
\begin{align*}
\mathcal{E}_{\ell}(\boldsymbol{u}, \alpha) &=\int_{\Omega} \widetilde{\mathcal{W}}(\mathbf{F},\alpha)  d \Omega + \frac{G_{c}}{c_{w}} \int_{\Omega} \bigg(\frac{w(\alpha)}{\ell}+\ell\|\nabla \alpha\|^{2} \bigg) d \Omega - \int_{\partial_N \Omega} \mathbf{\tilde{g}_0} \cdot \mathbf{u} d A \\
\mathcal{E}_{\ell}(\boldsymbol{u}, p, \alpha) &=\int_{\Omega} \bigg[ a(\alpha)\mathcal{W}(\mathbf{F}) - b(\alpha) \frac{p}{2} (J-1) \bigg] d \Omega + \frac{G_{c}}{c_{w}} \int_{\Omega} \bigg(\frac{w(\alpha)}{\ell}+\ell\|\nabla \alpha\|^{2} \bigg) d \Omega \\
\mathcal{E}_{\ell}(\boldsymbol{u}, p, \alpha) &=\int_{\Omega} a(\alpha)\mathcal{W}(\mathbf{F}) d \Omega - \int_{\Omega} b(\alpha) \frac{p}{2} (J-1) d \Omega + \frac{G_{c}}{c_{w}} \int_{\Omega} \bigg(\frac{w(\alpha)}{\ell}+\ell\|\nabla \alpha\|^{2} \bigg) d \Omega
\end{align*}
\textcolor{red}{Note that we are missing one term.} 

%===============================
\subsubsection{According to the code}
In the code we have the following for the energy functional of the energy problem 
\begin{align*}
\widetilde{W} (\mathbf{F}, \alpha) &= a(\alpha) \mathcal{W} (\mathbf{F}) - b(\alpha) p (J-1) - \frac{p^2}{2 \lambda} 
\end{align*}
Energy functional, where we ignore the surface term 
\begin{align*}
\mathcal{E}_{\ell}(\boldsymbol{u}, \alpha) &= \int_{\Omega} \widetilde{\mathcal{W}}(\mathbf{F},\alpha)  d \Omega + \frac{G_{c}}{c_{w}} \int_{\Omega} \bigg(\frac{w(\alpha)}{\ell}+\ell\|\nabla \alpha\|^{2} \bigg) d \Omega - \int_{\partial_N \Omega} \mathbf{\tilde{g}_0} \cdot \mathbf{u} d A \\
	&= \int_{\Omega} a(\alpha) \mathcal{W} (\mathbf{F}) d \Omega - \int_{\Omega} b(\alpha) p (J-1) d \Omega - \int_{\Omega} \frac{p^2}{2 \lambda} d \Omega + \frac{G_{c}}{c_{w}} \int_{\Omega} \bigg(\frac{w(\alpha)}{\ell}+\ell\|\nabla \alpha\|^{2} \bigg) d \Omega 
\end{align*}

%===============================
\subsubsection{According to a derivation from Bin2020 Paper}
Versus Eq. 21 where we drop $\lambda_b$ which is not a consideration in this formulation
\begin{align}
\mathcal{E}_{\ell}(\boldsymbol{u}, \alpha) =\int_{\Omega} a(\alpha) \mathcal{W}(\mathbf{F}, \alpha) d \Omega+\frac{G_{c}}{c_{w}} \int_{\Omega} \bigg(\frac{w(\alpha)}{\ell}+\ell\|\nabla \alpha\|^{2} \bigg) d \Omega 
\end{align}
Starting from Eq. 25 in the 2020 Li and Bouklas paper where $\kappa$ is the bulk modulus 
\begin{align}
\begin{split}
\mathcal{E}_{\ell}\left(\boldsymbol{u}, p, \Lambda, \alpha\right) &= \mathcal{E}_{\ell} (\boldsymbol{u}, \alpha ) + \int_{\Omega} \frac{p^{2}}{2 \kappa} d \Omega +\int_{\Omega} \Lambda(p+\sqrt{a^{3}(\alpha)} \kappa (J-1)) d \Omega \quad \Lambda = -p/\kappa \\
		&= \mathcal{E}_{\ell} (\boldsymbol{u}, \alpha ) + \int_{\Omega} \frac{p^{2}}{2 \kappa} d \Omega +\int_{\Omega} - \frac{p}{\kappa} (p+\sqrt{a^{3}(\alpha)} \kappa (J-1)) d \Omega \\
		&= \mathcal{E}_{\ell} (\boldsymbol{u}, \alpha ) + \int_{\Omega} \frac{p^{2}}{2 \kappa} d \Omega -\int_{\Omega} \frac{p^2}{\kappa} d \Omega - \int_{\Omega} \frac{p}{\kappa} \sqrt{a^{3}(\alpha)} \kappa (J-1)) d \Omega \\
\mathcal{E}_{\ell}\left(\boldsymbol{u}, p, \alpha\right) &= \mathcal{E}_{\ell} (\boldsymbol{u}, \alpha ) - \int_{\Omega} \frac{p^{2}}{2 \kappa} d \Omega - \int_{\Omega} \sqrt{a^{3}(\alpha)} p(J-1)) d \Omega
\end{split}
\end{align}
Substitute in $\mathcal{E}_{\ell} (\boldsymbol{u}, \alpha )$ and substitute Eq. \ref{StrainEnergyNH}
\begin{align*}
\begin{split}
\mathcal{E}_{\ell}\left(\boldsymbol{u}, p, \alpha\right) &= \int_{\Omega} a(\alpha) \mathcal{W}(\mathbf{F}) d \Omega+\frac{G_{c}}{c_{w}} \int_{\Omega} \bigg(\frac{w(\alpha)}{\ell}+\ell\|\nabla \alpha\|^{2} \bigg) d \Omega- \int_{\Omega} \frac{p^{2}}{2 \kappa} d \Omega - \int_{\Omega} \sqrt{a^{3}(\alpha)} p(J-1) d \Omega 
\end{split}
\end{align*}
After taking the directional derivative of the prior equation, we can introduce the stabilization term 
\begin{align*}
- \frac{\varpi h^2}{2\mu} \sqrt{a^3(\alpha)} \sum_{e=1}^{n_{el}}\int_{\Omega^{e}} J \mathbf{C}^{-1} : \big( \nabla p \cdot \nabla q \big) \,dV = 0 
\end{align*}

%===============================
%===============================
\subsection{Changes for 2D Plane-Stress Models}
Following the code, we have the following energy functional of the energy problem
\begin{align*}
\widetilde{W} (\mathbf{F}, \alpha) &= (a(\alpha) + k_\ell) \mathcal{W} (\mathbf{F}) - b(\alpha) p (J-1) - \frac{p^2}{2 \lambda} \\
\widetilde{W} (\mathbf{F}, \alpha) &= \big( a(\alpha) + k_\ell) \frac{\mu}{2} (I_c - 3 - 2 \ln J) - b(\alpha) p (J-1) - \frac{p^2}{2 \lambda}
\end{align*}
Where we know $k_\ell$ is a modeling parameter, so we can list the energy functional as: 
\begin{equation}
\widetilde{W} (\mathbf{F}, \alpha) = a (\alpha) \frac{\mu}{2} (I_c - 3 - 2 \ln J) - b(\alpha) p (J-1) - \frac{p^2}{2 \lambda} 
\end{equation}
Derive the 1st PK stress
\begin{align*}
\mathbf{P} &= \frac{\partial \widetilde{W} (\mathbf{F})}{\partial \mathbf{F}} \\
		    &= a (\alpha) \mu (\mathbf{F} - \mathbf{F}^{-T}) - b (\alpha) p \frac{\partial \det \mathbf{F}}{\partial \mathbf{F}} \\
\mathbf{P} &= a (\alpha) \mu (\mathbf{F} - \mathbf{F}^{-T}) - b (\alpha) p J \mathbf{F}^{-T}
\end{align*}
Taking the third component to be zero, in the plane stress case
\begin{align*}
\mathbf{P}_{33} = a (\alpha) \mu (\mathbf{F}_{33} - \mathbf{F}^{-1}_{33}) - b (\alpha) p J \mathbf{F}^{-1}_{33} &= 0 \\
(1- \alpha)^2 \mu (\mathbf{F}_{33} - \mathbf{F}^{-1}_{33}) - (1- \alpha)^3 p J \mathbf{F}^{-1}_{33} &= 0 \\
\mu (\mathbf{F}_{33} - \mathbf{F}^{-1}_{33}) - (1- \alpha) p J \mathbf{F}^{-1}_{33} &= 0 \\
\mathbf{F}_{33} - \mathbf{F}^{-1}_{33} - \frac{(1- \alpha) p J}{\mu} \mathbf{F}^{-1}_{33} &= 0 \\
\mathbf{F}_{33} \mathbf{F}_{33} - \mathbf{F}^{-1}_{33} \mathbf{F}_{33} - \frac{(1- \alpha) p J}{\mu} \mathbf{F}^{-1}_{33} \mathbf{F}_{33} &= 0 \\
\mathbf{F}_{33}^2 - 1 - \frac{(1- \alpha) p J}{\mu} &= 0
\end{align*}
This can be multiplied by its associated test function to obtain the weak form 
\begin{align*}
\int_{\Omega} \bigg( \mathbf{F}_{33}^2 - 1 - \frac{(1-\alpha) p J}{\mu} \bigg) v_{F_{33}} dV &= 0
\end{align*}
%===============================
%===============================
\subsubsection{Changes for 2D Discrete Crack Model}
If we are considering a discrete fracture method
\begin{align*}
\begin{split}
\mathcal{E}_{\ell}\left(\boldsymbol{u}, p, \alpha\right) &= \int_{\Omega} a(\alpha) \mathcal{W}(\mathbf{F}) d \Omega + \frac{G_{c}}{c_{w}} \int_{\Omega} \bigg(\frac{w(\alpha)}{\ell}+\ell\|\nabla \alpha\|^{2} \bigg) d \Omega- \int_{\Omega} \frac{p^{2}}{2 \kappa} d \Omega - \int_{\Omega} \sqrt{a^{3}(\alpha)} p(J-1) d \Omega \\
\mathcal{E}_{\ell}\left(\boldsymbol{u}, p, \alpha\right) &= \int_{\Omega} \mathcal{W}(\mathbf{F}) d \Omega - \int_{\Omega} \frac{p^{2}}{2 \kappa} d \Omega - \int_{\Omega} p(J-1) d \Omega \\
\end{split}
\end{align*}
with the stabilization term and plane stress in the weak form 
\begin{align*}
- \frac{\varpi h^2}{2\mu} \int_{\Omega} J \mathbf{C}^{-1} : \big( \nabla p \cdot \nabla q \big) \,dV &= 0 \\
\int_{\Omega} \bigg( \mathbf{F}_{33}^2 - 1 - \frac{p J}{\mu} \bigg) v_{F_{33}} dV &= 0
\end{align*}
where we have assumed for the energy functional 
\begin{align*}
\widetilde{\mathcal{W}} (\mathbf{F}, \alpha) = \frac{\mu}{2} (I_c - 3 - 2 \ln J) - p (J-1) - \frac{p^2}{2 \lambda}
\end{align*}
Therefore, we can calculate the 1st Piola Kirchoff Stress as:
\begin{align*}
\mathbf{P} &= \frac{\mu}{2} (2 \mathbf{F} - \frac{2}{J} J \mathbf{F}^{-T}) - p J \mathbf{F}^{-T} \\
		&= \mu (\mathbf{F} - \mathbf{F}^{-T}) - p J \mathbf{F}^{-T}
\end{align*}
Taking the third component to be zero
\begin{align*}
P_{33} &=  \mu (F_{33} - F_{33}^{-1}) - p J F_{33}^{-1} = 0 \\
		&= F_{33} - F_{33}^{-1} - \frac{p J}{\mu} F_{33}^{-1} = 0 \\
P_{33} &= F_{33}^2 - 1 - \frac{p J}{\mu} = 0
\end{align*}

%===============================
%===============================
\subsubsection{Changes for 2D displacement formulation}
Removing pressure terms
\begin{align*}
\begin{split}
\mathcal{E}_{\ell}\left(\boldsymbol{u}, p, \alpha\right) &= \int_{\Omega} a(\alpha) \mathcal{W}(\mathbf{F}) d \Omega + \frac{G_{c}}{c_{w}} \int_{\Omega} \bigg(\frac{w(\alpha)}{\ell}+\ell\|\nabla \alpha\|^{2} \bigg) d \Omega- \int_{\Omega} \frac{p^{2}}{2 \kappa} d \Omega - \int_{\Omega} \sqrt{a^{3}(\alpha)} p(J-1) d \Omega \\
\mathcal{E}_{\ell}\left(\boldsymbol{u}, p, \alpha\right) &= \int_{\Omega} a(\alpha) \mathcal{W}(\mathbf{F}) d \Omega + \frac{G_{c}}{c_{w}} \int_{\Omega} \bigg(\frac{w(\alpha)}{\ell}+\ell\|\nabla \alpha\|^{2} \bigg) d \Omega
\end{split}
\end{align*}
with plane stress in the weak form (no need for stabilization terms)
\begin{align*}
\int_{\Omega} \big( a(\alpha) \mu (F_{33} - F_{33}^{-1}) \big) v_{F_{33}} dV &= 0
\end{align*}
We have assumed the modified energy functional
\begin{equation}
\widetilde{W} (\mathbf{F}, \alpha) = a (\alpha) \frac{\mu}{2} (I_c - 3 - 2 \ln J) 
\end{equation}
Therefore, we can calculate the 1st Piola Kirchoff Stress as:
\begin{align*}
\mathbf{P} &= a (\alpha) \frac{\mu}{2} (2 \mathbf{F} - \frac{2}{J} J \mathbf{F}^{-T}) \\
		&= a (\alpha) \mu (\mathbf{F} - \mathbf{F}^{-T})
\end{align*}
Taking the third component to be zero
\begin{align*}
P_{33} &= a(\alpha) \mu (F_{33} - F_{33}^{-1}) = 0 
\end{align*}
%===============================
%===============================
%===============================
\section{Following Borden: Derivations of Analytical Phase Field}
Note the full potential energy functional, which can also be called the lagrangian
\begin{align*}
\mathcal{E}_{\ell} (\boldsymbol{u}, p, \alpha) = \int_{\Omega} a(\alpha) \mathcal{W}(\mathbf{F}) d \Omega - \int_{\Omega} \frac{p^{2}}{2 \kappa} d \Omega - \int_{\Omega} \sqrt{a^{3}(\alpha)} p(J-1) d \Omega + \frac{G_{c}}{c_{w}} \int_{\Omega} \bigg(\frac{w(\alpha)}{\ell}+\ell\|\nabla \alpha\|^{2} \bigg) d \Omega
\end{align*}
We can use the Euler-Lagrange equations to arrive at the equations of motion by taking the derivative with respect to displacement, pressure, and the scalar damage field. Starting with displacement:
\begin{align*}
\frac{\partial \mathcal{E}_{\ell}}{\partial \boldsymbol{u}} &= \int_{\Omega} a (\alpha) \frac{\partial \mathcal{W} (\mathbf{F})}{\partial \mathbf{u}} d \Omega 
\end{align*}
\begin{align*}
\frac{\partial \mathcal{E}_{\ell}}{\partial p} &= - \int_{\Omega} \frac{p}{\kappa} d \Omega - \int_{\Omega} \sqrt{a^3(\alpha)} (J-1) d \Omega
\end{align*}
\begin{align*}
\frac{\partial \mathcal{E}_{\ell}}{\partial \alpha} &= - \int_{\Omega} 2 (1 - \alpha) \, \mathcal{W} (\mathbf{F}) \, d \Omega + \int_{\Omega} 3 p (1- \alpha)^2 (J-1) d \Omega + \frac{G_{c}}{c_{w}} \int_{\Omega} \bigg[ \frac{1}{\ell} + 2 \ell \nabla^2 \alpha \bigg] d \Omega 
\end{align*}
Therefore we have three equations: \\
First should be mechanical eq, second is a an equation for pressure, 
%\begin{align*}
%a (\alpha) \pdv{\mathcal{W} (\mathbf{F})}{\mathbf{u}} &= 0 \quad \text{Chain Rule} \\
%a (\alpha) \pdv{\mathcal{W}}{\mathbf{F}} \pdv{\mathbf{F}}{\mathbf{u}} &= 0 \\
%a (\alpha) \pdv{\mathcal{W}}{\mathbf{F}} \pdv{(\nabla \mathbf{u} + \mathbf{I} )}{\mathbf{u}} &= 0 \\
%a (\alpha) \mathbf{P} \pdv{}{\mathbf{u}} \bigg( \pdv{\mathbf{u}}{\mathbf{x}} \bigg) &= 0 \quad \text{unsure of final steps}
%\end{align*}
\begin{align*}
-\frac{p}{\kappa} - \sqrt{a^3(\alpha)} (J-1) &= 0 \\
-\frac{p}{\kappa} - (1-\alpha)^3 (J-1) &= 0 \\
- \kappa (J-1)(1-\alpha)^3 &= p
\end{align*}
Lastly,
\begin{align*}
- 2 (1 - \alpha) \, \mathcal{W} (\mathbf{F}) + 3 p (1- \alpha)^2 (J-1) + \frac{G_{c}}{c_{w}} \bigg[ \frac{1}{\ell} + 2 \ell \nabla^2 \alpha \bigg] &= 0 
\end{align*}
Substitute second equation into third
\begin{align*}
- 2 (1 - \alpha) \, \mathcal{W} (\mathbf{F}) - 3 \kappa (1-\alpha)^5 (J-1)^2 + \frac{G_{c}}{c_{w}} \bigg[ \frac{1}{\ell} + 2 \ell \nabla^2 \alpha \bigg] &= 0 
\end{align*}

\subsection{Homogeneous Solution}
Therefore, we can study the homogeneous solution by ignoring spatial derivatives of $\alpha$
\begin{align*}
- 2 (1 - \alpha_{h}) \, \mathcal{W} (\mathbf{F}) - 3 \kappa (1-\alpha_{h})^5 (J-1)^2 + \frac{G_{c}}{c_{w} \ell} &= 0 
\end{align*}
We can expand and arrange
\begin{align*}
- 2 \mathcal{W} (\mathbf{F}) + 2 \alpha_{h} \mathcal{W} (\mathbf{F}) - 3 \kappa(J-1)^2 (1 - 5 \alpha_h + 10 \alpha_h^2 - 10 \alpha_h^3 + 5 \alpha_h^4 - \alpha_h^5) + \frac{G_{c}}{c_{w} \ell} &= 0 \\
- 2 \mathcal{W} (\mathbf{F}) + 2 \alpha_{h} \mathcal{W} (\mathbf{F}) + \frac{G_{c}}{c_{w} \ell} \\
- 3 \kappa(J-1)^2 + 15 \kappa(J-1)^2 \alpha_h - 30 \kappa(J-1)^2 \alpha_h^2 + 30 \kappa(J-1)^2 \alpha_h^3 - 15 \kappa(J-1)^2 \alpha_h^4 + \kappa(J-1)^2 \alpha_h^5 &= 0 \\
- 2 \mathcal{W} (\mathbf{F}) - 3 \kappa(J-1)^2 + \frac{G_{c}}{c_{w} \ell} + \big[ 2 \mathcal{W} (\mathbf{F}) + 15 \kappa(J-1)^2 \big] \alpha_h \\
- 30 \kappa(J-1)^2 \alpha_h^2 + 30 \kappa(J-1)^2 \alpha_h^3 - 15 \kappa(J-1)^2 \alpha_h^4 + \kappa(J-1)^2 \alpha_h^5 &= 0
\end{align*}

\subsection{Non-Homogeneous Solution}
Now for the Non-homogenous solution, we have the following
\begin{align*}
- 2 (1 - \alpha) \, \mathcal{W} (\mathbf{F}) - 3 \kappa (1-\alpha)^5 (J-1)^2 + \frac{G_{c}}{c_{w}} \bigg[ \frac{1}{\ell} + 2 \ell \nabla^2 \alpha \bigg] &= 0 
\end{align*}


% Beginning of multiline comment
\iffalse
%===============================
%===============================
%===============================
\section{Stabilized Finite Element Method}
\noindent\rule{\linewidth}{0.5pt} % Straight line across 
From this point, the notes haven't been revised in a long time \\
\noindent\rule{\linewidth}{0.5pt} % Straight line across 

%===============================
%===============================
\subsection{Gateaux Derivative}
The Gateaux derivative with respect to $(\bm{u},\alpha)$ in direction $(\bm{v}, \beta)$ under the irreversibility condition $\dot{\alpha}\ge0$.
\begin{equation}
d\mathcal{E}_\ell \left(\bm{u}, \alpha; \bm{v}, \beta\right) \ge 0.
\end{equation}
\noindent\rule{\linewidth}{0.5pt} % Straight line across 
Calculation of the Gateaux derivative
\begin{align*}
\begin{split}
d\mathcal{E}_\ell (\bm{u}, \bm{v}) (\alpha, \beta) &= \frac{d}{d \delta} \mathcal{E}_\ell (\bm{u} + \delta \bm{v}, \alpha + \delta \beta) \big\rvert_{\delta = 0} \\
	&= \frac{d}{d \delta} \mathcal{E}_\ell (\bm{u} + \delta \bm{v}, \alpha) \big\rvert_{\delta = 0} + \frac{d}{d \delta} \mathcal{E}_\ell (\bm{u}, \alpha + \delta \beta) \big\rvert_{\delta = 0}
\end{split}
\end{align*}
Starting with the first term: 
\begin{align*}
\frac{d}{d \delta} \mathcal{E}_\ell (\bm{u} + \delta \bm{v}, \alpha) \big\rvert_{\delta = 0} 
	&= \frac{d}{d \delta} \bigg[ \int_\Omega  \widetilde{\mathcal{W}}(\mathbf{I} + \nabla (\bm{u} + \delta \bm{v}),\alpha)\,  d\Omega - \int_{\partial_N\Omega} {\tilde{\bm{g}}}_0 \cdot (\bm{u} + \delta \bm{v}) \, dA \bigg] \bigg\rvert_{\delta = 0}\\
	&= \bigg[ \int_\Omega \frac{d \widetilde{\mathcal{W}}(\mathbf{I} + \nabla (\bm{u} + \delta \bm{v}),\alpha)}{d \delta} \, d\Omega -  \int_{\partial_N\Omega} {\tilde{\bm{g}}}_0 \cdot \frac{d(\bm{u} + \delta \bm{v})}{d \delta} \, dA \bigg] \bigg\rvert_{\delta = 0} \quad \text{chain rule} \\
	&= \bigg[ \int_\Omega \frac{d \widetilde{\mathcal{W}}(\mathbf{I} + \nabla (\bm{u} + \delta \bm{v}),\alpha)}{d (\mathbf{I} + \nabla (\bm{u} + \delta \bm{v})} \frac{d (\mathbf{I} + \nabla (\bm{u} + \delta \bm{v}))}{d \delta} \, d\Omega -  \int_{\partial_N\Omega} {\tilde{\bm{g}}}_0 \cdot \frac{d(\bm{u} + \delta \bm{v})}{d \delta} \, dA \bigg] \bigg\rvert_{\delta = 0}\\
	&= \bigg[ \int_\Omega \frac{d \widetilde{\mathcal{W}}(\mathbf{I} + \nabla \bm{u} + \delta \nabla \bm{v}, \alpha)}{d (\mathbf{I} + \nabla \bm{u} + \delta \nabla \bm{v})} \frac{d (\mathbf{I} + \nabla \bm{u} + \delta \nabla \bm{v})}{d \delta} \, d\Omega -  \int_{\partial_N\Omega} {\tilde{\bm{g}}}_0 \cdot \bm{v} \, dA \bigg] \bigg\rvert_{\delta = 0}\\
	&= \bigg[ \int_\Omega \frac{d \widetilde{\mathcal{W}}(\mathbf{I} + \nabla \bm{u} + \delta \nabla \bm{v}, \alpha)}{d (\mathbf{I} + \nabla \bm{u} + \delta \nabla \bm{v})} \nabla \bm{v} \, d\Omega -  \int_{\partial_N\Omega} {\tilde{\bm{g}}}_0 \cdot \bm{v} \, dA \bigg] \bigg\rvert_{\delta = 0} \\
	&= \int_\Omega \frac{d \widetilde{\mathcal{W}}(\mathbf{I} + \nabla \bm{u}, \alpha)}{d (\mathbf{I} + \nabla \bm{u})} \nabla \bm{v} \, d\Omega -  \int_{\partial_N\Omega} {\tilde{\bm{g}}}_0 \cdot \bm{v} \, dA \\
\frac{d}{d \delta} \mathcal{E}_\ell (\bm{u} + \delta \bm{v}, \alpha) \big\rvert_{\delta = 0} 
	&= \int_\Omega \frac{d \widetilde{\mathcal{W}}(\mathbf{F}, \alpha)}{d \mathbf{F}} \nabla \bm{v} \, d\Omega -  \int_{\partial_N\Omega} {\tilde{\bm{g}}}_0 \cdot \bm{v} \, dA 
\end{align*}
Second term: 
\begin{align*}
& \frac{d}{d \delta} \mathcal{E}_\ell (\bm{u}, \alpha + \delta \beta) \big\rvert_{\delta = 0} \\
	&= \frac{d}{d \delta} \bigg[ \int_\Omega  \widetilde{\mathcal{W}}(\mathbf{F}, \alpha + \delta \beta)\,  d\Omega +\frac{\mathcal{G}_c}{c_w}\int_\Omega \left(\frac{w(\alpha + \delta \beta)}{\ell} + \ell \Vert \nabla (\alpha + \delta \beta)  \Vert^2\right) dV \bigg] \bigg\rvert_{\delta = 0} \\
	&= \bigg[ \int_\Omega  \frac{d \widetilde{\mathcal{W}}(\mathbf{F}, \alpha + \delta \beta)}{d \delta} \,  d\Omega + \frac{\mathcal{G}_c}{c_w} \frac{d}{d \delta} \int_\Omega \left(\frac{w(\alpha + \delta \beta)}{\ell} + \ell \Vert \nabla (\alpha + \delta \beta)  \Vert^2\right) dV \bigg] \bigg\rvert_{\delta = 0} \\
	&= \bigg[ \int_\Omega \frac{d \widetilde{\mathcal{W}}(\mathbf{F}, \alpha + \delta \beta)}{d (\alpha + \delta \beta)} \frac{d (\alpha + \delta \beta)}{d \delta} \, d\Omega + \frac{\mathcal{G}_c}{c_w} \frac{1}{\ell} \int_\Omega \frac{d w(\alpha + \delta \beta)}{d \delta} \, dV+ \frac{\mathcal{G}_c}{c_w} \ell \int_{\Omega} \frac{\Vert \nabla (\alpha + \delta \beta)  \Vert^2}{d \delta} \, dV \bigg] \bigg\rvert_{\delta = 0} \\
	&= \bigg[ \int_\Omega \frac{d \widetilde{\mathcal{W}}(\mathbf{F}, \alpha + \delta \beta)}{d (\alpha + \delta \beta)} \beta \, d\Omega + \frac{\mathcal{G}_c}{c_w} \frac{1}{\ell} \int_\Omega \frac{d w(\alpha + \delta \beta)}{d (\alpha + \delta \beta)} \frac{d (\alpha + \delta \beta)}{d \delta} \, dV + \frac{\mathcal{G}_c}{c_w} \ell \int_{\Omega} 2 \nabla (\alpha + \delta \beta) \frac{\nabla (\alpha + \delta \beta)}{d \delta} \, dV \bigg] \bigg\rvert_{\delta = 0} \\
	&= \int_\Omega \frac{d \widetilde{\mathcal{W}}(\mathbf{F}, \alpha)}{d \alpha} \beta \, d\Omega + \bigg[ \frac{\mathcal{G}_c}{c_w} \frac{1}{\ell} \int_\Omega \frac{d w(\alpha + \delta \beta)}{d (\alpha + \delta \beta)} \beta \, dV + \frac{\mathcal{G}_c}{c_w} \ell \int_{\Omega} 2 \nabla (\alpha + \delta \beta) \nabla \beta \, dV \bigg] \bigg\rvert_{\delta = 0} \\
	&= \int_\Omega \frac{d \widetilde{\mathcal{W}}(\mathbf{F}, \alpha)}{d \alpha} \beta \, d\Omega +  \frac{\mathcal{G}_c}{c_w} \frac{1}{\ell} \int_\Omega \frac{d w(\alpha)}{d \alpha} \beta \, dV + 2 \ell \frac{\mathcal{G}_c}{c_w} \int_{\Omega} \nabla \alpha \cdot \nabla \beta \, dV \\
	&= \int_\Omega \frac{d \widetilde{\mathcal{W}}(\mathbf{F}, \alpha)}{d \alpha} \beta \, d\Omega + \frac{\mathcal{G}_c}{c_w \ell} \int_\Omega \bigg[ \frac{d w(\alpha)}{d \alpha} \beta + 2 \ell^2 (\nabla \alpha \cdot \nabla \beta) \bigg] \, dV 
\end{align*}
\noindent\rule{\linewidth}{0.5pt} % Straight line across 
First, consider Eq. \ref{eq_pressure}
\begin{align*}
p &= -\sqrt{a^3(\alpha)} \kappa (J - 1) \\
\frac{p}{\kappa} &= -\sqrt{a^3(\alpha)} (J - 1) \\
0 &= - \sqrt{a^3(\alpha)} (J - 1) - \frac{p}{\kappa} 
\end{align*}
Multiplying this by test function $q$ and integrating over volume, we obtain an equation that can be combined with the equations from the Gateaux Derivative. 
\begin{subequations}\label{WeakForm}
\begin{align}
\int_\Omega \frac{d \widetilde{\mathcal{W}}(\mathbf{F}, \alpha)}{d \mathbf{F}} \nabla \bm{v} \, d\Omega -  \int_{\partial_N\Omega} {\tilde{\bm{g}}}_0 \cdot \bm{v} \, dA &= 0 \\
\int_\Omega \bigg( - \sqrt{a^3(\alpha)} (J - 1) - \frac{p}{\kappa} \bigg) q dV &= 0 \\
\int_\Omega \frac{d \widetilde{\mathcal{W}}(\mathbf{F}, \alpha)}{d \alpha} \beta \, d\Omega + \frac{\mathcal{G}_c}{c_w \ell} \int_\Omega \bigg[ \frac{d w(\alpha)}{d \alpha} \beta + 2 \ell^2 (\nabla \alpha \cdot \nabla \beta) \bigg] \, dV &\geq 0 
\end{align}
\end{subequations}
The strong form
\begin{subequations}\label{StrongForm}
\begin{align}
\text{Div} \, \mathbf{P} = 0 \quad &\text{in} \quad \Omega \\
\mathbf{u} = \widetilde{\mathbf{u}}_0 \quad &\text{in} \quad \partial_{D}\Omega \\
\left[\mathbf{FS}\right]\bm{n} = {\tilde{\bm{g}}}_0 \quad &\text{on} \quad \partial_{N}\Omega,
\end{align}
\end{subequations}
where from Eq. \ref{eq_firstPK} we can substitute Eq. \ref{eq_pressure}
\begin{align*}
\begin{split}
\mathbf{P} &= a(\alpha) \frac{\mathcal{W}(\mathbf{F})}{\partial \mathbf{F}} + a^3(\alpha)\kappa\left(J-1\right)\frac{\partial J}{\partial \mathbf{F}} \quad \text{where} \, p=-\sqrt{a^3(\alpha)}\kappa\left(J-1\right) \\
\mathbf{P} &= a(\alpha) \frac{\mathcal{W}(\mathbf{F})}{\partial \mathbf{F}} - p \sqrt{a^3(\alpha)} \frac{\partial J}{\partial \mathbf{F}} 
\end{split}
\end{align*}
and write the mechanical equilibrium equation in Eq. \ref{StrongForm}:
\begin{equation}\label{Eqm}
\text{Div} \, \bigg[ a(\alpha) \frac{\mathcal{W}(\mathbf{F})}{\partial \mathbf{F}} - p \sqrt{a^3(\alpha)} \frac{\partial J}{\partial \mathbf{F}} \bigg] = 0 
\end{equation}
\noindent\rule{\linewidth}{0.5pt} % Straight line across 
Derivation of the KKT condition equations where $\nabla \beta \cdot \nabla \alpha = \nabla (\beta \nabla \alpha) - \beta \Delta \alpha$
\begin{align*}
\int_\Omega \frac{d \widetilde{\mathcal{W}}(\mathbf{F}, \alpha)}{d \alpha} \beta \, d\Omega + \frac{\mathcal{G}_c}{c_w \ell} \int_\Omega \bigg[ \frac{d w(\alpha)}{d \alpha} \beta + 2 \ell^2 (\nabla \alpha \cdot \nabla \beta) \bigg] \, dV &\geq 0 \\
\int_\Omega \frac{d \widetilde{\mathcal{W}}(\mathbf{F}, \alpha)}{d \alpha} \beta \, d\Omega + \frac{\mathcal{G}_c}{c_w \ell} \int_\Omega \frac{d w(\alpha)}{d \alpha} \beta dV + \frac{\mathcal{G}_c}{c_w \ell} \int_{\Omega} 2 \ell^2 (\nabla \alpha \cdot \nabla \beta)  \, dV &\geq 0 \\
\int_\Omega \frac{d \widetilde{\mathcal{W}}(\mathbf{F}, \alpha)}{d \alpha} \beta \, d\Omega +  \frac{\mathcal{G}_c}{c_w \ell} \int_\Omega \frac{d w(\alpha)}{d \alpha} \beta dV + \frac{\mathcal{G}_c}{c_w \ell} \int_{\Omega} 2 \ell^2 ( \nabla (\beta \nabla \alpha))  \, dV - \frac{\mathcal{G}_c}{c_w \ell} \int_{\Omega} 2 \ell^2 (\beta \Delta \alpha)  \, dV &\geq 0 \\
\int_\Omega \frac{d \widetilde{\mathcal{W}}(\mathbf{F}, \alpha)}{d \alpha} \beta \, d\Omega +  \frac{\mathcal{G}_c}{c_w \ell} \int_\Omega \frac{d w(\alpha)}{d \alpha} \beta dV  - \frac{\mathcal{G}_c}{c_w \ell} \int_{\Omega} 2 \ell^2 (\beta \Delta \alpha)  \, dV &\geq 0 \\
\bigg[ \int_\Omega \frac{d \widetilde{\mathcal{W}}(\mathbf{F}, \alpha)}{d \alpha} \, d\Omega +  \frac{\mathcal{G}_c}{c_w \ell} \int_\Omega \bigg( \frac{d w(\alpha)}{d \alpha}  - 2 \ell^2 \Delta \alpha\bigg) dV \bigg] \beta &\geq 0 \\
\int_\Omega \frac{d \widetilde{\mathcal{W}}(\mathbf{F}, \alpha)}{d \alpha} \, d\Omega +  \frac{\mathcal{G}_c}{c_w \ell} \int_\Omega \bigg( \frac{d w(\alpha)}{d \alpha}  - 2 \ell^2 \Delta \alpha\bigg) dV &\geq 0 
\end{align*}
\noindent\rule{\linewidth}{0.5pt} % Straight line across 
Grouping terms, we obtain
\begin{subequations}
\begin{align}
\frac{d \widetilde{\mathcal{W}}(\mathbf{F}, \alpha)}{d \alpha} + \frac{\mathcal{G}_c}{c_w \ell}  \bigg( \frac{d w(\alpha)}{d \alpha}  - 2 \ell^2 \Delta \alpha\bigg) &\geq 0 \quad \text{in} \quad \Omega \\
\dot{\alpha} \geq 0 \quad \text{in} \quad \Omega \\
\dot{\alpha} \bigg[ \frac{d \widetilde{\mathcal{W}}(\mathbf{F}, \alpha)}{d \alpha} + \frac{\mathcal{G}_c}{c_w \ell}  \bigg( \frac{d w(\alpha)}{d \alpha}  - 2 \ell^2 \Delta \alpha\bigg) \bigg] &\geq 0 \quad \text{in} \quad \Omega \\
\end{align}
\end{subequations}
Lastly, we have the following, (Neumann?)
\begin{equation}
\frac{\partial \alpha}{\partial \mathbf{n}} \geq 0 \quad \text{and} \quad \dot{\alpha} \frac{\partial \alpha}{\partial \mathbf{n}} = 0 \quad \text{on} \quad \partial \Omega
\end{equation}
\noindent\rule{\linewidth}{0.5pt} % Straight line across 
Multiply Eq. \ref{Eqm} with weighting function $\bm{v}+(\varpi h^2)/(2\mu)\mathbf{F}^{-T}\nabla q$
\begin{align*}
\int_{\Omega} \text{Div} \mathbf{P} \cdot \bigg[ \bm{v} + \frac{\varpi h^2}{2\mu} \mathbf{F}^{-T}\nabla q \bigg] \, dV &= 0 \\
\int_{\Omega} \text{Div} \mathbf{P} \cdot \bm{v} \, dV + \int_{\Omega} \text{Div} \mathbf{P} \cdot \bigg[ \frac{\varpi h^2}{2\mu} \mathbf{F}^{-T}\nabla q \bigg] \, dV &= 0 \\
\int_{\Omega} \text{Div} \mathbf{P} \cdot \bm{v} \, dV + \frac{\varpi h^2}{2\mu} \sum_{e=1}^{n_{el}}\int_{\Omega^{e}} \text{Div} \mathbf{P} \cdot \left(\mathbf{F}^{-T}\nabla q\right) \,dV &= 0 \\
\int_{\Omega} \text{Div} \mathbf{P} \cdot \bm{v} \, dV &\\
+ \frac{\varpi h^2}{2\mu} \sum_{e=1}^{n_{el}}\int_{\Omega^{e}} \text{Div} \bigg[ a(\alpha) \frac{\mathcal{W}(\mathbf{F})}{\partial \mathbf{F}} \bigg] \cdot \left(\mathbf{F}^{-T} \nabla q \right) \,dV - \frac{\varpi h^2}{2\mu} \sum_{e=1}^{n_{el}}\int_{\Omega^{e}} \text{Div} \bigg[ p \sqrt{a^3(\alpha)} \frac{\partial J}{\partial \mathbf{F}} \bigg] \cdot \left(\mathbf{F}^{-T} \nabla q \right) \,dV &= 0 \\
"" + "" - \frac{\varpi h^2}{2\mu} \sum_{e=1}^{n_{el}}\int_{\Omega^{e}} \nabla \big( p \sqrt{a^3(\alpha)} \big) J \mathbf{F}^{-T} \cdot \left(\mathbf{F}^{-T}\nabla q \right) \,dV &= 0 \\
"" + "" - \frac{\varpi h^2}{2\mu} \sum_{e=1}^{n_{el}}\int_{\Omega^{e}} \nabla \big( p \sqrt{a^3(\alpha)} \big) J \mathbf{F}^{-1} \mathbf{F}^{-T} \cdot \left( \mathbf{F}^{-1} \mathbf{F}^{-T}\nabla q \right) \,dV &= 0 \\
"" + "" - \frac{\varpi h^2}{2\mu} \sum_{e=1}^{n_{el}}\int_{\Omega^{e}} \nabla \big( p \sqrt{a^3(\alpha)} \big) J \cdot \left( \mathbf{C}^{-1} \nabla q \right) \,dV &= 0 \\
"" + "" - \frac{\varpi h^2}{2\mu} \sum_{e=1}^{n_{el}}\int_{\Omega^{e}} J \mathbf{C}^{-1} : \bigg[ \nabla \big( p \sqrt{a^3(\alpha)} \big) \cdot \nabla q \bigg] \,dV &= 0 
\end{align*}
where $\mathbf{P} = a(\alpha) \frac{\mathcal{W}(\mathbf{F})}{\partial \mathbf{F}} - p \sqrt{a^3(\alpha)} \frac{\partial J}{\partial \mathbf{F}}$ \\ \\
We also want to deal with the first term where $(fg)' = f' g + f g'$
\begin{align*}
\int_{\Omega} \text{Div} \mathbf{P} \cdot \bm{v} \, dV = \int_{\Omega} (\mathbf{F} \cdot \bm{v})_{,X} \, dV - \int_{\Omega} \mathbf{P} \cdot \frac{\partial \bm{v}}{\partial X} \, dV \\
\end{align*}
\noindent\rule{\linewidth}{0.5pt} % Straight line across 


Leaving
\begin{align}
\begin{split}
\int_{\Omega} \text{Div} \mathbf{P} \cdot \bm{v} \, dV 
+ \frac{\varpi h^2}{2\mu} \sum_{e=1}^{n_{el}}\int_{\Omega^{e}} \text{Div} \bigg[ a(\alpha) \frac{\mathcal{W}(\mathbf{F})}{\partial \mathbf{F}} \bigg] \cdot \left(\mathbf{F}^{-T}\nabla q \right) \,dV 
- \frac{\varpi h^2}{2\mu} \sum_{e=1}^{n_{el}}\int_{\Omega^{e}} J \mathbf{C}^{-1} : \bigg[ \nabla \big( p \sqrt{a^3(\alpha)} \big) \cdot \nabla q \bigg] \,dV = 0 
\end{split}
\end{align}
\fi
%\section{Numerical Examples}
%
%\subsection{Benchmark: Uniaxial Tension of a Hyperelastic Bar}
%
%\subsection{Revisiting Crack Nucleation in an Elastomer}


\begin{equation*}
\end{equation*}

\begin{equation}
\end{equation}

\end{document}