\documentclass[12pt,3p]{article}
\usepackage[utf8]{inputenc}
\usepackage[english]{babel}
 \usepackage[margin=0.5in]{geometry}
\usepackage{amsmath}
\usepackage{mathtools}
\usepackage{enumitem}
\usepackage{physics}
\usepackage[round,numbers]{natbib}
\usepackage[colorlinks = false]{hyperref}
\usepackage{tcolorbox}
\usepackage{xcolor}
\usepackage{amsmath}
\usepackage{cleveref} % Refer to range of equations
\usepackage{stmaryrd} % For double brackets for M jump
%This line makes .eps figures into .pdf
\usepackage{epstopdf}
\usepackage{bm}

\setcounter{tocdepth}{4} 
\setcounter{secnumdepth}{4}

\numberwithin{equation}{section}
\begin{document}

\title{Project: Surface Tension on Sphere \\
	\large{Updated: 03/06/20}}
\author{Ida Ang}
\date{\vspace{-5ex}}
\maketitle

\tableofcontents
\newpage

\section{Definitions}
$f_i$: nominal body force per unit volume \\
$T_i$: nominal traction on surface [force per area] \\
$\sigma_{ij}$: true stress $\rightarrow$ $S_{iJ}$: nominal stress 
\begin{equation*}
\bigg[ \frac{\text{Force}}{\text{Area}} \bigg] 
\end{equation*}
U: free energy of the gel \\
	\indent 
	$U_e$: elastic energy from stretching a polymer network \\
	\indent
	$U_m$: energy of mixing the solvent with polymer chains
\begin{equation*}
\bigg[ \frac{\text{Force}}{\text{Area}} \bigg] 
\end{equation*}
N: number of polymer chains per unit volume in the reference state 
\begin{equation*}
\bigg[ \frac{\text{Number of polymer chains}}{\text{Volume}} \bigg]
\end{equation*}
$k_B$: Boltzmann's Constant
\begin{equation*}
\bigg[ \frac{\text{Energy}}{\text{Temperature}} \bigg] = \bigg[ \frac{\text{Force} \times \text{Length}}{\text{Temperature}} \bigg] 
\end{equation*}
T: [Temperature] \\
K: Shear Modulus (Same units as $N k_B T$)
\begin{equation*}
\bigg[\text{Number of Polymer Chains} \times \frac{\text{Force}}{\text{Area}} \bigg] 
\end{equation*}
$\chi$: Flory-Huggins parameter for the enthalpy of mixing [dimensionless] \\
\textbf{F}: Deformation gradient [dimensionless] \\
C(\textbf{X}): Nominal concentration of solvent per unit volume in reference state \\
c(\textbf{X},t): True concentration of solvent per unit volume in current state \\
\begin{equation*}
\bigg[ \frac{\text{Concentration of Solvent}}{\text{Volume}} \bigg] 
\end{equation*}
$\Omega$: [Volume] \\
$\mu$(\textbf{X},t): chemical potential of the small molecules 
\begin{equation*}
\bigg[ \frac{\text{Energy}}{\text{Concentration}} \bigg] = \bigg[ \frac{\text{Force} \times \text{Length}}{\text{Concentration}} \bigg] 
\end{equation*}
dA(\textbf{X}): element of area [Area] $\rightarrow$ dV(\textbf{X}): element of volume [Volume] \\
$N_K (\textbf{X})$: unit vector normal to the interface in the reference state $\rightarrow$ $n_i (\textbf{X}, t)$: current state [Dimensionless] \\
Number of small molecules injected into: \\
	\indent r(\textbf{X},t): a volume element 
	\begin{equation*}
	\bigg[ \frac{\text{Number of small molecules}}{\text{Volume} \times \text{Time}} \bigg] 
	\end{equation*}
	\indent i(\textbf{X},t): an interface element 
	\begin{equation*}
	\bigg[ \frac{\text{Number of small molecules}}{\text{Area} \times \text{Time}} \bigg] 
	\end{equation*}
D: Solvent diffusivity 
\begin{equation*}
\bigg[ \frac{\text{Area}}{\text{Time}} \bigg]
\end{equation*}
$J_K$: nominal flux $\rightarrow j_i(\mathbf{X},t)$: true flux 
\begin{equation*}
\bigg[ \frac{\text{Number of Small Molecules}}{\text{Time} \times \text{Area}} \bigg] 
\end{equation*}
$M_{KL}$: Nominal mobility tensor
\begin{equation*}
\bigg[ \frac{\text{Number of Small Molecules}}{\text{Time} \times \text{Volume}} \bigg]
\end{equation*}

\newpage

%========================================================================
%========================================================================
%========================================================================
\section{Reformulation}
References: \\
A. Javili and P. Steinmann titled \textit{A finite element framework for continua with boundary energies. Part I: The two-dimensional case} \\
A. Javili and P. Steinmann titled \textit{A finite element framework for continua with boundary energies. Part II: The three-dimensional case} 

%% ==================================================================
%% ==================================================================
\subsection{Kinematics and energies of continua with boundaries}
Surfaces of bodies and interfaces between bodies exhibit different properties from the bulk. These effects can be modeled in terms of surface tension or an added term to the total potential energy, denoted by a bar 
\begin{align*} 
\text{Bulk}&: U = U_e + U_m - \mu C \\
\text{Surface}&: u_o
\end{align*}
Surface deformation gradient:
\begin{align}\label{deformationSurface}
\begin{split}
\bar{\mathbf{F}} &= \pdv{x_i}{X_K} \cdot \bar{\mathbf{I}} \\
\bar{\mathbf{F}} &= \mathbf{F} \cdot \bar{\mathbf{I}}
\end{split}
\end{align}
where the mixed-variant surface unit tensor in the material configuration is:
\begin{align}\label{surfaceUnitTensor}
\bar{\mathbf{I}} = \mathbf{I} - \mathbf{N} \otimes \mathbf{N}
\end{align}
Just as how the Jacobian is related to the volume, the surface J is related to the surface: 
\begin{align*}
J &= \frac{dv}{dV} = \det \mathbf{F} \\
\bar{J} &= \frac{da}{dA} = | \underbrace{\det \mathbf{F} \mathbf{F}^{-T}}_\text{cofactor} \mathbf{N} | 
\end{align*}
Can use the Nanson operator to define the surface jacobian, $\bar{J}$
\begin{align}\label{surfaceJacobian}
\begin{split}
\bar{J} &= | J \mathbf{F}^{-T} \mathbf{N} | \\
\bar{J} &= | \underbrace{\det \mathbf{F} \mathbf{F}^{-T}}_\text{cofactor} \mathbf{N} | 
\end{split}	   
\end{align}

%% ==================================================================
\subsubsection{Neo-Hookean type boundary potential}
The internal potential energy can be expressed as: 
\begin{align}\label{neoHookeanPotEnergy}
w_0 (\mathbf{\bar{F}}) = \frac{\bar{\lambda}}{2} \log^2 \bar{J} + \frac{\bar{\mu}}{2} (\mathbf{\bar{F}} : \mathbf{\bar{F}} - 2 - 2 \log \bar{J})
\end{align}

%% ==================================================================
\subsubsection{Surface tension boundary potential}
The internal potential energy can be expressed as: 
\begin{align}\label{STPotEnergy}
w_0 (\mathbf{\bar{F}}) = \gamma \bar{J}
\end{align}


%========================================================================
%========================================================================
\subsection{Equilibrium Equations (Strong Forms)}
Mechanical equilibrium equation: 
\begin{align}
\begin{split}
\nabla_{X} \cdot \mathbf{P}+\mathbf{b}_{\mathbf{0}} &= \mathbf{0} \quad \text { in } \quad V\\
\mathbf{P} \cdot \mathbf{N}-\overline{\boldsymbol{\nabla}}_{\boldsymbol{X}} \cdot \overline{\mathbf{P}}&=\check{\mathbf{T}} \quad \text { on } \quad S^{T}\\
\mathbf{u} &= \check{\mathbf{u}} \quad \text { on } \quad S^{u}\\
\llbracket \overline{\mathbf{P}} \cdot \overline{\mathbf{N}} \rrbracket &=0 \quad \text { on } \quad L
\end{split}
\end{align}
Mass conservation equation: 
\begin{align}
\begin{split}
\dot{C}+\nabla_{X} \cdot \mathbf{J} &=r \quad \text { in } \quad V \\
\mathbf{J} \cdot \mathbf{N} &=-i \quad \text { on } \quad S^{i} \\
\mu &= \check{\mu} \quad \text { on } \quad S^{\mu}
\end{split}
\end{align}

%========================================================================
\subsubsection{Mechanical Equilibrium} 
Rearrange and multiply with a test function $\delta \boldsymbol\varphi$ and integrate over domain 
\begin{align*}
\int (\div \mathbf{P}) \cdot \delta \boldsymbol\varphi dV + \int \mathbf{b_o} \cdot \delta \boldsymbol\varphi dV &= 0 \quad \text{Int by Parts} \\
\int \div (\mathbf{P^T} \cdot \delta \boldsymbol\varphi) dV - \int \mathbf{P} \cdot \grad \delta \boldsymbol\varphi dV+ \int \mathbf{b_o} \cdot \delta \varphi dV &= 0 \quad \text{Divergence} \\
\int (\mathbf{P} \cdot \mathbf{N}) \cdot \delta \boldsymbol\varphi dS - \int \mathbf{P} \cdot \grad \delta \boldsymbol\varphi dV+ \int \mathbf{b_o} \cdot \delta \boldsymbol\varphi dV &= 0 \quad \rightarrow \mathbf{P} \cdot \mathbf{N} = \bar{\div} \mathbf{\bar{P}} + \mathbf{\bar{b}_o} \\
\int (\bar{\div} \mathbf{\bar{P}}) \cdot \delta \boldsymbol\varphi dS + \int \mathbf{\bar{b}_o} \cdot \delta \boldsymbol\varphi dS - \int \mathbf{P} \cdot \grad \delta \boldsymbol\varphi dV+ \int \mathbf{b_o} \cdot \delta \boldsymbol\varphi dV &= 0 \quad \text{Int by Parts}\\
\int \bar{\div} (\mathbf{\bar{P}}^T \cdot \delta \boldsymbol\varphi) dS - \int \mathbf{\bar{P}} \cdot \bar{\nabla} \delta \boldsymbol\varphi dS + \int \mathbf{\bar{b}_o} \cdot \delta \boldsymbol\varphi dS - \int \mathbf{P} \cdot \grad \delta \boldsymbol\varphi dV+ \int \mathbf{b_o} \cdot \delta \boldsymbol\varphi dV &= 0 \quad \text{Divergence} \\
\int (\mathbf{\bar{P}} \cdot \mathbf{M}) \cdot \delta \boldsymbol\varphi dL - \int \mathbf{\bar{P}} \cdot \bar{\nabla} \delta \boldsymbol\varphi dS + \int \mathbf{\bar{b}_o} \cdot \delta \boldsymbol\varphi dS - \int \mathbf{P} \cdot \grad \delta \boldsymbol\varphi dV+ \int \mathbf{b_o} \cdot \delta \boldsymbol\varphi dV &= 0 \\
- \int \mathbf{\bar{P}} \cdot \bar{\nabla} \delta \boldsymbol\varphi dS + \int \mathbf{\bar{b}_o} \cdot \delta \boldsymbol\varphi dS - \int \mathbf{P} \cdot \grad \delta \boldsymbol\varphi dV+ \int \mathbf{b_o} \cdot \delta \boldsymbol\varphi dV &= 0 
\end{align*}
Rearrange 
\begin{equation}\label{MechWeakForm}
\int \mathbf{\bar{P}} \cdot \bar{\nabla} \delta \boldsymbol\varphi dS + \int \mathbf{P} \cdot \grad \delta \boldsymbol\varphi dV = \int \mathbf{\bar{b}_o} \cdot \delta \boldsymbol\varphi dS + \int \mathbf{b_o} \cdot \delta \boldsymbol\varphi dV
\end{equation}

%========================================================================
\subsubsection{Mass Conservation}
Multiply Eq. \ref{chemCon} with a test function $\mu$ and follow the same steps as above: 
\begin{align*}
\int {\dot{C}} \delta \mu dV + \int ( \div \mathbf{J}) \delta \mu dV &= \int r \delta \mu dV \quad \text{Int by Parts} \\ 
\int {\dot{C}} \delta \mu dV + \int \div (\mathbf{J}^T \delta \mu) dV - \int \mathbf{J} \cdot \grad \delta \mu dV &= \int r \delta \mu dV \quad \text{Divergence} \\
 \int {\dot{C}} \delta \mu dV + \int (\mathbf{J} \cdot \mathbf{N}) \delta \mu dV - \int \mathbf{J} \cdot \grad \delta \mu dV &= \int r \delta \mu dV \\
 \int {\dot{C}} \delta \mu dV - \int i \delta \mu dS - \int \mathbf{J} \cdot \grad \delta \mu dV &= \int r \delta \mu dV \\
 \int {\dot{C}} \delta \mu dV - \int \mathbf{J} \cdot \grad \delta \mu dV &= \int r \delta \mu dV + \int i \delta \mu dS 
\end{align*}

%========================================================================
%========================================================================
\subsection{Thermodynamic Theory}
Free energy density
\begin{align*}
U (\mathbf{F}, \mu) &= U_e (\mathbf{F}) + U_m (C) - \mu C \\
\bar{U} (\mathbf{\bar{F}}) &= u_e  (\mathbf{\bar{F}}) 
\end{align*}
The free energy density of the system changes at the rate: 
\begin{align*}
\delta U (\mathbf{F}, \mu) &= \pdv{U (\mathbf{F}, \mu)}{\mathbf{F}} \delta \mathbf{F}+ \pdv{U (\mathbf{F}, \mu)}{\mu} \delta \mu \\
\delta \bar{U} (\mathbf{\bar{F}}) &=  \pdv{u (\mathbf{\bar{F}})}{\mathbf{\bar{F}}} \delta \mathbf{\bar{F}}
\end{align*}
Taking in the sum 
\begin{align*}
 \frac{\delta G_{bulk}}{\delta t} &= \int \frac{\delta U}{\delta t} dV  - \int \mathbf{b_o} \frac{\delta \mathbf{x}}{\delta t} dV - \int \mu r dV - \int \mu i dS \\
\frac{\delta G_{surf}}{\delta t} &= \int \frac{\delta \bar{U}}{\delta t} dS - \int_{S} \check{\mathbf{T}} \frac{\delta \overline{\mathbf{x}}}{\delta t} d S 
\end{align*}
Together the gel and field of weights and pumps form a thermodynamic system
\begin{align*}
\frac{\delta G}{\delta t} &= \frac{\delta G_{bulk}}{\delta t} + \frac{\delta G_{surf}}{\delta t} \\
\frac{\delta G}{\delta t} &= \int_{V} \frac{\delta U}{\delta t} d V + \int_{S} \frac{\delta \bar{U}}{\delta t} d S - 
\underbrace{\int_{V} \mathbf{b}_{\mathbf{0}} \frac{\delta \mathbf{x}}{\delta t} d V - \int_{S} \check{\mathbf{T}} \frac{\delta \overline{\mathbf{x}}}{\delta t} d S}_\text{Rate of mechanical work}
- \underbrace{\int_{V} \mu r d V-\int_{S} \mu i d S}_\text{Rate of chemical work} 
\end{align*}
The change in free energy density over time 
\begin{align*}
\pdv{U}{t} &= \pdv{U}{\mathbf{F}} \frac{\delta \mathbf{F}}{\delta t}+ \pdv{U}{\mu} \frac{\delta \mu}{\delta t} + \frac{\delta (\mu C)}{\delta t} \\
\pdv{u}{t} &= \pdv{u}{\mathbf{\bar{F}}} \frac{\delta \mathbf{\bar{F}}}{\delta t}
\end{align*}
Rephrase the two weak forms where $\delta u$ is equivalent to $\delta \mathbf{x} / \delta t$ or $\delta \mathbf{\bar{x}} / \delta t $
\begin{align*}
\int \mathbf{\bar{P}} \cdot \bar{\nabla} \frac{\delta \mathbf{\bar{x}}}{\delta t} dS  &= \int \mathbf{\bar{b}_o} \cdot \frac{\delta \mathbf{\bar{x}}}{\delta t} dS 
\quad \rightarrow \quad 
\int \mathbf{\bar{P}} \cdot \frac{\delta \mathbf{\bar{F}}}{\delta t} dS = \int \mathbf{\bar{b}_o} \cdot \frac{\delta \mathbf{\bar{x}}}{\delta t} dS \\
\int \mathbf{P} \cdot \grad \frac{\delta \mathbf{x}}{\delta t} dV &= \int \mathbf{b_o} \cdot \frac{\delta \mathbf{x}}{\delta t} dV 
\quad \rightarrow \quad 
\int \mathbf{P} \cdot \frac{\delta \mathbf{F}}{\delta t} dV = \int \mathbf{b_o} \cdot \frac{\delta \mathbf{x}}{\delta t} dV
\end{align*}
Rewrite the weak form from $\delta \mu$ is $\mu$
\begin{align*}
\int \pdv{C}{t}  \mu dV - \int r \mu dV - \int i \mu dS = \int \mathbf{J} \cdot \grad \mu dV
\end{align*}
Bulk 
\begin{align*}
 \frac{\delta G_{bulk}}{\delta t} &= \int \frac{\delta U}{\delta t} dV  - \int \mathbf{b_o} \frac{\delta \mathbf{x}}{\delta t} dV - \int \mu r dV - \int \mu i dS \\
 	&= \int \pdv{U}{\mathbf{F}} \frac{\delta \mathbf{F}}{\delta t} dV + \int \pdv{U}{\mu} \frac{\delta \mu}{\delta t} dV + \int \frac{\delta (\mu C)}{\delta t} dV - \int \mathbf{b_o} \frac{\delta \mathbf{x}}{\delta t} dV - \int \mu r dV - \int \mu i dS \\
	 &= \int \pdv{U}{\mathbf{F}} \frac{\delta \mathbf{F}}{\delta t} dV + \int \pdv{U}{\mu} \frac{\delta \mu}{\delta t} dV + \int \frac{\delta \mu}{\delta t} C dV + \int \mu \frac{\delta C}{\delta t} dV- \int \mathbf{P} \cdot \frac{\delta \mathbf{F}}{\delta t} dV - \int \mu r dV - \int \mu i dS \\
	 &= \int \bigg( \pdv{U}{\mathbf{F}} - \mathbf{P} \bigg) \frac{\delta \mathbf{F}}{\delta t} dV + \int \bigg( \pdv{U}{\mu} + C \bigg) \frac{\delta \mu}{\delta t} dV + \int \mu \frac{\delta C}{\delta t} dV - \int \mu r dV - \int \mu i dS \\
 \frac{\delta G_{bulk}}{\delta t} &= \int \bigg( \pdv{U}{\mathbf{F}} - \mathbf{P} \bigg) \frac{\delta \mathbf{F}}{\delta t} dV + \int \bigg( \pdv{U}{\mu} + C \bigg) \frac{\delta \mu}{\delta t} dV + \int \mathbf{J} \cdot \grad \mu dV
\end{align*}
Surface
\begin{align*}
\frac{\delta G_{surf}}{\delta t} &= \int \frac{\delta u}{\delta t} dS - \int \mathbf{\bar{b}_o} \frac{\delta \bar{x}}{\delta t} dS \\
	&= \int \pdv{u}{\mathbf{\bar{F}}} \frac{\delta \mathbf{\bar{F}}}{\delta t} dS - \int \mathbf{\bar{b}_o} \frac{\delta \bar{x}}{\delta t} dS \\
	&= \int \pdv{u}{\mathbf{\bar{F}}} \frac{\delta \mathbf{\bar{F}}}{\delta t} dS - \int \mathbf{\bar{P}} \cdot \frac{\delta \mathbf{\bar{F}}}{\delta t} dS \\
\frac{\delta G_{surf}}{\delta t} &= \int \bigg( \pdv{u}{\mathbf{\bar{F}}} - \mathbf{\bar{P}} \bigg)  \frac{\delta \mathbf{\bar{F}}}{\delta t} dS 
\end{align*}
Therefore 
\begin{align*}
\frac{\delta G}{\delta t} = \int \bigg( \pdv{U}{\mathbf{F}} - \mathbf{P} \bigg) \frac{\delta \mathbf{F}}{\delta t} dV + \int \bigg( \pdv{u}{\mathbf{\bar{F}}} - \mathbf{\bar{P}} \bigg)  \frac{\delta \mathbf{\bar{F}}}{\delta t} dS + \int \bigg( \pdv{U}{\mu} + C \bigg) \frac{\delta \mu}{\delta t} dV + \int \mathbf{J} \cdot \pdv{\mu}{X} dV
\end{align*}
Thermodynamics dictates that the free energy of the system should never increase: 
\begin{align}\label{2ndLawThermo}
\frac{\delta G}{\delta t} \leq 0 
\end{align}
This leads to
\begin{align*}
\mathbf{P} &= \pdv{U}{\mathbf{F}} \\
\mathbf{\bar{P}} &= \pdv{u}{\mathbf{\bar{F}}} \\
C &= -\pdv{U}{\mu}
\end{align*}
We enforce the negative definite structure of the last integrand by adopting a kinetic law
\begin{equation}\label{kinLaw}
\mathbf{J} = - \mathbf{M} \cdot \pdv{\mu}{X} = - \mathbf{M} \cdot \grad \mu
\end{equation}

%========================================================================
%========================================================================
\subsection{Kinetic Law}
The flux relates to the gradient of the chemical potential
\begin{equation}\label{25}
\mathbf{j} = - \frac{cD}{kT} \pdv{\mu}{x}
\end{equation}
The true concentration relates to the nominal concentration as: 
\begin{equation}\label{26}
c = \frac{C}{\det \mathbf{F}}
\end{equation}
Recall an identity 
\begin{equation}\label{identity2}
\det (\mathbf{F}) \mathbf{N} dA = \mathbf{F} \mathbf{n} da \rightarrow \frac{\mathbf{N} dA}{\mathbf{n} da} = \frac{\mathbf{F}}{\det \mathbf{F}}
\end{equation}
The number of molecules crossing the material element per unit time can be written:
\begin{equation}\label{27}
\mathbf{j} \cdot \mathbf{n} da = \mathbf{J} \cdot \mathbf{N} dA
\end{equation}
Consequently, relate the true flux to the nominal flux.
\begin{align*}
\mathbf{j} &= \mathbf{J} \cdot \frac{\mathbf{N} dA}{\mathbf{n} da} \quad \text{substitute Eq. \ref{identity2}}\\
    &= \mathbf{J} \frac{\mathbf{F}}{\det \mathbf{F}} \quad \text{substitute Eq. \ref{kinLaw}} \\
    &= -\mathbf{M} \cdot \pdv{\mu}{X} \frac{\mathbf{F}}{\det \mathbf{F}} \quad \text{use chain rule} \\
    &= - \mathbf{M} \pdv{\mu}{x} \pdv{x}{X} \frac{\mathbf{F}}{\det \mathbf{F}} \quad \text{use Eq. \ref{25}} \\
- \frac{cD}{k_B T} \pdv{\mu}{x} &= - \mathbf{M} \pdv{\mu}{x} \pdv{x}{X} \cdot \pdv{x}{X} \frac{1}{\det \mathbf{F}} \\
- \frac{cD}{k_B T} \det \mathbf{F} &= - \mathbf{M} \pdv{x}{X} \cdot \pdv{x}{X} \\
- \frac{CD}{k_B T} &= - \mathbf{M} \pdv{x}{X} \cdot \pdv{x}{X} \\
\end{align*}
Rearrange in terms of $M_{KL}$, use Eq. \ref{26} and invoke incompressibility
\begin{align*}
\mathbf{M} &= \frac{CD}{k_B T} \mathbf{F}^{-1} \mathbf{F}^{-T} 
\end{align*}

%========================================================================
%========================================================================
\subsection{Strain Energy Density}
The constituents of the hydrogel are assumed to be incompressible: 
\begin{equation}\label{incompressible}
\det \mathbf{F} = 1 + \Omega C 
\end{equation}
Free energy due to stretching (Flory, 1953)
\begin{align}\label{Stretch}
\begin{split}
U_e (\boldsymbol{F}) &=  \frac{1}{2} N k_B T \big[ \tr \mathbf{C} - 3 - 2 \log (\det \mathbf{F}) \big] 
\end{split}
\end{align}
Flory-Huggins (Flory, 1942; Huggins, 1941) model for the energy of mixing 
\begin{align}\label{Mix}
\begin{split}
U_m(C) &= - \frac{k_B T}{\Omega} \bigg[ \Omega C \log(\frac{1 + \Omega C}{\Omega C}) + \frac{\chi}{1 + \Omega C} \bigg] \quad \text{Use. Eq. \ref{incompressible}} \\
U_m(C) &= - \frac{k_B T}{\Omega} \bigg[ (\det \mathbf{F} - 1) \log(\frac{\det \mathbf{F}}{\det \mathbf{F} - 1}) + \frac{\chi}{\det \mathbf{F}} \bigg] \quad \text{Use log property}\\
U_m(C) &= \frac{k_B T}{\Omega} \bigg[ (\det \mathbf{F} - 1) \log (\frac{\det \mathbf{F} - 1}{\det \mathbf{F}}) - \frac{\chi}{\det \mathbf{F}} \bigg] 
\end{split}
\end{align}
Total strain energy in terms of the deformation gradient: 
\begin{align*}
U (\mathbf{F}, \mu) &= \frac{1}{2} N k_B T \big[ \tr \mathbf{C} - 3 - 2 \log (\det \mathbf{F}) \big] + \frac{k_B T}{\Omega} \bigg[ (\det \mathbf{F} - 1) \log (\frac{\det \mathbf{F} - 1}{\det \mathbf{F}}) - \frac{\chi}{\det \mathbf{F}} \bigg] - \mu \big(\frac{\det \mathbf{F} - 1}{\Omega} \big)
\end{align*}

%========================================================================
%========================================================================
\subsection{Nominal Stress and Concentration}
The nominal stress and the concentration of the solvent can be obtained from the free energy density: 
\subsubsection{Nominal Stress}
\begin{align*}
\mathbf{S} &= \pdv{U}{\mathbf{F}} \\
\mathbf{S} &= \pdv{U}{I_1} \pdv{I_1}{\mathbf{F}} + \pdv{U}{I_3} \pdv{I_3}{\mathbf{F}} 
\end{align*}
where $I_1 = \tr \mathbf{C}$, $\mathbf{C} = \mathbf{F}^T \mathbf{F}$ and $I_3 = \det \mathbf{F}$; therefore, \\ \\ 
First Term:
\begin{align*}
\pdv{U}{I_1} &= \pdv{U}{\tr \mathbf{C}} = \frac{1}{2} N k_B T 
\end{align*}
Second Term:
\begin{align*}
\pdv{I_1}{\mathbf{F}} &= \pdv{I_1}{\mathbf{C}} \pdv{\mathbf{C}}{\mathbf{F}} \\
				&= \pdv{\tr \mathbf{C}}{\mathbf{C}} \pdv{(\mathbf{F}^T \mathbf{F})}{\mathbf{F}} \quad \text{where } \pdv{\tr \mathbf{A}}{\mathbf{A}} = I \\
			&= \mathbf{I} \bigg[ \pdv{\mathbf{F}}{\mathbf{F}} \mathbf{F} + \mathbf{F} \pdv{\mathbf{F}}{\mathbf{F}} \bigg] \\
\pdv{I_1}{\mathbf{F}} &= 2 \mathbf{F}
\end{align*}
Third Term:
\begin{align*}
\pdv{U}{I_3} &= \pdv{\hat{U}}{\det \mathbf{F}} \\
			&= \pdv{}{\det \mathbf{F}} \bigg[ \frac{1}{2} N k_B T \big[ \tr (\mathbf{C}) - 3 - 2 \log (\det \mathbf{F}) \big] + \frac{k_B T}{\Omega} \big[ (\det \mathbf{F} - 1) \log \big( \frac{\det \mathbf{F} - 1}{\det \mathbf{F}} \big) - \frac{\chi}{\det \mathbf{F}} \big] - \mu \big(\frac{\det \mathbf{F} - 1}{\Omega} \big) \bigg] \\
			&= \frac{1}{2} N k_B T \bigg(-2 \pdv{\log (\det \mathbf{F})}{\det \mathbf{F}} \bigg) + \frac{k_B T}{\Omega} \bigg[ \frac{1}{\det \mathbf{F}} + \log \bigg( \frac{\det \mathbf{F} - 1}{\det \mathbf{F}} \bigg) + \frac{\chi}{(\det \mathbf{F})^2} \bigg] - \frac{\mu}{\Omega} \bigg( \pdv{\det \mathbf{F}}{\det \mathbf{F}} \bigg)\\
			&= - N k_B T \frac{1}{\det \mathbf{F}} + \frac{k_B T}{\Omega} \bigg[ \frac{1}{\det \mathbf{F}} + \log \bigg( \frac{\det \mathbf{F} - 1}{\det \mathbf{F}} \bigg) + \frac{\chi}{(\det \mathbf{F})^2} \bigg] - \frac{\mu}{\Omega} \\
			&= N k_B T \bigg\{- \frac{1}{\det \mathbf{F}} + \frac{1}{N \Omega} \bigg[ \frac{1}{\det \mathbf{F}} + \log \bigg( \frac{\det \mathbf{F} - 1}{\det \mathbf{F}} \bigg) + \frac{\chi}{(\det \mathbf{F})^2} \bigg] - \frac{\mu}{N \Omega k_B T} \bigg\} \\
\pdv{U}{I_3} &= N k_B T \bigg\{- \frac{1}{\det \mathbf{F}} + \frac{1}{N \Omega} \bigg[ \frac{1}{\det \mathbf{F}} + \log \bigg( \frac{\det \mathbf{F} - 1}{\det \mathbf{F}} \bigg) + \frac{\chi}{(\det \mathbf{F})^2} - \frac{\mu}{k_B T} \bigg] \bigg\} 
\end{align*}
Fourth Term: 
\begin{align*}
\pdv{I_3}{F_{iJ}} &= \pdv{\det \mathbf{F}}{F_{iJ}} \quad \text{where } \pdv{\det \mathbf{F}}{F} = (\det \mathbf{F}) \mathbf{F}^{-T} \\
			&= \det \mathbf{F} F_{iJ}^{-T}
\end{align*}
Therefore: 
\begin{align*}
\mathbf{P} &= \pdv{\hat{U}}{I_1} \pdv{I_1}{\mathbf{F}} + \pdv{\hat{U}}{I_3} \pdv{I_3}{\mathbf{F}} \\
	  &= \frac{1}{2} N k_B T (2 \mathbf{F}) + N k_B T \bigg\{- \frac{1}{\det \mathbf{F}} + \frac{1}{N \Omega} \bigg[ \frac{1}{\det \mathbf{F}} + \log \bigg( \frac{\det \mathbf{F} - 1}{\det \mathbf{F}} \bigg) + \frac{\chi}{(\det \mathbf{F})^2} - \frac{\mu}{k_B T} \bigg] \bigg\} \det \mathbf{F} \mathbf{F}^{-T} \\
P &= N k_B T \bigg \langle \mathbf{F} + \bigg\{- \frac{1}{\det \mathbf{F}} + \frac{1}{N \Omega} \bigg[ \frac{1}{\det \mathbf{F}} + \log \bigg( \frac{\det \mathbf{F} - 1}{\det \mathbf{F}} \bigg) + \frac{\chi}{(\det \mathbf{F})^2} - \frac{\mu}{k_B T} \bigg] \bigg\} \det \mathbf{F} \mathbf{F}^{-T} \bigg \rangle
\end{align*}
\subsubsection{Concentration}
\begin{align}\label{Concentration}
\begin{split}
C (\mathbf{F}, \mu) = - \frac{\delta \hat{U}}{\delta \mu} &= - \bigg( - \frac{\det \mathbf{F} - 1}{\Omega} \bigg) \\
						C (\mathbf{F}, \mu) &= \frac{\det \mathbf{F} - 1}{\Omega}
\end{split}
\end{align}

%========================================================================
%========================================================================
\subsection{Weak Forms and Time Integration}
The weak forms are as followed: 
\begin{align*}
\int \mathbf{\bar{P}} \cdot \bar{\nabla} \delta \boldsymbol\varphi dS  &= \int \mathbf{\bar{b}_o} \cdot \delta \boldsymbol\varphi dS \\
\int \mathbf{P} \cdot \grad \delta \boldsymbol\varphi dV &= \int \mathbf{b_o} \cdot \delta \boldsymbol\varphi dV \\
\int \pdv{C}{t} \delta \mu dV - \int \mathbf{J} \cdot \grad \delta \mu dV &= \int r \delta \mu dV + \int i \delta \mu dS 
\end{align*}
The first two weak forms are not time-dependent (all in current time step) \\ \\
Backward Euler Scheme on the second weak form where \textbf{superscript t indicates previous time step} and no superscript indicates current time step
\begin{align*}
\int \frac{C - C^t}{\Delta t} \delta \mu dV - \int \mathbf{J} \cdot \frac{\delta \mu}{\delta X} dV - \int i \delta \mu dS - \int r \delta \mu dV &= 0 \\
\int (C - C^t) \delta \mu dV - \int \mathbf{J} \cdot \frac{\delta \mu}{\delta X} \Delta t dV - \int i \delta \mu \Delta t dS - \int r \delta \mu \Delta t dV &= 0 
\end{align*}

%========================================================================
%========================================================================
%========================================================================
\section{Normalization}
Normalization parameters below:
\begin{align*}
\text{Lengths normalized by thickness (dry)}&: l = \tilde{l} H \\
\text{Surfaces normalized by}&: dA = d \tilde{A} H^2 \\
\text{Volumes normalized by}&: dV = d \tilde{V} H^3 \text{ or } dV = d \tilde{V} \Omega \\ 
\text{Time normalized by}&: t = \tilde{t} \tau \quad \text{where } \tau = \frac{H^2}{D} \\
\text{Chemical Potential normalized by}&: \mu = \tilde{\mu} k_B T \\
\text{Stress normalized by}&: S= \tilde{S} N k_B T \quad \gamma = \tilde{\gamma} N k_B T \\
\text{Concentration normalized by}&: \Omega C = \tilde{C} \rightarrow C = \frac{\tilde{C}}{\Omega} 
\end{align*}
Normalization of nominal stress
\begin{align*}
\mathbf{P} &= N k_B T \bigg \langle \mathbf{F} + \bigg\{- \frac{1}{\det \mathbf{F}} + \frac{1}{N \Omega} \bigg[ \frac{1}{\det \mathbf{F}} + \log \bigg( \frac{\det \mathbf{F} - 1}{\det \mathbf{F}} \bigg) + \frac{\chi}{(\det \mathbf{F})^2} - \frac{\mu}{k_B T} \bigg] \bigg\} \det \mathbf{F} \mathbf{F}^{-T}\bigg \rangle \\
\mathbf{\tilde{P}} N k_B T &= N k_B T \bigg \langle \mathbf{F} + \bigg\{- \frac{1}{\det \mathbf{F}} + \frac{1}{N \Omega} \bigg[ \frac{1}{\det \mathbf{F}} + \log \bigg( \frac{\det \mathbf{F} - 1}{\det \mathbf{F}} \bigg) + \frac{\chi}{(\det \mathbf{F})^2} - \frac{\tilde{\mu} k_B T}{k_B T} \bigg] \bigg\} \det \mathbf{F} \mathbf{F}^{-T}\bigg \rangle \\
\mathbf{\tilde{P}} &= \mathbf{F} + \bigg\{- \frac{1}{\det \mathbf{F}} + \frac{1}{N \Omega} \bigg[ \frac{1}{\det \mathbf{F}} + \log \bigg( \frac{\det \mathbf{F} - 1}{\det \mathbf{F}} \bigg) + \frac{\chi}{(\det \mathbf{F})^2} - \tilde{\mu} \bigg] \bigg\} \det \mathbf{F} \mathbf{F}^{-T} \\
\mathbf{\tilde{P}} &= \mathbf{F} + \bigg\{- \frac{1}{J} + \frac{1}{N \Omega} \bigg[ \frac{1}{J} + \log \bigg( \frac{J - 1}{J} \bigg) + \frac{\chi}{J^2} - \tilde{\mu} \bigg] \bigg\} J \mathbf{F}^{-T}
\end{align*}
Normalization of concentration
\begin{align*}
C &= \frac{(\det \mathbf{F} - 1)}{\Omega} \\
\frac{\tilde{C}}{\Omega} &= \frac{(\det \mathbf{F} - 1)}{\Omega} \\
\tilde{C} &= \det \mathbf{F} - 1 = J - 1
\end{align*}

%========================================================================
%========================================================================
\subsection{Normalization of Weak Forms}
\begin{align*}
\int \mathbf{\bar{P}} \cdot \bar{\nabla} \delta \boldsymbol\varphi dS - \int \mathbf{\bar{b}_o} \cdot \delta \boldsymbol\varphi dS &= 0 \\
\int \mathbf{\tilde{\bar{P}}}H N k_B T \cdot \bar{\nabla} \delta \tilde{\boldsymbol\varphi} d\tilde{S} H^2 - \int \mathbf{\tilde{\bar{b}}_o} N k_B T \cdot \delta \tilde{\boldsymbol\varphi} H d \tilde{S} H^2 &= 0 \\
\int \mathbf{\tilde{\bar{P}}} \Omega N k_B T \cdot \bar{\nabla} \delta \tilde{\boldsymbol\varphi} d\tilde{S} - \int \mathbf{\tilde{\bar{b}}_o} N k_B T \cdot \delta \tilde{\boldsymbol\varphi} d \tilde{S} \Omega &= 0 \\
N \Omega k_B T \bigg[ \int \mathbf{\tilde{\bar{P}}} \cdot \bar{\nabla} \delta \tilde{\boldsymbol\varphi} d\tilde{S} - \int \mathbf{\tilde{\bar{b}}_o} \cdot \delta \tilde{\boldsymbol\varphi} d \tilde{S} \bigg] &= 0 
\end{align*}
\begin{align*}
\int \mathbf{P} \cdot \grad \delta \boldsymbol\varphi dV - \int \mathbf{b_o} \cdot \delta \boldsymbol\varphi dV &= 0\\
\int \mathbf{\tilde{P}} N k_B T \cdot \grad \delta \tilde{\boldsymbol\varphi} d\tilde{V} \Omega - \int \mathbf{\tilde{b}_o} \frac{N k_B T}{H} \cdot \delta \tilde{\boldsymbol\varphi} H d\tilde{V} \Omega &= 0 \\
\int \mathbf{\tilde{P}} N \Omega k_B T \cdot \grad \delta \tilde{\boldsymbol\varphi} d\tilde{V} - \int \mathbf{\tilde{b}_o} N \Omega k_B T \cdot \delta \tilde{\boldsymbol\varphi} d\tilde{V} &= 0 \\
N \Omega k_B T \bigg[ \int \mathbf{\tilde{P}}  \cdot \grad \delta \tilde{\boldsymbol\varphi} d\tilde{V} - \int \mathbf{\tilde{b}_o} \cdot \delta \tilde{\boldsymbol\varphi} d\tilde{V} \bigg] &= 0 
\end{align*}
Combine
\begin{align*}
N \Omega k_B T \bigg[ \int \mathbf{\tilde{\bar{P}}} \cdot \bar{\nabla} \delta \tilde{\boldsymbol\varphi} d\tilde{S} + \int \mathbf{\tilde{P}}  \cdot \grad \delta \tilde{\boldsymbol\varphi} d\tilde{V} - \int \mathbf{\tilde{\bar{b}}_o} \cdot \delta \tilde{\boldsymbol\varphi} d \tilde{S} - \int \mathbf{\tilde{b}_o} \cdot \delta \tilde{\boldsymbol\varphi} d\tilde{V} \bigg] &= 0 \\
\end{align*}
Normalize kinetic law
\begin{align}\label{fluxLawFinNorm}
\begin{split}
\mathbf{J} &= - \frac{CD}{k_B T} \mathbf{F^{-1}} \mathbf{F^{-T}} \pdv{\mu}{X} \\
   		&= - \frac{\tilde{C}}{\Omega} \frac{D}{k_B T} \mathbf{F^{-1}} \mathbf{F^{-T}} \pdv{\mu}{X} \quad \text{where } \tau = H^2/D \\
\mathbf{\tilde{J}} &= - \frac{\tilde{C}}{\Omega} \frac{H^2}{\tau k_B T} \mathbf{F^{-1}} \mathbf{F^{-T}} \pdv{ (\tilde{\mu} k_B T) }{(\tilde{X} H)} \\
\mathbf{\tilde{J}} &= - \frac{\tilde{C}}{\Omega} \frac{H}{\tau} \mathbf{F^{-1}} \mathbf{F^{-T}} \pdv{ \tilde{\mu} }{\tilde{X}} \\
\end{split}
\end{align}
Using the normalized Flux equation, Eq. \ref{fluxLawFinNorm} in the first step 
\begin{align*}
\int (C - C^t) \delta \mu dV - \int \mathbf{J} \cdot \frac{\delta \mu}{\delta X} \Delta t dV - \int i \delta \mu \Delta t dS - \int r \delta \mu \Delta t dV &= 0 \\
\int \frac{(\tilde{C} - \tilde{C}^t)}{\Omega} \delta \tilde{\mu} k_B T d\tilde{V} \Omega - \int \mathbf{\tilde{J}} \cdot \frac{\delta (\tilde{\mu} k_B T)}{\delta \tilde{X} H} \Delta \tilde{t} \tau d \tilde{V} \Omega - \int \frac{\tilde{i}}{H^2 \tau} \delta \tilde{\mu} k_B T \Delta \tilde{t} \tau d\tilde{S} H^2 - \int \frac{\tilde{r}}{\Omega \tau} \delta \tilde{\mu} k_B T \Delta \tilde{t} \tau d \tilde{V} \Omega &= 0 \\
k_B T \int (\tilde{C} - \tilde{C}^t) \delta \tilde{\mu} d\tilde{V} - k_B T\int \mathbf{\tilde{J}} \cdot \frac{\delta \tilde{\mu}}{\delta \tilde{X}} \Delta \tilde{t} \frac{\tau}{H} \Omega d\tilde{V} - k_B T \int \tilde{i} \delta \tilde{\mu} \Delta \tilde{t} d\tilde{S} - k_B T \int \tilde{r} \delta \tilde{\mu}  \Delta \tilde{t} d \tilde{V} &= 0 \\
k_B T \bigg[ \int (\tilde{C} - \tilde{C}^t) \delta \tilde{\mu} d\tilde{V} + \int \frac{\tilde{C}}{\Omega} \frac{H}{\tau} \mathbf{F^{-1}} \mathbf{F^{-T}} \pdv{ \tilde{\mu} }{\tilde{X}} \cdot \frac{\delta \tilde{\mu}}{\delta \tilde{X}} \Delta \tilde{t} \frac{\tau}{H} \Omega d\tilde{V} - \int \tilde{i} \delta \tilde{\mu} \Delta \tilde{t} d\tilde{S} - \int \tilde{r} \delta \tilde{\mu}  \Delta \tilde{t} d \tilde{V} \bigg] &= 0 \\
k_B T \bigg[ \int (\tilde{C} - \tilde{C}^t) \delta \tilde{\mu} d\tilde{V} + \int \tilde{C} \mathbf{F^{-1}} \mathbf{F^{-T}} \pdv{ \tilde{\mu} }{\tilde{X}} \cdot \frac{\delta \tilde{\mu}}{\delta \tilde{X}} \Delta \tilde{t} d\tilde{V} - \int \tilde{i} \delta \tilde{\mu} \Delta \tilde{t} d\tilde{S} - \int \tilde{r} \delta \tilde{\mu}  \Delta \tilde{t} d \tilde{V} \bigg] &= 0 
\end{align*}
Combine two equations and divide by $N \Omega k_B T$, leaving the final normalized weak form 
\begin{align*}
\frac{1}{N \Omega} \bigg[ \int (\tilde{C} - \tilde{C}^t) \delta \tilde{\mu} d\tilde{V} + \int \tilde{C} \mathbf{F^{-1}} \mathbf{F^{-T}} \pdv{ \tilde{\mu} }{\tilde{X}} \cdot \frac{\delta \tilde{\mu}}{\delta \tilde{X}} \Delta \tilde{t} d\tilde{V} - \int \tilde{i} \delta \tilde{\mu} \Delta \tilde{t} d\tilde{S} - \int \tilde{r} \delta \tilde{\mu}  \Delta \tilde{t} d \tilde{V} \bigg] &= 0 \\
\int \mathbf{\tilde{\bar{P}}} \cdot \bar{\nabla} \delta \tilde{\boldsymbol\varphi} d\tilde{S} + \int \mathbf{\tilde{P}}  \cdot \grad \delta \tilde{\boldsymbol\varphi} d\tilde{V} - \int \mathbf{\tilde{\bar{b}}_o} \cdot \delta \tilde{\boldsymbol\varphi} d \tilde{S} - \int \mathbf{\tilde{b}_o} \cdot \delta \tilde{\boldsymbol\varphi} d\tilde{V} &= 0
\end{align*}
where the previous time step is $\tilde{C^t}$

%========================================================================
%========================================================================
%========================================================================
\section{Initial Conditions}

%========================================================================
%========================================================================
\subsection{Initial Condition for Chemical Potential without Surface Tension}
Initially stress free gel: 
\begin{align*}
\tilde{S} =  F_{iJ} + \bigg\{- \frac{1}{\det \mathbf{F}} + \frac{1}{N \Omega} \bigg[ \frac{1}{\det \mathbf{F}} + \log \bigg( \frac{\det \mathbf{F} - 1}{\det \mathbf{F}} \bigg) + \frac{\chi}{(\det \mathbf{F})^2} - \tilde{\mu} \bigg] \bigg\} \det \mathbf{F} F_{iJ}^{-T} &= 0 \\
		F_{iJ} - F_{iJ}^{-T} + \frac{1}{N \Omega} \bigg[ \frac{1}{\det \mathbf{F}} + \log \bigg( \frac{\det \mathbf{F} - 1}{\det \mathbf{F}} \bigg) + \frac{\chi}{(\det \mathbf{F})^2} - \tilde{\mu} \bigg] \det \mathbf{F} F_{iJ}^{-T} &= 0 \\
		F_{iJ} - F_{iJ}^{-T} + \frac{1}{N \Omega} \bigg[ F_{iJ}^{-T} + \det \mathbf{F} F_{iJ}^{-T} \log \bigg( \frac{\det \mathbf{F} - 1}{\det \mathbf{F}} \bigg) + \frac{\chi}{\det \mathbf{F}} F_{iJ}^{-T}\bigg] &=  \tilde{\mu} \frac{1}{N \Omega} \det \mathbf{F} F_{iJ}^{-T} \\
		 \frac{N \Omega}{\det \mathbf{F} F_{iJ}^{-T}} \bigg( F_{iJ} - F_{iJ}^{-T} \bigg) + \frac{1}{\det \mathbf{F} F_{iJ}^{-T}} \bigg[ F_{iJ}^{-T} + \det \mathbf{F} F_{iJ}^{-T} \log \bigg( \frac{\det \mathbf{F} - 1}{\det \mathbf{F}} \bigg) + \frac{\chi}{\det \mathbf{F}} F_{iJ}^{-T}\bigg] &=  \tilde{\mu} \\
		  \frac{N \Omega}{\det \mathbf{F} F_{iJ}^{-T}} \bigg( F_{iJ} - F_{iJ}^{-T} \bigg) + \frac{1}{\det \mathbf{F}} + \log \bigg( \frac{\det \mathbf{F} - 1}{\det \mathbf{F}} \bigg) + \frac{\chi}{(\det \mathbf{F})^2} &=  \tilde{\mu} \quad \text{where } \det \mathbf{F} = \lambda_o^3 \\
		  N \Omega \frac{\lambda_o}{\lambda_o^3} \bigg( \lambda_o - \frac{1}{\lambda_o} \bigg) + \frac{1}{\lambda_o^3} + \log \bigg( \frac{\lambda_o^3 - 1}{\lambda_o^3} \bigg) + \frac{\chi}{\lambda_o^6} &= \tilde{\mu} 
\end{align*}
This gives the initial normalized chemical potential
\begin{align*}
 \tilde{\mu} = N \Omega \bigg( \frac{1}{\lambda_o} - \frac{1}{\lambda_o^3} \bigg) + \frac{1}{\lambda_o^3} + \log \bigg( \frac{\lambda_o^3 - 1}{\lambda_o^3} \bigg) + \frac{\chi}{\lambda_o^6} 
\end{align*}

%========================================================================
%========================================================================
\subsection{Initial Condition for Chemical Potential with Surface Tension}
Paper Reference: \textit{Elatocapillarity: Surface Tension and the Mechanics of Soft Solids} by Style, Jagota, Hui, and Dufresne. \\ \\
The essential idea is a connection between surface stress and bulk stress at the interface. The governing interfacial equation is a condition for static equilibrium. 
\begin{equation*}
\int_{S} \sigma_{1} \cdot \mathbf{n} \mathrm{d} S-\int_{S} \boldsymbol{\sigma}_{2} \cdot \mathbf{n} \mathrm{d} S+\oint_{C} \mathbf{\Upsilon} \cdot \mathbf{b} \mathrm{d} l=0
\end{equation*}
Using the surface divergence theorem, we can obtain the generalization of Laplace's Law
\begin{align*}
\boldsymbol{\sigma} \cdot \mathbf{n} + \tilde{\div} \boldsymbol{\Upsilon} &= 0 \quad \text{Assume surface stress is isotropic} \\
\boldsymbol{\sigma} \cdot \mathbf{n} + \tilde{\div} (\gamma \mathbf{I}) &= 0 \\
\boldsymbol{\sigma} \cdot \mathbf{n} + \gamma \tilde{\div} \mathbf{I} &= 0 \\
\boldsymbol{\sigma} \cdot \mathbf{n} + \gamma (\tilde{\div} \mathbf{n}) \mathbf{n} &= 0 \\
\boldsymbol{\sigma} \cdot \mathbf{n} + \gamma \kappa \mathbf{n} &= 0 
\end{align*}
Written in spherical coordinates
\begin{align*}
\begin{bmatrix}
\sigma_{rr} & 0 				       & 0 \\
0		  & \sigma_{\theta \theta} & 0 \\
0		  & 0 				       & \sigma_{\varphi \varphi} 
\end{bmatrix}
\begin{bmatrix}
1 \\
0 \\
0
\end{bmatrix} 
+ \gamma \kappa
\begin{bmatrix}
1 \\
0 \\
0
\end{bmatrix} &= 
\begin{bmatrix}
0 \\
0 \\
0
\end{bmatrix} \\
\sigma_{rr} + \gamma \kappa &= 0
\end{align*}
What is $\kappa$ in the case of a sphere? 
\begin{align*}
\text{Principal/Mean}: \kappa &= \frac{1}{r} \\
\text{Gaussian}: \kappa &= \frac{1}{r^2} 
\end{align*}
Here we use two times the mean curvature
\begin{align*}
\sigma_r &= -\frac{2 \gamma}{r_{out}} \quad \text{where } \lambda_r = \frac{r_{out}}{R_{out}} \rightarrow r_{out} = \lambda_r R_{out} \\
\sigma_r &= - \frac{2 \gamma}{\lambda_r R_{out}} \quad \text{where } R_{out} = 1.0 \\
\sigma_r &= - \frac{2 \gamma}{\lambda_r} 
\end{align*}
Another way to think about this derivation is in terms of the Young-Laplace Equation, where we have two mediums $\alpha$ and $\beta$. 
\begin{align*}
\gamma = (P^\beta - P^{\alpha}) \pdv{V^{\beta}}{A^\beta}
\end{align*}
The volume and surface area of a sphere are given as:
\begin{align*}
\text{Volume: } & V^\beta = \frac{4}{3} \pi r^3 \\
\text{Surface Area: } & A ^\beta= 4 \pi r^2
\end{align*}
Therefore
\begin{align*}
\gamma &= (P^\beta - P^{\alpha}) \frac{r}{2} \\
\frac{2 \gamma}{r} &= \Delta P 
\end{align*}
The constitutive expression given is (spherical coordinates)
\begin{align*}
\sigma_i  &= \frac{\lambda_i}{J} \pdv{}{\lambda_i} U_e + \frac{d}{dJ} U_m - \frac{\mu}{\Omega} \\
\sigma_r &= \frac{\lambda_r}{J} \pdv{}{\lambda_r} U_e + \frac{d}{dJ} U_m - \frac{\mu}{\Omega}
\end{align*} 
Free energy due to stretching (Flory, 1953) and mixing: 
\begin{align*}
U_e (\boldsymbol{F}) &= \frac{1}{2} N k_B T \big[ F_{iK} F_{iK} - 3 - 2 \log (\det \mathbf{F}) \big]  \\
U_e (\lambda_r, \lambda_\theta, \lambda_\varphi) &= \frac{1}{2} N k_B T \big[ \lambda_r^2 + \lambda_\theta^2 + \lambda_\varphi^2 - 3 - 2 \log (\lambda_r \lambda_\theta \lambda_\varphi) \big] 
\end{align*}
\begin{align*}
U_m(\mathbf{F}) &= \frac{k_B T}{\Omega} \bigg[ (\det \mathbf{F} - 1) \log (\frac{\det \mathbf{F} - 1}{\det \mathbf{F}}) - \frac{\chi}{\det \mathbf{F}} \bigg] \\
U_m(J)  &= \frac{k_B T}{\Omega} \bigg[ (J- 1) \log (\frac{J - 1}{J}) - \frac{\chi}{J} \bigg] 
\end{align*}
Calculate the derivatives
\begin{align*}
\pdv{U_e}{\lambda_r} &= \frac{1}{2} N k_B T \pdv{}{\lambda_r} \big[ \lambda_r^2 + \lambda_\theta^2 + \lambda_\varphi^2 - 3 - 2 \log (\lambda_r \lambda_\theta \lambda_\varphi) \big] \\
				 &= \frac{1}{2} N k_B T \big[ 2 \lambda_r - 2 \frac{\lambda_\theta \lambda_\varphi}{\lambda_r \lambda_\theta \lambda_\varphi} \big] \\
\pdv{U_e}{\lambda_r} &= N k_B T \big[ \lambda_r - \frac{1}{\lambda_r} \big] 
\end{align*}
\begin{align*}
\frac{dU_m}{dJ} &= \frac{k_B T}{\Omega} \frac{d}{dJ} \bigg[ (J- 1) \log (\frac{J - 1}{J}) - \frac{\chi}{J} \bigg] \\
\frac{dU_m}{dJ} &= \frac{k_B T}{\Omega} \bigg[ \frac{1}{J} + \log \big( 1 - \frac{1}{J} \big) + \frac{\chi}{J^2} \bigg]
\end{align*}
Substitute into constitutive relationship:
\begin{align*}
\sigma_r &= \frac{\lambda_r}{J} \pdv{}{\lambda_r} U_e + \frac{d}{dJ} U_m - \frac{\mu}{\Omega} \\
\sigma_r &= N k_B T \frac{\lambda_r}{J} \big[ \lambda_r - \frac{1}{\lambda_r} \big] + \frac{k_B T}{\Omega} \bigg[ \frac{1}{J} + \log \big( 1 - \frac{1}{J} \big) + \frac{\chi}{J^2} \bigg] - \frac{\mu}{\Omega} \\
\tilde{\sigma}_r N k_B T &= N k_B T \frac{\lambda_r}{J} \big[ \lambda_r - \frac{1}{\lambda_r} \big] + \frac{k_B T}{\Omega} \bigg[ \frac{1}{J} + \log \big( 1 - \frac{1}{J} \big) + \frac{\chi}{J^2} \bigg] - \frac{\tilde{\mu} k_B T}{\Omega} \\
\tilde{\sigma}_r &= \frac{1}{\lambda_r} - \frac{1}{\lambda_r^3} + \frac{1}{N \Omega} \bigg[ \frac{1}{\lambda_r^3} + \log \big( 1 - \frac{1}{\lambda_r^3} \big) + \frac{\chi}{\lambda_r^6} \bigg] - \frac{\tilde{\mu}}{N \Omega} \\
\frac{\tilde{\mu}}{N \Omega} &= \frac{1}{\lambda_r} - \frac{1}{\lambda_r^3} + \frac{1}{N \Omega} \bigg[ \frac{1}{\lambda_r^3} + \log \big( 1 - \frac{1}{\lambda_r^3} \big) + \frac{\chi}{\lambda_r^6} \bigg] - \tilde{\sigma}_r \\
\tilde{\mu} &= N \Omega \bigg( \frac{1}{\lambda_r} - \frac{1}{\lambda_r^3} \bigg) + \frac{1}{\lambda_r^3} + \log \big( 1 - \frac{1}{\lambda_r^3} \big) + \frac{\chi}{\lambda_r^6} - N \Omega \tilde{\sigma}_r \quad \text{Substitute } \sigma_r = - \frac{2 \gamma}{\lambda_r} \\
\tilde{\mu} &= N \Omega \bigg( \frac{1}{\lambda_r} - \frac{1}{\lambda_r^3} \bigg) + \frac{1}{\lambda_r^3} + \log \big( 1 - \frac{1}{\lambda_r^3} \big) + \frac{\chi}{\lambda_r^6} + N \Omega \frac{2 \gamma}{\lambda_r}
\end{align*} 

%========================================================================
%========================================================================
%========================================================================
\section{Length-Scales}

%========================================================================
%========================================================================
\subsection{Elastocapillary Length Scale}
Refer to the paper \textit{Elastocapillarity: Surface Tension and the Mechanics of Soft Solids} by Style, Jagota, Hui, and Dufresne \\ \\
The length scale is given by:
\begin{align}\label{elastocapLengthScale}
l_e = \frac{\gamma}{G_0} 
\end{align}
where in Bouklas and Huang 
\begin{align*}
G_0 = \frac{N k_B T}{\lambda_0} \quad \quad G_\infty = \frac{N k_B T}{\lambda_{\infty}^f}
\end{align*}
The elastocapillary number (dimensionless) is defined with the curvature
\begin{align}
n_{ec} = \frac{\gamma \kappa}{G_0}
\end{align}
Considering instantaneous length scale, we can derive the elastocapillary length scale with respect to the constants we define:
\begin{align*}
l_e &= \frac{\gamma}{G_0} \\
l_e &= \frac{\gamma \lambda_0}{N k_B T} \\
\end{align*}
where the normalization for the surface energy, $\gamma$, is 
\begin{align*}
\tilde{\gamma} = \frac{\gamma H^2}{k_B T}
\end{align*}
Normalized 
\begin{align*}
\tilde{l}_e &= \frac{k_B T \tilde{\gamma}}{H^2} \frac{\lambda_0}{N k_B T} \quad \text{where } H = 1 \\
\tilde{l}_e &= \frac{\tilde{\gamma} \lambda_0}{N}
\end{align*}


\end{document}