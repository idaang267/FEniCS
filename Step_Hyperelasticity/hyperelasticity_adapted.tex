\documentclass[12pt,3p]{article}
\usepackage[T1]{fontenc}
\usepackage[utf8]{inputenc}
\usepackage[english]{babel}
\usepackage[margin=0.75in]{geometry}
\usepackage{amsmath}
\usepackage{mathtools}
\usepackage{enumitem}
\usepackage{physics}

\usepackage[round,numbers]{natbib}
\usepackage[colorlinks = false]{hyperref}

\begin{document}

\title{FEniCS: Incompressible hyperelasticity with weak form \\
	\large{} }
\author{Ida Ang}
\date{\vspace{-5ex}}
\maketitle

\section{Problem Definition}
This document is based on the original FEniCS hyperelasticity demo, but uses a different formulation. In the original demo, the weak form is not written explicitly and instead the demo uses an energy minimization scheme. 

\subsection{Energy Density Function}
\subsubsection{Compressibility}
The energy density function for compressible Neo-Hookean materials is: 
\begin{equation}\label{EDComp}
W = \frac{\mu}{2} (I_1 - 3 - 2 \ln J) + \frac{\lambda}{2} (\ln J)^2
\end{equation}
where $\mu$ and $\lambda$ are Lamé Parameters and can be defined in terms of the Young's modulus, E, and Poisson's ratio $\nu$
\begin{equation}\label{lame}
\mu = \frac{E}{2 (1 + \nu)} \quad \lambda = \frac{E \nu}{(1+ \nu) (1 - 2 \nu)} 
\end{equation}
The first and third invariant of the deformation:
\begin{equation}\label{invariants}
I_1 = \trace(\mathbf{C}) \quad I_3 = \mathbf{J} = \det(\mathbf{F})  
\end{equation}
\subsubsection{Incompressibility}
For an incompressible material, \textbf{J} = det(\textbf{F}) = 1; therefore, Eq. \ref{EDComp} can be modified to: 
\begin{align*}
W &= \frac{\mu}{2} (I_1 - 3 - 2 \ln (1)) + \frac{\lambda}{2} (\ln (1))^2 \\
W &= \frac{\mu}{2} (I_1 - 3) 
\end{align*}
To enforce incompressibility, introduce a lagrange multiplier, p, which acts like a stress term. 
\begin{equation}\label{EnergyDensity}
W(\mathbf{F}) = \frac{\mu}{2} (I_1 - 3) + \text{p} (\mathbf{J}-1) 
\end{equation}
Now the formulation only uses one Lamé parameter and the first invariant

\subsection{Equilibrium equation}
P is the nominal stress tensor and b is the body forces: 
\begin{equation}\label{EqmEq}
P_{iJ,J} + b_i = 0 \rightarrow \pdv{P_{iJ}}{X_J} + b_i = 0 
\end{equation}
The nominal stress tensor is defined as: 
\begin{equation}\label{PW}
P_{iJ} = \pdv{W}{F_{iJ}}
\end{equation}
Start by writing the energy density function Eq. \ref{EnergyDensity} in terms of the invariants
\begin{align*}
W(\mathbf{F}) &= \frac{\mu}{2}  [ \trace(\mathbf{F^T} \mathbf{F}) - 3 ] + p (\det \mathbf{F} - 1) \quad \text{Take derivative with respect to F (Eq. \ref{EqmEq})} \\
P = \pdv{W}{\mathbf{F}} &= \frac{\mu}{2}  \bigg( \pdv{\trace(\mathbf{F^T} \mathbf{F})}{\mathbf{F}} \bigg) + p \pdv{\det \mathbf{F} }{\mathbf{F}}
\end{align*}
Proof of the derivative $\pdv{\trace(\mathbf{F^T} \mathbf{F})}{\mathbf{F}}$ in indicial: 
\begin{align*}
\pdv{\trace(\mathbf{F^T} \mathbf{F})}{\mathbf{F}} &= \pdv{F_{kI} F_{kI}}{F_{pQ}} \quad \text{Product Rule}\\ 
									&= \pdv{F_{kI}}{F_{pQ}} F_{kI} + F_{kI} \pdv{F_{kI}}{F_{pQ}} \\
									&= \delta_{kp} \delta_{IQ} F_{kI} + F_{kI} \delta_{kp} \delta_{IQ} \\
									&= F_{pQ} + F_{pQ} = 2 F_{pQ}
\end{align*}
Therefore, we have the following relationship: 
\begin{equation}\label{proof1}
\pdv{\trace(\mathbf{F^T} \mathbf{F})}{\mathbf{F}} = 2 \mathbf{F}
\end{equation}
Second derivative: 
\begin{align*}
\pdv{\det \mathbf{F} }{\mathbf{F}} = \det \mathbf{F} \mathbf{F^{-T}} = \mathbf{J} \mathbf{F^{-T}}
\end{align*}
Substituting these relationships in we can obtain the nominal stress tensor, P: 
\begin{equation}\label{P}
\mathbf{P} = \frac{\mu}{2}  2 \mathbf{F} + \text{p} \mathbf{J} \mathbf{F^{-T}} = \mu \mathbf{F} + \text{p} \mathbf{J} \mathbf{F^{-T}}
\end{equation}

\section{Weak Form}
Take Eq. \ref{EqmEq} and multiply by a test function 
\begin{align}\label{befInt}
\begin{split}
\pdv{P_{iJ}}{X_J} \mathbf{e_i} \cdot v_j \mathbf{e_j} &= - b_i \mathbf{e_i} \cdot v_j \mathbf{e_j} \\
\pdv{P_{iJ}}{X_J} v_j \delta_{ij} &= - b_i v_j \delta_{ij} \\
\pdv{P_{iJ}}{X_J} v_i &= - b_i v_i \quad \text{Integrate over domain} \\
\int_{\Omega_o} \pdv{P_{iJ}}{X_J} v_i dV &= - \int_{\Omega_o} b_i v_i dV
\end{split}
\end{align}
Use integration by parts: 
\begin{align*}
(fg)' = f'g + f g' \rightarrow f'g &= (fg)' - f g' \\
			\pdv{P_{iJ}}{X_j} v_i &= (P_{iJ} v_i)_{,J} - P_{iJ} v_{i,J}
\end{align*}
Substitute into Eq. \ref{befInt}:
\begin{align*}
\int_{\Omega_o} (P_{iJ} v_i)_{,J} dV - \int_{\Omega_o} P_{iJ} v_{i,J} dV &= - \int_{\Omega_o} b_i v_i dV \quad \text{Use divergence theorem} \\
\int_{\partial \Omega_o} P_{iJ} N_J v_i dS - \int_{\Omega_o} P_{iJ} v_{i,J} dV &= - \int_{\Omega_o} b_i v_i dV \quad \text{Recognize traction } P_{iJ} N_J = T_i\\
\int_{\partial \Omega_o} T_i v_i dS - \int_{\Omega_o} P_{iJ} v_{i,J} dV &= - \int_{\Omega_o} b_i v_i dV \quad \text{rearrange} \\
\int_{\Omega_o} P_{iJ} v_{i,J} dV &= \int_{\Omega_o} b_i v_i dV + \int_{\partial \Omega_o} T_i v_i dS 
\end{align*}
Rewrite in direct notation to obtain weak form: 
\begin{equation}\label{wForm}
 \int_{\Omega_o} P : \text{Grad}(v) dV = \int_{\Omega_o} b v dV + \int_{\partial \Omega_o} T v dS
\end{equation}
Finally, in order to enforce incompressibility we use Eq. \ref{invariants} where:
\begin{equation*}
\det F = 1 \rightarrow \det F - 1 = 0 \rightarrow J - 1 = 0
\end{equation*}
This can be multiplied by a test function and integrated over the domain $\Omega_o$
\begin{equation}\label{wFormIn}
\int_{\Omega_o} (J - 1) \tau dV = 0
\end{equation}

\section{Bilinear and Linear Form}
We have the bilinear form and linear form where we have two test functions
\begin{equation*}
a\big((\sigma,u),(\tau, v) \big) = L\big((\tau, v) \big)  \indent \forall (\tau, v) \in \Sigma_0 \times V 
\end{equation*}
Therefore, combining Eq. \ref{wForm} and \ref{wFormIn}, we have the bilinear and linear forms: 
\begin{equation}\label{bilinear}
a\big((\sigma,u),(\tau, v) \big) = \int_{\Omega_o} \big( P : \text{Grad}(v) + (J-1) \cdot \tau \big) dV
\end{equation}
\begin{equation}\label{linear}
L\big((\tau, v) \big) = \int_{\Omega_o} b v dV + \int_{\partial \Omega_o} T v dS
\end{equation}
The bilinear and linear form (Eq. \ref{bilinear} and \ref{linear})can also be combined into one statement: 
\begin{equation}\label{F}
a\big((\sigma,u),(\tau, v) \big) - L\big((\tau, v) \big) = \int_{\Omega_o} \big( P : \text{Grad}(v) + (J-1) \cdot \tau \big) dV - \int_{\Omega_o} b v dV - \int_{\partial \Omega_o} T v dS = 0 
\end{equation}
Solving for two unknowns, displacement (u) and hydrostatic pressure (p), which will allow us to obtain the nominal stress field (P)

\section{FEniCS Implementation}
First import modules: \\
{\fontfamily{qcr}\selectfont
from dolfin import * \\
import numpy as np \\ \\
}
SNES solver parameters are set similarly to the hyperelastic circle contact problem (code not included here) 

\subsubsection{User Parameters}
User parameters allow for control of the number of loop iterations.   \\
{\fontfamily{qcr}\selectfont
user\_par = Parameters("user") \\ \\
}
We are ramping the boundary condition displacement from 0 to 0.5 in 5 steps. \\
{\fontfamily{qcr}\selectfont
user\_par.add("u\_min",0.) \\
user\_par.add("u\_max",0.5) \\
user\_par.add("u\_nsteps",5) \\ \\ 
}
Add user parameters in the global parameter set \\
{\fontfamily{qcr}\selectfont
parameters.add(user\_par) \\ \\
}
Parse parameters from command line
{\fontfamily{qcr}\selectfont
parameters.parse() \\
info(parameters,True) \\
user\_par = parameters.user 
}

\subsubsection{Domain}
The domain is a unit cube: 
\[ \Omega = (0,1) \times (0,1) \times (0,1)\] 
Create a unit cube with 20 elements with 21 (20 + 1) vertices in one direction and 10 elements with 11 (10 + 1) vertices in the other two directions: \\
{\fontfamily{qcr}\selectfont
mesh = UnitCubeMesh(20, 10, 10) \\ \\
}
There are two unknowns in our formulation, displacement and hydrostatic pressure (lowercase p) Therefore, define a vector space of degree 2 for displacement (P2) and degree 1 for pressure (P1): \\
{\fontfamily{qcr}\selectfont
P2 = VectorElement("Lagrange", mesh.ufl\_cell(), 2) \\
P1 = FiniteElement("Lagrange", mesh.ufl\_cell(), 1) \\ \\
}
Taylor-Hood elements, {\fontfamily{qcr}\selectfont TH}, are elements defined where one vector space is one degree higher than the other. These type of elements are stable for this formulation. In our case, the vector space for displacement is one degree higher than that of pressure: \\ 
{\fontfamily{qcr}\selectfont
TH = MixedElement([P2,P1]) \\
V  = FunctionSpace(mesh, TH) 
}

\subsubsection{Boundary Conditions}
Use the following definitions for the boundary conditions, where we have left and right Dirichlet boundary conditions and one Neumann boundary condition: \\ \\
\textbf{Dirichlet}: \\ 
{\fontfamily{qcr}\selectfont
left =  CompiledSubDomain("near(x[0], side) \&\& on\_boundary", side = 0.0) \\
right = CompiledSubDomain("near(x[0], side) \&\& on\_boundary", side = 1.0) \\ \\
}
On $\Gamma_{D_0}$, we can define an initial displacement function to be applied to the right boundary.
\begin{equation*}
u = \bigg[ 0, s \bigg( s + (y - s) \cos \frac{\pi}{15} - (z - s) \sin \frac{\pi}{15} - y \bigg), s \bigg(s + (y - s) \sin \frac{\pi}{15} - (z - s) \cos \frac{\pi}{15} - x \bigg) \bigg]
\end{equation*}
This function gives a twist to the right side of the cube, which is why y and z are specified and x is zero. In this formulation, it is increased to a total theta = $\frac{\pi}{3}$: \\
{\fontfamily{qcr}\selectfont
r = Expression(("scale*0.0", \\
\indent "scale* \\
\indent (scale + (x[1] - scale)*cos(theta) - (x[2] - scale)*sin(theta) - x[1])", \\
\indent "scale* \\
\indent (scale + (x[1] - scale)*sin(theta) + (x[2] - scale)*cos(theta) - x[2])"), \\
\indent scale = 0, theta = pi/15, degree=2) \\ \\
}
On $\Gamma_{D_1}$, we fix the left side with no displacement: \\
{\fontfamily{qcr}\selectfont
c = Constant((0.0, 0.0, 0.0)) \\ \\
}
Combine the left and right boundary conditions with the correct expressions in {\fontfamily{qcr}\selectfont bcs } \\ 
{\fontfamily{qcr}\selectfont
bcl = DirichletBC(V, c, left) \\
bcr = DirichletBC(V, r, right) \\
bcs = [bcl, bcr] \\ \\ 
}
\textbf{Neumann}: \\
On $\Gamma_N = \frac{\partial \Omega}{\Gamma_D}$, define traction. Define body forces in the y direction \\
{\fontfamily{qcr}\selectfont
T  = Constant((0.1,  0.0, 0.0)) \\ \\
}
Define body forces in the y-direction (downwards) \\
{\fontfamily{qcr}\selectfont
B  = Constant((0.0,  -0.5, 0.0)) 
}

\subsubsection{Trial and Test Functions}
Define the trial, test functions and unknown functions.  \\ 
{\fontfamily{qcr}\selectfont
du = TrialFunction(V) \\
v  = TestFunction(V) \\          
w  = Function(V) \\ \\
}
Based on our weak forms (Eq. \ref{wForm} and \ref{wFormIn}), we have two test functions, $v$ and $\tau$, and two unknowns, displacement and pressure. Therefore, split the test function, v, and unknown function, w.
Call $v$, the test function or displacement $v_u$ and $\tau$, the test function for pressure, $v_p$. Splitting w will allow us to solve for the unknown displacement and hydrostatic pressure. \\
{\fontfamily{qcr}\selectfont
(v\_u, v\_p) = split(v) \\          
(u, p) = split(w) 
}

\subsubsection{Kinematics}
First find the spatial dimensions (length) of the displacement tensor \\
{\fontfamily{qcr}\selectfont
d = len(u)          \\ \\
}
Use dimension (d) to to define the identity tensor. {\fontfamily{qcr}\selectfont Identity} is an inbuilt function: \\
{\fontfamily{qcr}\selectfont
I = Identity(d)     \\ \\ 
}
Obtain the deformation gradient by first defining the displacement as the difference between the reference and current configuration: 
\begin{align}\label{Deformation}
\begin{split}
u &= x - X \quad \text{Take gradient with respect to current configuration} \\
\pdv{u}{X} &= \pdv{x}{X} - \pdv{X}{X} \quad \text{where } F = \pdv{x}{X} \\
\nabla u &= F - I \quad \text{Rearrange}\\
I + \nabla u &= F
\end{split}
\end{align}
Right Cauchy-Green Tensor:
\begin{equation}\label{CauchyGreen}
C = F^T F
\end{equation}
Define Eq. \ref{Deformation} and \ref{CauchyGreen}, the deformation tensor and the right Cauchy-Green Tensor. Note, {\fontfamily{qcr}\selectfont grad} is also an inbuilt functions. \\
{\fontfamily{qcr}\selectfont
F = I + grad(u)    \\
C = F.T*F \\ \\
}
Define the invariant of the deformation tensor given in Eq. \ref{invariants}, where {\fontfamily{qcr}\selectfont det } is a reserved keyword:  \\
{\fontfamily{qcr}\selectfont
J  = det(F) 
}

\subsubsection{Weak Form Definition}
Material parameters can be defined in two ways, one more compact,  as follows: \\
{\fontfamily{qcr}\selectfont
E, nu = 10.0, 0.3 \\ 
}
or \\
{\fontfamily{qcr}\selectfont
E = 10.0 \\ 
nu = 0.3 \\ \\
}
We can define Eq. \ref{lame}, $\mu$, as a constant not an expression because $\nu$ is defined in the code: \\
{\fontfamily{qcr}\selectfont
mu = Constant(E/(2*(1 + nu))) \\ \\ 
}
Define Eq. \ref{F} by first defining the nominal stress tensor, P, Eq. \ref{P}:
\begin{equation*}
P = \mu \mathbf{F} + p \mathbf{J} \mathbf{F^{-T}}
\end{equation*}
{\fontfamily{qcr}\selectfont
def P(u): \\
\indent return mu*F + p*J*inv(F.T) \\ \\
}
Define the weak form (don't worry about the name being the same as the deformation gradient). 
\begin{align*}
F &= a((\sigma,u),(\tau, v)) + L((\tau, v)) \\
F &= \int_{\Omega_o} \big( P : \text{Grad}(v) + (J-1) \cdot \tau \big) dV - \int_{\Omega_o} b v dV - \int_{\partial \Omega_o} T v dS = 0 
\end{align*}
Note that body force and traction are constants, the automatic inference algorithm will use 1 Gauss point to represent those integrals. For the first term, the algorithm will estimate the polynomial degree and use many Gauss points to approximate the integral. Using the line {\fontfamily{qcr}\selectfont (metadata={"quadrature\_degree": 4})} allows us to specify the number of Gauss points explicitly: \\
{\fontfamily{qcr}\selectfont
F = (inner(P(u), grad(v\_u)) \\
\indent \indent + inner(J-1., v\_p))*dx(metadata={"quadrature\_degree": 4}) \\
\indent \indent - dot(B, v\_u)*dx - dot(T, v\_u)*ds \\ \\
}
If we don't specify the quadrature degree, a warning will appear about computational time: \\ \\ 
Take the directional derivative of F in order to obtain the Jacobian. Recall that w consists of both displacement and hydrostatic pressure, but du refers to the incremental displacement:
\begin{equation}
J = D_{du}F = \frac{dF (u + \epsilon du ; v )}{d \epsilon} \Bigr\rvert_{\epsilon = 0}
\end{equation}
{\fontfamily{qcr}\selectfont
J\_o = derivative(F, w, du)
}

\subsubsection{Solve \& Save}
Solve the variational problem, where the fourth term specifies the Jacobian of the problem: \\
{\fontfamily{qcr}\selectfont
varproblem = NonlinearVariationalProblem(F, w, bcs, J=J\_o) \\ \\
} 
Define the solver and give it both the solver parameters (not the user parameters) \\
{\fontfamily{qcr}\selectfont
solver = NonlinearVariationalSolver(varproblem) \\
solver.parameters.update(snes\_solver\_parameters) \\ \\ 
} 
If info is set to True, solver parameters will be printed out. Solve problem with {\fontfamily{qcr}\selectfont
solver.solve()} \\
{\fontfamily{qcr}\selectfont
info(solver.parameters, False) \\ 
(iter, converged) = solver.solve() \\ \\
}
Loading parameter (list of values to update displacement) \\
{\fontfamily{qcr}\selectfont
u\_list = np.linspace(user\_par.u\_min,user\_par.u\_max,user\_par.u\_nsteps) \\ \\
}
Save results to an .xdmf file since we have multiple fields \\
{\fontfamily{qcr}\selectfont
file\_results = XDMFFile("results.xdmf") \\ \\
}
Set up the loop to increase displacement. The first statement in the loop updates the expression r with the scaling parameter within u\_list \\
{\fontfamily{qcr}\selectfont
for scale in u\_list: \\
\indent \indent r.scale = scale \\
\indent \indent solver.solve() \\ \\
} 
To save the solution, make a deep copy instead of a shallow copy. \\
{\fontfamily{qcr}\selectfont
\indent \indent (u, p) = w.split() \\ \\
} 
Using the rename function, we can specify the name of the variable we are viewing in Paraview:
{\fontfamily{qcr}\selectfont
\indent \indent u.rename("Displacement", "u") \\
\indent \indent p.rename("Pressure", "p") \\ \\
}
Save in Paraview format: \\
{\fontfamily{qcr}\selectfont
\indent \indent file\_results.write(u,scale) \\
\indent \indent file\_results.write(p,scale) \\ \\
}

\end{document}
