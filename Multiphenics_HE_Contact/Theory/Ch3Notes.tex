\documentclass[12pt,3p]{article}
\usepackage[utf8]{inputenc}
\usepackage[english]{babel}
 \usepackage[margin=0.5in]{geometry}
 \usepackage{amsmath}
 \usepackage{amsfonts} % Math fonts
\usepackage{mathtools}
\usepackage{enumitem}
\usepackage{physics}
\usepackage[round,numbers]{natbib}
\usepackage[colorlinks = false]{hyperref}


\numberwithin{equation}{section}
\begin{document}

\title{Ch3: Continuum Solid Mechanics and Weak Form}
\author{Ida Ang}
\date{\vspace{-5ex}}
\maketitle

\tableofcontents
\newpage

\section{Kinematics}
\textbf{Definitions} \\
$\mathbf{E_A}$: Orthogonal basis in the reference or Lagrange/Material configuration \\
$\mathbf{e_i}$: Orthogonal basis in the current or Eulerian/Spatial configuration \\
X: Reference or Lagrange/Material referring to basis $\mathbf{E_A}$ \\
x: Current or Eulerian/Spatial referring to basis $\mathbf{e_i}$\\
$\mathbf{X}$: position of particle X in the reference configuration \\
d\textbf{X}: material line element $\rightarrow$ d\textbf{x}: spatial line element  \\
F: Deformation Gradient \\
J: Jacobian determinant \\
\textbf{R}: Rotation Tensor, two-feild tensor connecting the reference to the current configuration\\
\textbf{U}: Right stretch tensor with a basis in the reference configuration \\
\textbf{V}: Left stretch tensor with a basis in the current configuration \\
d\textbf{A}: surface element in reference configuration $\rightarrow$ d\textbf{a}: surface element in current configuration \\
d\textbf{V}: volume element in reference configuration $\rightarrow$ d\textbf{v}: volume element in current configuration \\
\textbf{u}: Displacement vector \\
\textbf{H}: Displacement gradient with respect to \textbf{X}

\subsection{Motion and deformation gradient}
\begin{align}\label{31}
\begin{split}
x &= \varphi (\mathbf{X}) \quad \text{Position of particle X in the configuration } \varphi \\
  &= \varphi_t (X) = \varphi (X, t) \quad \text{temporal series of configurations}
\end{split}
\end{align}
The reference configuration can be defined with the following equation where $\mathbf{X}$ is the position of particle X in this configuration:
\begin{equation*}
\mathbf{X} = \varphi_0 (X)
\end{equation*}
For practical reasons, we don't need to differentiate between $\mathbf{X}$ and X, simplifying the notation to 
\begin{align}\label{33}
\begin{split}
\mathbf{x} &= \pmb{\varphi} (\mathbf{X}, t) \quad \text{Direct notation} \\
x_i &= \varphi_i (X_A, t)\quad \text{Indicial notation}
\end{split}
\end{align}
Deformation gradient \textbf{F}:
\begin{equation}\label{35}
d\mathbf{x} = \mathbf{F} d\mathbf{X}
\end{equation}
The components of the deformation gradient:
\begin{align}\label{36}
\begin{split}
\mathbf{F} &= \pdv{\mathbf{x}}{\mathbf{X}} = \text{Grad} \mathbf{x} = \text{Grad} \pmb{\varphi} (\mathbf{X}, t) \quad \text{Refer to Eq. } \ref{33}\\
F_{iA} (\mathbf{e_i} \otimes \mathbf{E_A}) &= \pdv{x_i}{X_A} (\mathbf{e_i} \otimes \mathbf{E_A}) = x_{i,A} (\mathbf{e_i} \otimes \mathbf{E_A})
\end{split}
\end{align}
The gradient in Eq. \ref{36} is a linear operator. To preserve the continuous structure during deformation, the mapping has to be one-to-one. \textbf{\textit{In other words, the deformation gradient cannot be singular.}} This is equivalent to: 
\begin{align}\label{37}
\begin{split}
J &= \det \mathbf{F} \neq 0 \quad \text{Non-singularity} \\
J &> 0 \quad \text{Exclude self-penetration} \\
\end{split}
\end{align}
Since $\mathbf{F}$ is non-singular, its inverse exists:
\begin{align}\label{38}
d \mathbf{X} = \mathbf{F}^{-1} d \mathbf{x}
\end{align}
Therefore the components of the inverse of the deformation gradient can be written as: 
\begin{align}\label{39}
\begin{split}
\mathbf{F}^{-1} &= \pdv{\mathbf{X}}{\mathbf{x}} \\
					 &= (F_{iA})^{-1} (\mathbf{E_A} \otimes \mathbf{e_i}) = \bigg( \pdv{x_i}{X_A} \bigg)^{-1} (\mathbf{E_A} \otimes \mathbf{e_i}) = \bigg( \pdv{X_A}{x_i} \bigg) (\mathbf{E_A} \otimes \mathbf{e_i}) 
\end{split}
\end{align}
By polar decomposition theorem:
\begin{align}\label{310}
\begin{split}
\mathbf{F} &= \mathbf{RU} = \mathbf{VR} \\
F_{iA} &= R_{iB} U_{BA} = V_{ik} R_{kA}
\end{split}
\end{align}
Transformations of area and volume elements can be derived. The equation for the transformation of area is known as Nanson's formula, derived below
\begin{align*}
\text{Property of dot products} &= \mathbf{u} \cdot \mathbf{Av} = \mathbf{v} \cdot \mathbf{A}^T \mathbf{u} = \mathbf{A}^T \mathbf{u} \cdot \mathbf{v} \\
d \mathbf{v} &= (\mathbf{n} da) \cdot \mathbf{N} d \mathbf{X} = d\mathbf{X} \cdot \mathbf{F}^T (\mathbf{n} da) = \mathbf{F}^T (\mathbf{n} da) \cdot d\mathbf{X} \\
d\mathbf{v} &= d\mathbf{V} \quad \text{Equate to dv in equation above}\\
\mathbf{F}^T (\mathbf{n} da) \cdot d\mathbf{X} &= J (\mathbf{N} dA) \cdot d\mathbf{X} \quad \text{Rearrange} \\
0 &= J (\mathbf{N} dA) \cdot d\mathbf{X} - 
\mathbf{F}^T (\mathbf{n} da) \cdot d\mathbf{X} \\
0 &= \big[ J (\mathbf{N} dA)- 
\mathbf{F}^T (\mathbf{n} da) \big] \cdot d\mathbf{X} \quad \text{Line element} \neq 0 \text{; therefore,} \\
0 &= J (\mathbf{N} dA) - 
\mathbf{F}^T (\mathbf{n} da) \\
\mathbf{F}^T (\mathbf{n} da) &= J (\mathbf{N} dA)
\end{align*}
Therefore, Nanson's formula can be stated
\begin{align}\label{311}
\begin{split}
\mathbf{n} da &= J \mathbf{F}^{-T} \mathbf{N} dA = J \mathbf{F}^{-T} d\mathbf{A} 
\end{split}
\end{align}
The transformation for volume elements is
\begin{align}\label{312}
\begin{split}
dv = J dV \rightarrow J &= \frac{dv}{dV} \\
							    &= 1 \quad \text{Incompressible, no change in volume from current to reference }
\end{split}
\end{align}
Displacement can be defined as
\begin{align}\label{313}
\begin{split}
\mathbf{u} (\mathbf{X},t) &= \mathbf{x} - \mathbf{X} \\
	&= \pmb{\varphi} (\mathbf{X},t) - \mathbf{X}
\end{split}
\end{align}
Relationship between the deformation gradient and displacement:
\begin{align}\label{314}
\begin{split}
\mathbf{x} &= \mathbf{u} + \mathbf{X}  \quad \text{Take gradient of equation} \\
\pdv{\mathbf{x}}{\mathbf{X}} &= \pdv{\mathbf{u}}{\mathbf{X}} + \pdv{\mathbf{X}}{\mathbf{X}} \\
\mathbf{F} &= \text{Grad} \mathbf{u} + \mathbf{I} \\
\mathbf{F} &= \mathbf{H} + \mathbf{I}
\end{split}
\end{align}


\subsection{Strain measures}
\textbf{Definitions} \\
\textbf{C}: Right Cauchy-Green tensor refers to the initial/reference configuration \\
\textbf{b}: Left Cauchy-Green tensor refers to the current configuration \\
\textbf{E}: Green-Lagrangian strain tensor (Reference) \\
\textbf{e}: Almansi strain tensor (Current)\\ \\
Definition of \textbf{C}:
\begin{align}\label{315}
\begin{split}
\mathbf{C} &= \mathbf{F}^T \mathbf{F} \\
C_{AB} &= F_{Ai}^T F_{iB} = F_{iA} F_{iB}\\
				&= \mathbf{U}^T \mathbf{U} = \mathbf{U}^2 
\end{split}
\end{align}
Definition of \textbf{b}:
\begin{align*}
\mathbf{b} &= \mathbf{F} \mathbf{F}^T \\
b_{ik} &= F_{iA} F_{kA}
\end{align*}
The Green-Lagrangian strain tensor can be defined as:
\begin{align}\label{316}
\begin{split}
\mathbf{E} &= \frac{1}{2} (\mathbf{F}^T \mathbf{F} - \mathbf{I}) = \frac{1}{2} (\mathbf{C} - \mathbf{I}) \\
E_{AB} &= \frac{1}{2} (F_{iA} F_{iB} - \delta_{AB})
\end{split}
\end{align}
The Almansi strain tensor is with repsect to the current configuration 
\begin{align}\label{321}
\begin{split}
\mathbf{e} &= \frac{1}{2} (\mathbf{I} - \mathbf{b}^{-1}) \\
e_{ik} &= \frac{1}{2} (\delta_{ik} - (F_{iA})^{-1} (F_{kA})^-1)
\end{split}
\end{align}
The Almansi strain tensor and the Green-Lagrangian strain tensor can be connected by teh following transformation:
\begin{equation}\label{322}
\mathbf{E} = \mathbf{F}^T \mathbf{e} \mathbf{F}
\end{equation}

\subsection{Transformations of vectors and tensors}
\textbf{Definitions} \\ 
Pull back operation: From current to initial/reference configuration \\
Push forward operation: From initial/reference to current configuration \\ 

The gradient of a scalar field G(\textbf{X}) = g(\textbf{x}), we have 
\begin{align}\label{323}
\begin{split}
\pdv{G}{X_A} &= \pdv{g}{x_i} \pdv{x_i}{X_A} \\
\text{Grad} G &= \text{grad} g \mathbf{F} \\
\text{Grad} G &= \mathbf{F}^T \text{grad} g
\end{split}
\end{align}
and,
\begin{align}\label{324}
\begin{split}
\pdv{g}{x_i} &= \pdv{G}{X_A} \pdv{X_A}{x_i} \\
\text{grad} g &= \text{Grad} G \mathbf{F}^{-1} \\
\text{grad} g &= \mathbf{F}^{-T}\text{Grad} G
\end{split}
\end{align}
In the same way, for the gradient of a vector field \textbf{W}(\textbf{X}) = \textbf{w} (\textbf{x}), we obtain:
\begin{align}\label{325}
\begin{split}
\text{Grad} \mathbf{W} &= \text{grad} \mathbf{w} \mathbf{F} \\ 
\text{Grad} \mathbf{W} \mathbf{F}^{-1} &= \text{grad} \mathbf{w}
\end{split}
\end{align}

\subsection{Time derivatives}
Some problems are time or history dependent. \\ \\
The velocity of a material (\textbf{X}) point in the reference configurations is defined by the material time derivative. 
\begin{align}
\begin{split}
\end{split}
\end{align}


\section{Balance Laws}
\textbf{Definitions}: \\
$\rho_o$: density in the initial/reference configuration \\
$\rho$: density in the current configuration \\
$\pmb{\sigma}$: Cauchy stress tensor \\

\subsection{Balance of mass}
The balance of mass is given by the following relationship:
\begin{align}\label{337}
m = \int_{B} \rho_o dV = \int_{\varphi(B)} \rho dV = const. 
\end{align}
We can assume along with Eq. \ref{312}
\begin{equation}\label{338}
J = \frac{\rho_o}{\rho} = \frac{dv}{dV}
\end{equation}

\subsection{Local balance of momentum and moments of momentum}
Linear momentum balance implies (in direct and indicial notation)
\begin{align*}
\frac{D}{Dt} \int_r \mathbf{v} \rho dv &= \int_{s} \mathbf{t} ds + \int_{r} \mathbf{b} \rho dv \\
\frac{D}{Dt} \int_r v_i \rho dv &= \int_{s} t_i ds + \int_{r} b_i \rho dv
\end{align*}
Using balance of mass (Eq. \ref{338}) where $dv = J dV$ and $\rho = \frac{\rho_o}{J}$
\begin{align*}
\frac{D}{Dt} \int_{R} V_i \frac{\rho_o}{J} J dV &= \int_{s} t_i ds + \int_{r} b_i \rho dv \\
\frac{D}{Dt} \int_{R} V_i \rho_o dV &= \int_{s} t_i ds + \int_{r} b_i \rho dv \quad \text{where } \frac{D(V_i)}{Dt} = A_i \\
\int_{R} A_i \rho_o dV &= \int_{s} t_i ds + \int_{r} b_i \rho dv \quad \text{transform to reference state} \\
\int_{r} a_i \rho dv &= \int_{s} t_i ds + \int_{r} b_i \rho dv \quad \text{recognize } t_i = \sigma_{ik} n_k \\
\int_{r} a_i \rho dv &= \int_{s} \sigma_{ik} n_k ds + \int_{r} b_i \rho dv \quad \text{Use divergence theorem} \\
\int_{r} a_i \rho dv &= \int_{r} \pdv{\sigma_{ik}}{x_k} dv + \int_{r} b_i \rho dv \\
0 &= \int_{r} \bigg( \pdv{\sigma_{ik}}{x_k} + b_i \rho - a_i \rho  \bigg) dv
\end{align*}
Therefore, the local balance of momentum can be written as
\begin{align}\label{339}
\begin{split}
\text{div} \pmb{\sigma} + \rho \bar{\textbf{b}} &= \rho \dot{\textbf{v}} \\
\text{div} \pmb{\sigma} + \rho \bar{\textbf{b}} &= \rho \textbf{a} \\
\pdv{\sigma_{ik}}{x_i} + \rho \bar{b_k} &= \rho \bar{a_i} \quad \text{Quasi-static problems ignore the inertial term} \\
\pdv{\sigma_{ik}}{x_i} + \rho \bar{b_k} &= 0 \quad \text{For small structures, neglect body forces} \\
\pdv{\sigma_{ik}}{x_i} &= 0 
\end{split}
\end{align}
Furthermore, we have Cauchy's Theorem
\begin{align}\label{340}
\begin{split}
\textbf{t} &= \pmb{\sigma}^T \textbf{n} \\
t_i &= \sigma_{ik} n_i
\end{split}
\end{align}
The local balance of angular momentum proves the symmetry of the Cauchy stress tensor
\begin{align}
\begin{split}
\pmb{\sigma} &= \pmb{\sigma}^T \\
\sigma_{ik} &= \sigma_{ki}
\end{split}
\end{align}

\subsection{First law of thermodynamics}

\subsection{Transformation to the initial configuration, different stress tensors}
\begin{align}
\begin{split}
\end{split}
\end{align}

\begin{align}
\begin{split}
\end{split}
\end{align}

\begin{align}
\begin{split}
\end{split}
\end{align}

\begin{align}
\begin{split}
\end{split}
\end{align}

\begin{align}
\begin{split}
\end{split}
\end{align}




\section{Weak Form of Balance of Momentum, Variational Principles}
\textbf{Definitions} \\
$\pmb{\eta}$: Virtual displacement or test function \\
$\mathbf{P}$: First Piola-Kirchoff stress tensor \\
$\mathbf{S}$: Second Piola-Kirchoff stress tensor \\
\textbf{E}: Green Lagrangian strain tensor \\
\textbf{C}: Right Cauchy-Green tensor \\
Grad $\pmb{\eta}$: The virtual variation of the deformation gradient, $\delta$ \textbf{F} \\

\subsection{Weak Form of Balance of Momentum in the Initial Configuration}
Local equilibrium equation
\begin{equation}
\text{Div} \mathbf{P} + \rho_o \mathbf{\bar{b}} = \rho_o \mathbf{\dot{v}}
\end{equation}
Convert to the weak form by multiplying by a test function, $\pmb{\eta}$. REQUIRES EDIT
\begin{align}\label{356}
\begin{split}
\pdv{P_{iA}}{X_A} \eta_i + \rho_o \bar{b}_i \eta_i &= \rho_o \dot{v}_i \eta_i \quad \text{Integrate over domain} \\
\int_{\mathcal{B}} \pdv{\mathbf{P}}{X} \pmb{\eta} dV + \int_{\mathcal{B}} \rho_o \mathbf{\bar{b}} \pmb{\eta} dV &= \int_{\mathcal{B}} \rho_o \mathbf{\dot{v}} \pmb{\eta} dV \quad \text{Integration by Parts} \\
\int_{\mathcal{B}} (\mathbf{P} \pmb{\eta})_{,X} dV - \int_{\mathcal{B}}\mathbf{P} \pdv{\pmb{\eta}}{X} dV + \int_{\mathcal{B}} \rho_o \mathbf{\bar{b}} \pmb{\eta} dV &= \int_{\mathcal{B}} \rho_o \mathbf{\dot{v}} \pmb{\eta} dV \quad \text{Divergence theorem} \\
\int_{\delta \mathcal{B}} \mathbf{P} \mathbf{N} \pmb{\eta} dA - \int_{\mathcal{B}}\mathbf{P} \pdv{\pmb{\eta}}{X} dV + \int_{\mathcal{B}} \rho_o (\mathbf{\bar{b}} - \mathbf{\dot{v}}) \pmb{\eta} dV &= 0 \quad \text{Traction} \\
\int_{\delta \mathcal{B}} \mathbf{\bar{t}} \pmb{\eta} dA - \int_{\mathcal{B}}\mathbf{P} \pdv{\pmb{\eta}}{X} dV + \int_{\mathcal{B}} \rho_o (\mathbf{\bar{b}} - \mathbf{\dot{v}}) \pmb{\eta} dV &= 0 \quad \text{Rearrange} \\
\int_{\mathcal{B}} \mathbf{P} \pdv{\pmb{\eta}}{X} dV - \int_{\mathcal{B}} \rho_o (\mathbf{\bar{b}} - \mathbf{\dot{v}}) \pmb{\eta} dV - \int_{\delta \mathcal{B}} \mathbf{\bar{t}} \pmb{\eta} dA &= 0 \quad \text{Direct notation} \\
\int_{\mathcal{B}} \mathbf{P} \cdot \text{Grad} \pmb{\eta} dV - \int_{\mathcal{B}} \rho_o (\mathbf{\bar{b}} - \mathbf{\dot{v}}) \cdot \pmb{\eta} dV - \int_{\delta \mathcal{B}} \mathbf{\bar{t}} \cdot \pmb{\eta} dA &= 0
\end{split}
\end{align}
We can exchange the first PK stress tensor with the second PK stress tensor:
\begin{align*}
\begin{split}
\int_{\mathcal{B}} \mathbf{P} \cdot \text{Grad} \pmb{\eta} dV - \int_{\mathcal{B}} \rho_o (\mathbf{\bar{b}} - \mathbf{\dot{v}}) \cdot \pmb{\eta} dV - \int_{\delta \mathcal{B}} \mathbf{\bar{t}} \cdot \pmb{\eta} dA &= 0 \quad \text{where } \mathbf{P = FS} \\
\int_{\mathcal{B}} \mathbf{FS} \cdot \text{Grad} \pmb{\eta} dV - "" - "" &= 0 \\
\int_{\mathcal{B}} \mathbf{S} \cdot \text{Grad} \pmb{\eta} \mathbf{F} dV - "" - "" &= 0 \\
\int_{\mathcal{B}} \mathbf{S} \cdot \mathbf{F}^T \text{Grad} \pmb{\eta} dV - "" - "" &= 0 \\
\int_{\mathcal{B}} \mathbf{S} \cdot \frac{1}{2} (\mathbf{F}^T \text{Grad} \pmb{\eta} + \text{Grad}^T \pmb{\eta} \mathbf{F}) dV - "" - "" &= 0 \\
\int_{\mathcal{B}} \mathbf{S} \cdot \delta \mathbf{E} dV + \int_{\mathcal{B}} \rho_o (\mathbf{\bar{b}} - \mathbf{\dot{v}}) \cdot \pmb{\eta} dV - \int_{\delta \mathcal{B}} \mathbf{\bar{t}} \cdot \pmb{\eta} dA &= 0
\end{split}
\end{align*}
where we know the following relationship:
\begin{align}
\begin{split}
\delta \mathbf{E} &= \frac{1}{2} (\mathbf{F}^T \text{Grad} \pmb{\eta} + \text{Grad}^T \pmb{\eta} \mathbf{F}) \\
			&= \frac{1}{2} (\mathbf{F}^T \delta \mathbf{F} + \delta \mathbf{F}^T \mathbf{F})
\end{split}
\end{align}
This makes use of the symmetry of \textbf{S}, so that the antisymmetric part of $\mathbf{F}^T \text{Grad} \pmb{\eta} $ disappears in the scalar product. This can be rewritten in direct and indicial notation: 
\begin{align}\label{359}
\begin{split}
G(\pmb{\varphi}, \pmb{\eta}) &= \underbrace{\int_{\mathcal{B}} \mathbf{S} \cdot \delta \mathbf{E} dV}_\text{Virtual internal work} + \underbrace{\int_{\mathcal{B}} \rho_o (\mathbf{\bar{b}} - \mathbf{\dot{v}}) \cdot \pmb{\eta} dV - \int_{\delta \mathcal{B}} \mathbf{\bar{t}} \cdot \pmb{\eta} dA}_\text{Virtual external work} = 0 \\
G(\pmb{\varphi}, \pmb{\eta}) &= \int_{\mathcal{B}} S_{AB} \delta E_{AB} dV + \int_{\mathcal{B}} \rho_o (\bar{b}_A - \dot{v}_A) \eta_A dV - \int_{\delta \mathcal{B}} \bar{t}_A \eta_A dA = 0
\end{split}
\end{align}

\subsection{Spatial Form of the Weak Formulation}
\textbf{Definitions} \\ 
Push-forward operation: from Reference/Initial (X) to Current (x) \\
Pull-forward operation: from Current (x) to Reference/Initial (X) \\ \\
Starting in the reference configuration, we can use several definitions to \textit{push-forward} into the current configuration: 
\begin{align*}
\int_{\mathcal{B}} \mathbf{P} \cdot \text{Grad} \pmb{\eta} dV - \int_{\mathcal{B}} \rho_o (\mathbf{\bar{b}} - \mathbf{\dot{v}}) \cdot \pmb{\eta} dV - \int_{\delta \mathcal{B}} \mathbf{\bar{t}} \cdot \pmb{\eta} dA &= 0 \\
\int_{\mathcal{B}} J \pmb{\sigma} \mathbf{F}^{-T} \cdot \text{Grad} \pmb{\eta} dV - \int_{\mathcal{B}} \rho_o (\mathbf{\bar{b}} - \mathbf{\dot{v}}) \cdot \pmb{\eta} dV - \int_{\delta \mathcal{B}} \mathbf{\bar{t}} \cdot \pmb{\eta} dA &= 0 \\
\int_{\mathcal{B}} J \pmb{\sigma} \cdot \text{Grad} \pmb{\eta} \mathbf{F}^{-1} dV - \int_{\mathcal{B}} \rho_o (\mathbf{\bar{b}} - \mathbf{\dot{v}}) \cdot \pmb{\eta} dV - \int_{\delta \mathcal{B}} \mathbf{\bar{t}} \cdot \pmb{\eta} dA &= 0 \\
\int_{\mathcal{B}} J \pmb{\sigma} \cdot \text{grad} \pmb{\eta} dV - \int_{\mathcal{B}} \rho_o (\mathbf{\bar{b}} - \mathbf{\dot{v}}) \cdot \pmb{\eta} dV - \int_{\delta \mathcal{B}} \mathbf{\bar{t}} \cdot \pmb{\eta} dA &= 0 \quad \text{where } J = \frac{dv}{dV} \\
\int_{\varphi(\mathcal{B})} \pmb{\sigma} \cdot \text{grad} \pmb{\eta} dv - \int_{\varphi(\mathcal{B})} \rho_o (\mathbf{\bar{b}} - \mathbf{\dot{v}}) \cdot \pmb{\eta} \frac{dv}{J} - \int_{\delta \mathcal{B}} \mathbf{\bar{t}} \cdot \pmb{\eta} dA &= 0 \quad \text{where } \rho_o = \rho J \\
\int_{\varphi(\mathcal{B})} \pmb{\sigma} \cdot \text{grad} \pmb{\eta} dv - \int_{\varphi(\mathcal{B})} \rho (\mathbf{\bar{b}} - \mathbf{\dot{v}}) \cdot \pmb{\eta} dv - \int_{\varphi (\delta \mathcal{B})} \mathbf{\bar{t}} \cdot \pmb{\eta} da &= 0
\end{align*}
where symmetry is applied to the traction term to transform the final term into the current configuration. This leaves the following relationship analogous to Eq. \ref{359} except in the current configuration. 
\begin{equation}\label{363}
g(\pmb{\varphi}, \pmb{\eta}) = \int_{\varphi(\mathcal{B})} \pmb{\sigma} \cdot \text{grad} \pmb{\eta} dv - \int_{\varphi(\mathcal{B})} \rho (\mathbf{\bar{b}} - \mathbf{\dot{v}}) \cdot \pmb{\eta} dv - \int_{\varphi (\delta \mathcal{B})} \mathbf{\bar{t}} \cdot \pmb{\eta} da = 0
\end{equation}

\subsection{Minimum of Total Potential Energy}
\textbf{Definitions} \\
W: Strain energy function stored within the body  \\ \\
The classical minimum principle of the total elastic potential can be formulated. One has to consider the potential energy of the conservative (path-independent) forces applied. 
\begin{align}\label{364}
\begin{split}
\Pi (\pmb{\varphi}) = \int_{B} [ W (\mathbf{C}) - \rho_o \bar{\mathbf{b}} \cdot \pmb{\varphi}] dV - \int_{\delta B_\sigma} \mathbf{\bar{t}} \cdot \pmb{\varphi} dA \Rightarrow MIN
\end{split}
\end{align}
The minimum can be computed as a variation of Eq. \ref{364}
\begin{align}\label{365}
\begin{split}
\end{split}
\end{align}

\begin{align}
\begin{split}
\end{split}
\end{align}

\begin{align}
\begin{split}
\end{split}
\end{align}

\end{document}