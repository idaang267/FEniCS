\documentclass[12pt,3p]{article}
\usepackage[utf8]{inputenc}
\usepackage[english]{babel}
 \usepackage[margin=0.5in]{geometry}
 \usepackage{amsmath}
 \usepackage{amsfonts} % Math fonts
\usepackage{mathtools}
\usepackage{enumitem}
\usepackage{physics}
\usepackage[round,numbers]{natbib}
\usepackage[colorlinks = false]{hyperref}


\numberwithin{equation}{section}
\begin{document}

\title{Ch6: Contact Boundary Value Problem and Weak Form}
\author{Ida Ang}
\date{\vspace{-5ex}}
\maketitle

\tableofcontents
\newpage

% 6.2 
\section{Frictionless Contact in Finite Deformation Problems}
\subsubsection{Definitions}
dV: element of volume \\
dA: area of the element \\
$B^\gamma$: domain occupied by body, $B^\gamma$ \\
$\tau^\gamma$: Kirchhoff stress \\
$\bar{\text{f}}^\gamma$: body force \\
$\bar{\text{t}}^\gamma$: surface traction on the boundary of $B^\gamma$ \\
$\varphi^\gamma$: Deformation unknown \\
$\eta^\gamma$: Test function or virtual displacement \\
$C_c$ are contract contributions associated with the active constraint set 

\subsubsection{Method}
Variational inequality that the solution to the contact problem has to fulfill:
\begin{align}\label{620}
\begin{split}
\sum_{\gamma = 1}^2 \int_{B^\gamma} \tau^\gamma \cdot \text{grad} (\eta^{\gamma} - \varphi^\gamma) dV \geq \sum_{\gamma = 1}^2 \int_{B^\gamma} \bar{\text{f}}^\gamma \cdot (\eta^\gamma - \varphi^\gamma) dV - \int_{\Gamma_\sigma^\gamma} \bar{\text{t}}^\gamma \cdot ( \eta^\gamma - \varphi^\gamma) dA \\
\end{split}
\end{align}
If the contact interface is known, we can write the weak form as an equality using Eq. \ref{620}
\begin{align}\label{624}
\begin{split}
\sum_{\gamma = 1}^2 \Bigg\{ \int_{B^\gamma} \tau^\gamma \cdot \text{grad} \eta^\gamma dV -  \int_{B^\gamma} \tau^\gamma \cdot \text{grad} \varphi^\gamma dV \Bigg\} &\geq \sum_{\gamma = 1}^2 \Bigg\{ \int_{B^\gamma} \bar{\text{f}}^\gamma \cdot \eta^\gamma dV - \int_{B^\gamma} \bar{\text{f}}^\gamma \cdot \varphi^\gamma dV \Bigg\} \\
& \quad \quad \quad -\int_{\Gamma_\sigma^\gamma} \bar{\text{t}}^\gamma \cdot \eta^\gamma dA + \int_{\Gamma_\sigma^\gamma} \bar{\text{t}}^\gamma \cdot \varphi^\gamma dA \\
\sum_{\gamma = 1}^2 \Bigg\{ \int_{B^\gamma} \tau^\gamma \cdot \text{grad} \eta^{\gamma} dV -\int_{B^\gamma} \bar{\text{f}}^\gamma \cdot \eta^\gamma dV \Bigg\} + \int_{\Gamma_\sigma^\gamma} \bar{\text{t}}^\gamma \cdot \eta^\gamma dA + &\\
\sum_{\gamma = 1}^2 \Bigg\{ - \int_{B^\gamma} \tau^\gamma \cdot \text{grad} \varphi^{\gamma} dV +\int_{B^\gamma} \bar{\text{f}}^\gamma \cdot \varphi^\gamma dV \Bigg\} - \int_{\Gamma_\sigma^\gamma} \bar{\text{t}}^\gamma \cdot \varphi^\gamma dA &\geq 0 \\
\sum_{\gamma = 1}^2 \Bigg\{ \int_{B^\gamma} \tau^\gamma \cdot \text{grad} \eta^{\gamma} dV -\int_{B^\gamma} \bar{\text{f}}^\gamma \cdot \eta^\gamma dV \Bigg\} + \int_{\Gamma_\sigma^\gamma} \bar{\text{t}}^\gamma \cdot \eta^\gamma dA + C_c &= 0 
\end{split}
\end{align}
Test functions are zero at the boundary $\Gamma_\varphi^\gamma$ where deformations are prescribed

% 6.3 
\section{Treatment of Contact Constraints}
% =================PENALTY METHOD===============================
\subsection{Penalty Method}
\subsubsection{Definitions}
$\epsilon_N, \epsilon_T$: penalty parameters \\
$g_N$: non-penetration function \\
$g_{\bar{N}}$: penetration function \\
$\mathbf{g_T}$: tangential gap function or tangential motion  \\
$\varphi_{\epsilon}$: approximate deformation field \\
$\varphi$: exact deformation field

\subsubsection{Method}
Non-penetration function:
\begin{align}\label{nonPenFun}
g_{N} = (\mathbf{x}^2 - \mathbf{\bar{x}}^1) \cdot \mathbf{\bar{n}}^1 \geq 0
\end{align}
Constraint condition for penetration function: 
\begin{align}\label{penFun}
g_{\bar{N}} = \begin{cases}
    			(\mathbf{x}^2 - \mathbf{\bar{x}}^1) \cdot \mathbf{\bar{n}}^1, & \text{if } (\mathbf{x}^2 - \mathbf{\bar{x}}^1) \cdot \mathbf{\bar{n}}^1 < 0 \\
    			0, & \text{otherwise}.
  		   \end{cases}
\end{align}
Adding a penalty term to $\Pi$
\begin{align}\label{631}
\begin{split}
\Pi_c^P = \frac{1}{2} \int_{\Gamma_c} \big[ \epsilon_N (g_{\bar{N}})^2 + \epsilon_T \mathbf{g_T} \cdot \mathbf{g_T} \big] dA \quad \epsilon_N, \epsilon_T > 0 
\end{split}
\end{align}
Take the variation of Eq. \ref{631} using the commutative property of dot products 
\begin{align*}
C_c^P &= \frac{1}{2} \int_{\Gamma_c} \big( 2 \epsilon_N g_{\bar{N}} \delta g_{\bar{N}} + \epsilon_T \mathbf{g_T} \cdot \delta \mathbf{g_T} + \epsilon_T \delta \mathbf{g_T} \cdot \mathbf{g_T} \big) dA \\
	&= \frac{1}{2} \int_{\Gamma_c} \big( 2 \epsilon_N g_{\bar{N}} \delta g_{\bar{N}} + 2 \epsilon_T \mathbf{g_T} \cdot \delta \mathbf{g_T} \big) dA \\
C_c^P &= \int_{\Gamma_c} \big(\epsilon_N g_{\bar{N}} \delta g_{\bar{N}} + \epsilon_T \mathbf{g_T} \cdot \delta \mathbf{g_T} \big) dA 
\end{align*}
As in the Lagrange multiplier method, we must distinguish between stick and slip:
\begin{align}\label{633}
\begin{split}
\text{Pure stick (?): }  C_c^{stick} &= \int_{\Gamma_c} \epsilon_N g_{\bar{N}} \delta g_{\bar{N}} dA,  \quad \epsilon_N > 0 \\
\text{Pure sliding: } C_c^{slip} &= \int_{\Gamma_c} \big(\epsilon_N g_{\bar{N}} \delta g_{\bar{N}} + \mathbf{t_T} \cdot \delta \mathbf{g_T} \big) dA, \quad \epsilon_N > 0
\end{split}
\end{align}
For the slip equation, an equation of the following type is used (refer to Chapter 5.2):
\begin{equation*}
\mathbf{t_T} = \epsilon_T \mathbf{g_T} 
\end{equation*}
Large number for the penalty parameters leads to an ill-conditioned numerical problem

\subsubsection{Remarks}
Similarly as for the Lagrange multiplier remark, we do not need to distinguish between the normal and tangential directions. 
\begin{align}
\begin{split}
C_c^{stick} &= \int_{\Gamma_c} \epsilon_N g_{\bar{N}} \delta g_{\bar{N}} dA \quad \text{Choose equal penalty parameters for all directions } \epsilon = \epsilon_N = \epsilon_T \\
		  &= \int_{\Gamma_c} \epsilon (g \cdot \delta g) dA \quad \text{Refer to Eq. \ref{629}} \\
		  &= \int_{\Gamma_c} \epsilon (\mathbf{x}^2 - \mathbf{\bar{x}}^1) \cdot (\eta^2 - \bar{\eta}^1) dA
\end{split}
\end{align}
A value of $\epsilon$ has to be chosen to prevent an ill-conditioned numerical problem. This will mean the constraint equation is only fulfilled approximately, resulting in a deformation field, $\varphi_\epsilon$ which differs from the exact field $\varphi$. Show:
\begin{equation*}
\abs{\abs{\varphi - \varphi_\epsilon}} \rightarrow 0 \text{ for } \epsilon \rightarrow \infty
\end{equation*}

% =================LAGRANGE MULTIPLIER METHOD===============================
\subsection{Lagrange multiplier method}
\subsubsection{Definitions}
$\Pi_c$: contact contributions from lagrange multiplier \\
$\lambda_N$: Lagrange multiplier in normal direction, analogous to contact pressure \\
$\lambda_T$: Lagrange multiplier in tangential direction \\
$g_N$: normal gap function \\
$\mathbf{g_T}$: tangential gap function or tangential motion $\rightarrow \mathbf{\dot{g_T}}$: relative sliding velocity \\
$p_N$: contact normal pressure \\
$\mu$: sliding friction coefficient \\
$\mathbf{t_T}$: Tangential stress vector 

\subsubsection{Method}
Using Lagrange multipliers to add constraints to a weak form:
\begin{align}\label{626}
\Pi_c^{LM} = \int_{\Gamma_c} (\lambda_N g_N + \mathbf{\lambda_T} \cdot \mathbf{g_T}) dA
\end{align}
Take the variation of Eq. \ref{626} where each term gives two variations:
\begin{align}
\begin{split}
C_c^{LM} &= \int_{\Gamma_c} \big( \underbrace{ \delta \lambda_N g_N + \lambda_N \delta g_N}_\text{Associated with normal contact} + \underbrace{\delta \mathbf{\lambda_T} \cdot \mathbf{g_T} + \mathbf{\lambda_T} \cdot \delta \mathbf{g_T}}_\text{Associated with tangent contact (stick)} \big) dA \\
C_c^{LM} &= \underbrace{\int_{\Gamma_c} (\lambda_N \delta g_N + \mathbf{\lambda_T} \cdot \delta \mathbf{g_T}) dA}_\text{Virtual work of lagrange multipliers} + \underbrace{\int_{\Gamma_c} (\delta \lambda_N g_N + \delta \mathbf{\lambda_T} \cdot \mathbf{g_T}) dA}_\text{Enforcement of constraints}
\end{split}
\end{align}

\subsubsection{Stick and slip as constraints}
In pure stick, there is no relative tangential velocity:
\begin{equation}\label{stick}
\mathbf{\dot{g}_T} = 0 \Leftrightarrow \mathbf{g_T} = 0 
\end{equation}
This constraint condition, formulated in the current configuration, imposes a nonlinear constraint equation on the motion in the contact interface \\ \\
When the tangential forces are above a certain limit, slipping will occur as the contacting surfaces move relative to each other. This is called Coulomb's friction law: 
\begin{equation}\label{slip}
\mathbf{t_T} = - \mu \abs{p_N} \frac{\dot{\mathbf{g_T}}}{\abs{\abs{\mathbf{\dot{g_T}}}}} \quad \text{if } \abs{\abs{\mathbf{t_T}}} > \mu \abs{p_N}
\end{equation}
Therefore, for pure stick (Eq. \ref{stick}), we obtain the following constraint equation in which $\lambda_T$ follows as a reaction. For sliding, a tangential stress vector, $t_T$, is determined from the constitutive law for frictional slip (Eq. \ref{slip}). ($\mathbf{\lambda_T} \cdot \delta \mathbf{g_T} \rightarrow \mathbf{t_T} \cdot \delta \mathbf{g_T}$)
\begin{align}\label{628}
\begin{split}
\text{Pure stick: } C_c^{stick} &= \int_{\Gamma_c} (\lambda_N \delta g_N + \mathbf{\lambda_T} \cdot \delta \mathbf{g_T}) dA + \int_{\Gamma_c} \delta \lambda_N g_N dA \\
\text{Pure sliding: } C_c^{slip} &= \int_{\Gamma_c} (\lambda_N \delta g_N + \mathbf{t_T} \cdot \delta \mathbf{g_T}) dA + \int_{\Gamma_c} \delta \lambda_N g_N dA 
\end{split}
\end{align}

\subsubsection{Remarks}
When only stick occurs on the contact interface, we do not distinguish between the normal and tangential directions. Therefore, the constraint condition is given in terms of the deformation of the slave point:
\begin{equation}\label{629}
\mathbf{x^2} - \mathbf{x^1} (\bar{\mathbf{\xi}}) = \mathbf{x^2} - \mathbf{\bar{x}}^1 = 0 
\end{equation}
\textit{Is this equivalent to gap function?} \\ \\
Therefore, the equation for contact contribution is altered:
\begin{align}\label{630}
\begin{split}
C_c^{stick} &= \int_{\Gamma_c} (\lambda_N \delta g_N + \mathbf{\lambda_T} \cdot \delta \mathbf{g_T}) dA + \int_{\Gamma_c} \delta \lambda_N g_N dA \\
		&= \int_{\Gamma_c} (\lambda \cdot \delta g + \lambda \cdot \delta g) dA + \int_{\Gamma_c} \delta \lambda \cdot g dA \\
		&= \int_{\Gamma_c} 2 ( \lambda \cdot \delta g) dA + \int_{\Gamma_c} \delta \lambda \cdot g dA \\
		&= \int_{\Gamma_c} \lambda \cdot (\eta^2 - \bar{\eta}^1) dA + \int_{\Gamma_c} \delta \lambda \cdot [ \mathbf{x^2} - \mathbf{x^1} (\bar{\mathbf{\xi}}) ] dA \quad \text{Not sure how to account for extra 2}\\
C_c^{stick} &= \int_{\Gamma_c} \lambda \cdot (\eta^2 - \bar{\eta}^1) dA + \int_{\Gamma_c} \delta \lambda \cdot [ \mathbf{x^2} - \mathbf{x^1} ] dA 
\end{split}
\end{align}
\textit{Does this indicate the following relationship for the variational gap?}
\begin{equation}\label{varGap}
\delta g = \eta^2 - \bar{\eta}^1
\end{equation}

% =================Augmented Lagrange Methods=====================
\subsection{Augmented Lagrange Methods}
The main idea of this method is combining the penalty method with Lagrange multiplier methods. 
\subsubsection{Methods}
The augmented Lagrange functional is introduced for normal contact: 
\begin{align}\label{658}
\Pi_{N}^{AM} = 
\begin{dcases}
    \int_{\Gamma_c} (\lambda_N g_N + \frac{1}{2} \epsilon_N g_N^2) d \Gamma & \text{for } \hat{\lambda}_N \leq 0, \\
    \int_{\Gamma_c} -\frac{1}{2 \epsilon_N} |\lambda_N|^2 d \Gamma               & \text{for } \hat{\lambda}_N > 0
\end{dcases}
\end{align}
where $ \hat{\lambda}_N \leq 0$ indicates that the gap is closed and $\hat{\lambda}_N > 0$ means the gap is open. We introduce the following augmented lagrangian term.
\begin{align*}
\hat{\lambda}_N = \lambda_N + \epsilon_N g_N
\end{align*}
Taking the variation of Eq. \ref{658} gives the following, where we recognize the augmented lagrangian 
\begin{align*}
C_{N}^{AM} &= 
\begin{dcases}
    \int_{\Gamma_c} (\delta \lambda_N g_N + \lambda_N \delta g_N + \epsilon_N g_N \delta g_N) d \Gamma & \text{for } \hat{\lambda}_N \leq 0, \\
    \int_{\Gamma_c} -\frac{1}{\epsilon_N} \lambda_N \delta \lambda_N d \Gamma               & \text{for } \hat{\lambda}_N > 0
\end{dcases} \\
&= 
\begin{dcases}
    \int_{\Gamma_c} \bigg( \delta \lambda_N g_N + (\lambda_N + \epsilon_N g_N) \delta g_N \bigg) d \Gamma & \text{for } \hat{\lambda}_N \leq 0, \\
    \int_{\Gamma_c} -\frac{1}{\epsilon_N} \lambda_N \delta \lambda_N d \Gamma               & \text{for } \hat{\lambda}_N > 0
\end{dcases} 
\end{align*}
Final equation for variation: 
\begin{align}\label{659}
C_{N}^{AM} &= 
\begin{dcases}
    \int_{\Gamma_c} \bigg( \delta \lambda_N g_N + \hat{\lambda}_N \delta g_N \bigg) d \Gamma & \text{for } \hat{\lambda}_N \leq 0, \\
    \int_{\Gamma_c} -\frac{1}{\epsilon_N} \lambda_N \delta \lambda_N d \Gamma               & \text{for } \hat{\lambda}_N > 0
\end{dcases}
\end{align}

\subsubsection{Remarks}
Uzawa algorithm formulation

 
\subsection{Direct Constraint Elimination}
Recall the definition of displacement: 
\begin{align}\label{Displacement}
\begin{split}
u &= \text{Current Coordinates} - \text{Reference Coordinates} \\
   &= x - X \\
X + u &= x  
\end{split}
\end{align}
One can formulate the inequality constraint, Eq. \ref{nonPenFun}, as an equality constraint:
\begin{align}\label{EqualityCon}
g_{N} &= (\mathbf{x}^2 - \mathbf{\bar{x}}^1) \cdot \mathbf{\bar{n}}^1 = 0 
\end{align}
This yields the following: 
\begin{align}\label{638}
\begin{split}
(\mathbf{x}^2 - \mathbf{\bar{x}}^1) \cdot \mathbf{\bar{n}}^1 &= 0 \\ 
\mathbf{x}^2 \cdot \mathbf{\bar{n}}^1 - \mathbf{\bar{x}}^1 \cdot \mathbf{\bar{n}}^1 &= 0 \\
\mathbf{x}^2 \cdot \mathbf{\bar{n}}^1 &= \mathbf{\bar{x}}^1 \cdot \mathbf{\bar{n}}^1 \quad \text{Eq. \ref{Displacement}} \\
(\mathbf{X}^2 + \mathbf{u}^2) \cdot \mathbf{\bar{n}}^1 &= (\mathbf{\bar{X}}^1 + \mathbf{\bar{u}}^1) \cdot \mathbf{\bar{n}}^1
\end{split}
\end{align}
This local elimination works well for node-to-node contact elements, but not for arbitrary discretization. In such a case, the point of departure is the Lagrange multiplier method using Eq. \ref{EqualityCon}:
\begin{align}\label{639}
\begin{split}
\int_{\Gamma_c} \delta \lambda_N g_N d \Gamma = \int_{\Gamma_C} \delta \lambda_N (\mathbf{x}^2 - \mathbf{\bar{x}}^1) \cdot \mathbf{\bar{n}}^1 d \Gamma &= 0 \\
\int_{\Gamma_C} \delta \lambda_N \mathbf{x}^2 \cdot \mathbf{\bar{n}}^1 d \Gamma - \int_{\Gamma_C} \delta \lambda_N \mathbf{\bar{x}}^1 \cdot \mathbf{\bar{n}}^1 d \Gamma &= 0 \\
\int_{\Gamma_C} \delta \lambda_N \mathbf{x}^2 \cdot \mathbf{\bar{n}}^1 d \Gamma &= \int_{\Gamma_C} \delta \lambda_N \mathbf{\bar{x}}^1 \cdot \mathbf{\bar{n}}^1 d \Gamma
\end{split}
\end{align}
Eq. \ref{639} eliminates the unknowns on one side of the contact interface $\Gamma_c$ preserving the positive-definite structure.

\subsection{Constitutive Equation of the Interface}
\subsubsection{Definitions}
$p_N$: normal pressure \\
$g_N$: non-penetration function \\
$c_N$: constitutive material parameter determined from experiments \\
d: distance function \\
n: constitutive material parameter determined from experiments usually within 2 $\leq$ n $\leq$ 3.33 

\subsubsection{Method}
Must read chapter 5.1 and 5.2 for more understanding. \\ \\
We do not add a constraint equation as in the case of the Lagrange multiplier or penalty method: The contact term when the constraint is active is given by:
\begin{align}\label{640}
\begin{split}
C_c = \int_{\Gamma_c} (p_N \delta g_N + \mathbf{t_T} \cdot \delta \mathbf{g_T}) dA
\end{split}
\end{align}
If we introduce the constitutive equation for normal pressure, we can obtain the standard penalty method by n = -1 and $\zeta = 0$
\begin{align}\label{512}
\begin{split}
p_N = c_N d^n &= c_N (\zeta - g_N)^{-1/n} \quad \text{Obtaining standard penalty method} \\
			&= c_N (- g_N)^1 \\
		 p_N &= c_N g_N
\end{split}
\end{align}
Choosing n = -1 is artificial. Furthermore, the use of this formulation usually leads to ill-conditioned system of equations

\subsection{Nitsche Method}
\subsubsection{Definitions}
$p_N^\gamma$: We have contact of two bodies, so the contact pressure/stress is associated with the body \\$B^\gamma$: body \\
$\epsilon_N$: Penalty parameter to prevent ill-conditioning (this is not a penalty method) \\
$N_c^\gamma$: vector which describes the projection of the stress field at the boundary in the normal direction \\
$\nabla^s \mathbf{u}$: symmetrical displacement gradient \\ 
$\pmb{\eta}^{\gamma}$: Variation related to body \\ \\
Recall the definition for stress
\begin{equation}\label{Stress}
\pmb{\sigma} (\pmb{u}^\gamma) = \mathbb{C} [\pmb{\varepsilon} (\mathbf{u}^\gamma)]
\end{equation}
Recall the definition for traction and use the definition for stress (Eq. \ref{Stress})
\begin{align}\label{CauchyT}	% Also equation 647
\begin{split}
\mathbf{t} &= \pmb{\sigma^T} \mathbf{n} \\
	   t_i &= \sigma_{ik} n_k \\
	   \mathbf{t}^\gamma &= \pmb{\sigma} (\mathbf{u}^\gamma) \mathbf{n}^\gamma \\
	   				&= \mathbb{C} [\pmb{\varepsilon} (\mathbf{u}^\gamma)] \mathbf{n}^\gamma
\end{split}
\end{align}

\subsubsection{Method}
Instead of Lagrange multipliers, the stress vector in the contact interface is computed from the stress field of the bodies. This leads to another set of boundary terms: 
\begin{align}\label{641}
\Pi_c^N = - \int_{\gamma_c} \frac{1}{2} (p_N^1 + p_N^2) g_N dA + \frac{1}{2} \int_{\Gamma_c} \epsilon_N [g_N]^2 dA
\end{align}
In the first term, the contribution from the contact pressure is expressed in the formulation in an average/mean sense. In the last term, the standard penalty term is introduced to prevent ill-conditioning of the global equation system (Eq. \ref{631}). \\ \\
The contact stresses are defined in terms of the displacement field 
\begin{align}\label{642}
\begin{split}
p_N^\gamma = \mathbf{t}^\gamma \cdot \mathbf{\bar{n}}^\gamma &= \mathbf{\bar{n}}^\gamma \cdot \mathbf{t}^\gamma \quad \text{Eq. \ref{CauchyT}}\\
		     &= \mathbf{\bar{n}}^\gamma \cdot \big[ \pmb{\sigma} (\mathbf{u}^\gamma) \mathbf{n}^\gamma \big] \quad \text{Eq. \ref{Stress}} \\
		     &= \mathbf{\bar{n}}^\gamma \cdot \big[ \mathbb{C} [\pmb{\varepsilon} (\mathbf{u}^\gamma)] \mathbf{n}^\gamma \big] \quad \text{Not sure}\\
p_N^\gamma &= \mathbf{N}_c^\gamma \cdot \nabla^s \mathbf{u}^\gamma
\end{split}
\end{align}
Where we can express the variation of normal pressure: 
\begin{align}\label{644}
\begin{split}
p_N^\gamma &= \mathbf{\bar{n}}^\gamma \cdot \big[ \mathbb{C} [\pmb{\varepsilon} (\mathbf{u}^\gamma)] \mathbf{n}^\gamma \big] = \mathbf{N}_c^\gamma \cdot \nabla^s \mathbf{u}^\gamma \\
\delta p_N^\gamma &= \mathbf{\bar{n}}^\gamma \cdot \big[ \mathbb{C} [\pmb{\varepsilon} (\pmb{\eta}^\gamma)] \mathbf{n}^\gamma \big] = \mathbf{N}_c^\gamma \cdot \nabla^s \pmb{\eta}^\gamma
\end{split}
\end{align}
We can calculate the variation of Eq. \ref{641} and use Eq. \ref{644}
\begin{align}\label{643}
\begin{split}
C_c^N = &- \int_{\Gamma_c} \frac{1}{2} (\delta p_N^1 + \delta p_N^2) g_N dA - \int_{\Gamma_c} \frac{1}{2} (p_N^1 + p_N^2) \delta g_N dA + \int_{\Gamma_c} \epsilon_N g_N \delta g_N dA \\
C_c^N = &- \int_{\Gamma_c} \frac{1}{2} \big( \mathbf{N}_c^1 \cdot \nabla^s \pmb{\eta}^1 + \mathbf{N}_c^2 \cdot \nabla^s \pmb{\eta}^2 \big) g_N dA - \int_{\Gamma_c} \frac{1}{2} \big( \mathbf{N}_c^1 \cdot \nabla^s \mathbf{u}^1 + \mathbf{N}_c^2 \cdot \nabla^s \mathbf{u}^2 \big) \delta g_N dA \\
 &+ \int_{\Gamma_c} \epsilon_N g_N \delta g_N dA
\end{split}
\end{align}
For the stick case, recall that stick does not differentiate between normal and tangential, and the penalty parameter $\epsilon = \epsilon_N$
\begin{align}\label{646}
\begin{split}
\Pi_c^N &= - \int_{\Gamma_c} \frac{1}{2} (p_N^1 + p_N^2) g_N dA + \frac{1}{2} \int_{\Gamma_c} \epsilon_N [g_N]^2 dA \\
		&= - \int_{\Gamma_c} \frac{1}{2} (p^1 + p^2) g dA + \frac{1}{2} \int_{\Gamma_c} \epsilon g^2 dA \\
		&= - \int_{\Gamma_c} \frac{1}{2} (\mathbf{t}^1 + \mathbf{t}^2) \cdot (\mathbf{x}^2 - \mathbf{\bar{x}}^1) dA + \int_{\Gamma_c} \epsilon (\mathbf{x}^2 - \mathbf{\bar{x}}^1) \cdot (\mathbf{x}^2 - \mathbf{\bar{x}}^1) dA
\end{split}
\end{align}
Taking the variation of Eq. \ref{646}
\begin{align}\label{648}
\begin{split}
C_c^N &= - \int_{\Gamma_c} \frac{1}{2} (\delta \mathbf{t}^1 + \delta \mathbf{t}^2) \cdot (\mathbf{x}^2 - \mathbf{x}^1) dA - \int_{\Gamma_c} \frac{1}{2} (\mathbf{t}^1 + \mathbf{t}^2) \cdot (\delta \pmb{x}^2 - \delta \pmb{x}^1) dA + \int_{\Gamma_c} \epsilon (\mathbf{x}^2 - \mathbf{\bar{x}}^1) \cdot (\delta \pmb{x}^2 - \delta \pmb{x}^1) dA \\
	  &= - \int_{\Gamma_c} \frac{1}{2} (\delta \mathbf{t}^1 + \delta \mathbf{t}^2) \cdot (\mathbf{x}^2 - \mathbf{x}^1) dA - \int_{\Gamma_c} \frac{1}{2} (\mathbf{t}^1 + \mathbf{t}^2) \cdot (\pmb{\eta}^2 - \pmb{\eta}^1) dA + \int_{\Gamma_c} \epsilon (\mathbf{x}^2 - \mathbf{\bar{x}}^1) \cdot (\pmb{\eta}^2 - \pmb{\eta}^1) dA
\end{split}
\end{align}
Traction and the variation of traction are as follows 
\begin{align}\label{649}
\begin{split}
\mathbf{t}^\gamma &= \pmb{\sigma} (\mathbf{u}^\gamma) \mathbf{n}^\gamma = \mathbb{C} [\pmb{\varepsilon} (\mathbf{u}^\gamma)] \mathbf{n}^\gamma \\
\delta \mathbf{t}^\gamma &=  \pmb{\sigma} (\pmb{\eta}^\gamma) \mathbf{n}^\gamma = \mathbb{C} [\pmb{\varepsilon} (\pmb{\eta}^\gamma)] \mathbf{n}^\gamma
\end{split}
\end{align}
In the nonlinear case, the Nitsche method becomes more complex, since the variations of the tractions depend upon the type of constitutive equations used to model the solid

\subsection{Perturbed Lagrange Formulation}
\subsubsection{Definitions}
$\Pi^\alpha$: Total energy of the two bodies coming into contact \\
$\Pi_c^{PL}$: Energy related to the contact interface

\subsubsection{Method}
This formulation combines both penalty and Lagrange multiplier methods in a mixed formulation:
\begin{equation}\label{650}
\Pi^{PL} = \sum_{\alpha = 1}^2 \Pi^\alpha + \Pi_c^{PL}
\end{equation}
The last term is given by:
\begin{align}\label{651}
\begin{split}
\Pi_c^{PL} = \int_{\Gamma_c} \bigg[ \lambda_N g_N - \frac{1}{2 \epsilon_N} \lambda_N^2 + \pmb{\lambda}_T \cdot \pmb{g}_T - \frac{1}{2 \epsilon_T} \pmb{\lambda}_T \cdot \pmb{\lambda}_T \bigg] d \Gamma
\end{split}
\end{align}
The Lagrange multiplier term is regularized by the second term in the integral, which can be viewed as the complementary energy due to the Lagrange multiplier. \\ \\
With some term grouping, the variation leads to the following:
\begin{align}\label{652}
\begin{split}
C_c^{PL} &= \int_{\Gamma_c} \bigg[ \delta \lambda_N g_N + \lambda_N \delta g_N - \frac{1}{\epsilon_N} \delta \lambda_N \lambda_N + \pmb{\lambda}_T \cdot \delta \pmb{g}_T + \delta \pmb{\lambda}_T \cdot \pmb{g}_T - \frac{1}{2 \epsilon_T} \pmb{\lambda}_T \cdot \delta \lambda_T - \frac{1}{2 \epsilon_T} \delta \pmb{\lambda}_T \cdot \pmb{\lambda}_T \bigg] d \Gamma \\
		&= \int_{\Gamma_c} \bigg[ \lambda_N \delta g_N + \delta \lambda_N g_N - \frac{1}{\epsilon_N} \delta \lambda_N \lambda_N + \pmb{\lambda}_T \cdot \delta \pmb{g}_T + \delta \pmb{\lambda}_T \cdot \pmb{g}_T - \frac{1}{\epsilon_T} \delta \pmb{\lambda}_T \cdot \pmb{\lambda}_T \bigg] d \Gamma \\
C_c^{PL} &= \int_{\Gamma_c} \bigg[ \underbrace{\lambda_N \delta g_N}_\text{Lagrange Formulation} + \underbrace{\delta \lambda_N \big( g_N - \frac{1}{\epsilon_N} \lambda_N \big)}_\text{Constitutive Law} + \underbrace{\pmb{\lambda}_T \cdot \delta \pmb{g}_T}_\text{Lagrange Formulation} + \underbrace{\delta \pmb{\lambda}_T \cdot \big( \pmb{g}_T - \frac{1}{\epsilon_T} \pmb{\lambda}_T \big)}_\text{Constitutive Law} \bigg] d \Gamma
\end{split}
\end{align}

\subsubsection{Remarks}
The first and third terms are associated with the Lagrange multiplier formulation. The second and fourth terms yield the constitutive laws $\lambda_N = \epsilon_N g_N$ and $\pmb{\lambda_T} = \epsilon_T \pmb{g}_T$ \\ \\
Can obtain penalty formulation and classical Lagrange multiplier formulation \\ \\
This formulation is only valid for the frictionless and the stick cases. For sliding, we have to use an incremental constitutive equation like Coulomb's law which cannot be stated in the form of a complementary energy

\subsection{Barrier Method}
\subsubsection{Definitions}
$g_{N_{+}}$: Indicates that one body does not penetrate the other one (Eq. \ref{nonPenFun}) \\
$\epsilon_N$: barrier parameter ($\epsilon_N < 0$) \\
b: barrier function 

\subsubsection{Method}
Constraint functional of type 
\begin{equation}\label{653}
\Pi_c^B = \epsilon_N \int_{\Gamma_c} b (g_{N_{+}}) d \Gamma
\end{equation}
The barrier function can be chosen as:
\begin{align}\label{654.655}
b(g_N) &= - \frac{1}{g_N} \\
b(g_N) &= - \ln [ \min \{1, -g_N\}]
\end{align}
The second function in Eq. \ref{654.655} is non-differentiable because of the \{\} expression. Taking the variation using the first function in Eq. \ref{654.655} yields 
\begin{align}\label{656}
\begin{split}
\Pi_c^B &= \epsilon_N \int_{\Gamma_c} - \frac{1}{g_N} d \Gamma \\
C_c^B &= \epsilon_N \int_{\Gamma_c} + \frac{1}{g_N^2} \delta g_N d \Gamma = \int_{\Gamma_c} \frac{\epsilon_N}{g_N^2} \delta g_N  d \Gamma
\end{split}
\end{align}

\subsubsection{Remarks}
The advantage of this method is that all constraints are always active and no on/off switch has to be applied to distinguish between active and passive constraints, as in the Lagrange multiplier or penalty methods.  \\ \\
The disadvantage is that a feasible starting point must be found that fulfills all constraints, and ill conditioning can occur (similarly to the penalty method). \\ \\
This method isn't used often in computational contact mechanics, but can be used in combination with augmented Lagrange techniques (next section). \\ \\
The barrier method has the displacement field as the unknown. One can construct \textit{primal-dual methods} based on the barrier method. Primal-dual methods are used in optimization problems but not extensively for contact mechanics. 

\subsection{Cross-constraint Method}
\subsubsection{Definitions}

\subsubsection{Methods}


\section{Linearization of the Contact Contribution}
The Newton scheme relies on linearization of the weak form. Here is a general form of the contact contribution where normal and tangential contact terms are distinguished: 
\begin{align}\label{6113}
\begin{split}
C_c^G &= \int_{\Gamma_c} c_N d \Gamma + \int_{\Gamma_c} c_T d \Gamma \\
	&= \int_{\Gamma_c} c_N (\lambda_N, g_N, \delta \lambda_N, \delta g_N) d \Gamma + \int_{\Gamma_c} c_T (\pmb{\lambda}_T, \pmb{g}_T, \delta \pmb{\lambda}_T, \delta \pmb{g}_T; p_N)d \Gamma \\
\end{split}
\end{align}
where $c_N$ and $c_T$ are functions of variations and variables. \\ \\
Due to geometrical nonlinearities of the finite deformation problem, the variations $\delta g_N$ and $\delta \pmb{g_T}$ also contribute to the linearization, often resulting in very complex terms. \\ \\
The linearization of the Lagrange parameters $\lambda_N$ and $\pmb{\lambda}_T$ are zero \\ \\
Variation of the tangential contribution is complicated, because the stick and slip cases have to be distinguished between. 

\subsection{Normal Contact}
The linearization of $c_n$ can be computed: 
\begin{align}\label{6114}
\begin{split}
\pdv{c_N}{\pmb{u}} \Delta \pmb{u} &= \pdv{c_N}{\lambda_N} \Delta \lambda_N + \pdv{c_N}{g_N} \Delta g_N + \pdv{c_N}{\delta \lambda_N} \Delta \delta \lambda_N + \pdv{c_N}{\delta g_N} \Delta \delta g_N \quad \text{Where } \delta \lambda_N = 0 \\
\pdv{c_N}{\pmb{u}} \Delta \pmb{u} &= \pdv{c_N}{\lambda_N} \Delta \lambda_N + \pdv{c_N}{g_N} \Delta g_N + \pdv{c_N}{\delta g_N} \Delta \delta g_N 
\end{split}
\end{align}

As an example, say we use the Lagrange multiplier method where the normal contact term is given by the following equation: 
\begin{align}
\begin{split}
c_N &= \lambda_N \delta g_N + \delta \lambda_N g_N \\
\pdv{c_N}{\lambda_N} &= \delta g_N \quad \pdv{c_N}{g_N} = \lambda_N \\
\pdv{c_N}{\delta \lambda_N} &= g_N \quad \pdv{c_N}{\delta g_N} = \delta \lambda_N
\end{split}
\end{align}
Therefore, the linearization of $c_n$ for the Lagrange multiplier method would be: 
\begin{align}
\begin{split}
\pdv{c_N}{\pmb{u}} \Delta \pmb{u} &= \pdv{c_N}{\lambda_N} \Delta \lambda_N + \pdv{c_N}{g_N} \Delta g_N + \pdv{c_N}{\delta \lambda_N} \Delta \delta \lambda_N + \pdv{c_N}{\delta g_N} \Delta \delta g_N \quad \text{Where } \\
			&= \delta g_N \Delta \lambda_N + \lambda_N \Delta g_N + g_N \Delta \delta \lambda_N + \delta \lambda_N \Delta \delta g_N \quad \text{Where } \delta \lambda_N = 0 \\
			&= \delta g_N \Delta \lambda_N + \lambda_N \Delta g_N
\end{split}
\end{align}
\textbf{Definitions from Chapter 4} \\
Non-penetration function:
\begin{equation}\label{46}
g_N = (\mathbf{x}^2 - \mathbf{\bar{x}}^1) \cdot \bar{\mathbf{n}}^1
\end{equation}
We then have to take into account the projection of point x2 onto the master surface parameterized by the convective coordinates $\xi^1$ and $\xi^2$
\begin{equation}\label{428}
\delta g_N = [ \pmb{\eta}^2 - \pmb{\bar{\eta}}^1 - \mathbf{\bar{x}}^1_{,\alpha} \delta \xi^\alpha ] \cdot \mathbf{\bar{n}}_1 + [\mathbf{x}^2 - \mathbf{\bar{x}}^1] \cdot \delta \mathbf{\bar{n}}^1
\end{equation}
Where $\pmb{\eta}^\alpha = \delta \mathbf{x}^\alpha$
\begin{equation}\label{429}
\delta g_N = [\pmb{\eta}^2 - \pmb{\bar{\eta}}^1] \cdot \mathbf{\bar{n}}^1
\end{equation}
\textbf{Using these definitions}, the linearization of $\Delta g_N$ has the same structure as the variation of $g_N$ (Eq. \ref{429}) where we exchange $\pmb{\eta}^\alpha$ by $\Delta \mathbf{u}^\alpha$
\begin{align}\label{6115}
\begin{split}
\delta g_N &= [\Delta \mathbf{u}^2 - \Delta \mathbf{\bar{u}}^1] \cdot \mathbf{\bar{n}}^1 \\
		&=  [\Delta \mathbf{u}^2 - \Delta \mathbf{\bar{u}}^1 (\bar{\xi}^1, \bar{\xi}^2)] \cdot \mathbf{\bar{n}}^1 (\bar{\xi}^1, \bar{\xi}^2)
\end{split}
\end{align}
The linearization of the gap in the normal direction has to be computed from the full variation, Eq. \ref{428}. 
Start with Eq. \ref{46}
\begin{align}\label{6116}
\begin{split}
g_N &= (\mathbf{x}^2 - \mathbf{\bar{x}}^1) \cdot \bar{\mathbf{n}}^1 \\
g_N \bar{\mathbf{n}}^1 &= \mathbf{x}^2 - \mathbf{\bar{x}}^1 (\bar{\xi}^1, \bar{\xi}^2) 
\end{split}
\end{align}
Use this relationship to obtain $\delta g_N$ when multiplied by $\mathbf{\bar{n}}^1$
\begin{align}
\begin{split}
\delta g_N &= [ \pmb{\eta}^2 - \pmb{\bar{\eta}}^1 - \mathbf{\bar{x}}^1_{,\alpha} \delta \xi^\alpha ] \cdot \mathbf{\bar{n}}_1 + [\mathbf{x}^2 - \mathbf{\bar{x}}^1] \cdot \delta \mathbf{\bar{n}}^1 \\
\delta g_N - [\mathbf{x}^2 - \mathbf{\bar{x}}^1] \cdot \delta \mathbf{\bar{n}}^1 &= [ \pmb{\eta}^2 - \pmb{\bar{\eta}}^1 - \mathbf{\bar{x}}^1_{,\alpha} \delta \xi^\alpha ] \cdot \mathbf{\bar{n}}_1 \quad \text{Use Eq. \ref{6116}}\\
\delta g_N \mathbf{\bar{n}}_1 - g_N \bar{\mathbf{n}}^1 \cdot \delta \mathbf{\bar{n}}^1  \mathbf{\bar{n}}_1 &= \pmb{\eta}^2 - \pmb{\bar{\eta}}^1 - \mathbf{\bar{x}}^1_{,\alpha} \delta \xi^\alpha \\
\delta g_N \mathbf{\bar{n}}_1 - g_N \delta \mathbf{\bar{n}}^1 (\bar{\mathbf{n}}^1 \cdot \mathbf{\bar{n}}_1) &= \pmb{\eta}^2 - \pmb{\bar{\eta}}^1 - \mathbf{\bar{x}}^1_{,\alpha} \delta \xi^\alpha \\
\delta g_N \mathbf{\bar{n}}_1 - g_N \delta \mathbf{\bar{n}}^1 &= \pmb{\eta}^2 - \pmb{\bar{\eta}}^1 - \mathbf{\bar{x}}^1_{,\alpha} \delta \xi^\alpha
\end{split}
\end{align}
$\xi$: Deformation dependent surface coordinates 
$\pmb{\bar{\xi}}$: Values related to projection point denoted by bar \\



\begin{align}
\begin{split}
\end{split}
\end{align}

\begin{align}
\begin{split}
\end{split}
\end{align}
\end{document}