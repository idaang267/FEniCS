\documentclass[12pt,3p]{article}
\usepackage[utf8]{inputenc}
\usepackage[english]{babel}
 \usepackage[margin=0.5in]{geometry}
 \usepackage{amsmath}
 \usepackage{amsfonts} % Math fonts
\usepackage{mathtools}
\usepackage{enumitem}
\usepackage{physics}
\usepackage[round,numbers]{natbib}
\usepackage[colorlinks = false]{hyperref}


\numberwithin{equation}{section}
\begin{document}

\title{Ch6: Contact Boundary Value Problem and Weak Form \\
	\large{Normal Contact Only}}
\author{Ida Ang}
\date{\vspace{-5ex}}
\maketitle

\tableofcontents
\newpage

% 6.2 
\section{Frictionless Contact in Finite Deformation Problems}
\subsubsection{Definitions}
dV: element of volume \\
dA: area of the element \\
$B^\gamma$: domain occupied by body, $B^\gamma$ \\
$\tau^\gamma$: Kirchhoff stress \\
$\bar{\text{f}}^\gamma$: body force \\
$\bar{\text{t}}^\gamma$: surface traction on the boundary of $B^\gamma$ \\
$\varphi^\gamma$: Deformation unknown \\
$\eta^\gamma$: Test function or virtual displacement \\
$C_c$ are contract contributions associated with the active constraint set 

\subsubsection{Method}
Variational inequality that the solution to the contact problem has to fulfill:
\begin{align}\label{620}
\begin{split}
\sum_{\gamma = 1}^2 \int_{B^\gamma} \tau^\gamma \cdot \text{grad} (\eta^{\gamma} - \varphi^\gamma) dV \geq \sum_{\gamma = 1}^2 \int_{B^\gamma} \bar{\text{f}}^\gamma \cdot (\eta^\gamma - \varphi^\gamma) dV - \int_{\Gamma_\sigma^\gamma} \bar{\text{t}}^\gamma \cdot ( \eta^\gamma - \varphi^\gamma) dA \\
\end{split}
\end{align}
If the contact interface is known, we can write the weak form as an equality using Eq. \ref{620}
\begin{align}\label{624}
\begin{split}
\sum_{\gamma = 1}^2 \Bigg\{ \int_{B^\gamma} \tau^\gamma \cdot \text{grad} \eta^\gamma dV -  \int_{B^\gamma} \tau^\gamma \cdot \text{grad} \varphi^\gamma dV \Bigg\} &\geq \sum_{\gamma = 1}^2 \Bigg\{ \int_{B^\gamma} \bar{\text{f}}^\gamma \cdot \eta^\gamma dV - \int_{B^\gamma} \bar{\text{f}}^\gamma \cdot \varphi^\gamma dV \Bigg\} \\
& \quad \quad \quad -\int_{\Gamma_\sigma^\gamma} \bar{\text{t}}^\gamma \cdot \eta^\gamma dA + \int_{\Gamma_\sigma^\gamma} \bar{\text{t}}^\gamma \cdot \varphi^\gamma dA \\
\sum_{\gamma = 1}^2 \Bigg\{ \int_{B^\gamma} \tau^\gamma \cdot \text{grad} \eta^{\gamma} dV -\int_{B^\gamma} \bar{\text{f}}^\gamma \cdot \eta^\gamma dV \Bigg\} + \int_{\Gamma_\sigma^\gamma} \bar{\text{t}}^\gamma \cdot \eta^\gamma dA + &\\
\sum_{\gamma = 1}^2 \Bigg\{ - \int_{B^\gamma} \tau^\gamma \cdot \text{grad} \varphi^{\gamma} dV +\int_{B^\gamma} \bar{\text{f}}^\gamma \cdot \varphi^\gamma dV \Bigg\} - \int_{\Gamma_\sigma^\gamma} \bar{\text{t}}^\gamma \cdot \varphi^\gamma dA &\geq 0 \\
\sum_{\gamma = 1}^2 \Bigg\{ \int_{B^\gamma} \tau^\gamma \cdot \text{grad} \eta^{\gamma} dV -\int_{B^\gamma} \bar{\text{f}}^\gamma \cdot \eta^\gamma dV \Bigg\} + \int_{\Gamma_\sigma^\gamma} \bar{\text{t}}^\gamma \cdot \eta^\gamma dA + C_c &= 0 
\end{split}
\end{align}
Test functions are zero at the boundary $\Gamma_\varphi^\gamma$ where deformations are prescribed

% 6.3 
\section{Treatment of Contact Constraints}
% =================PENALTY METHOD===============================
\subsection{Penalty Method}
\subsubsection{Definitions}
$\epsilon_N$: penalty parameter \\
$g_N$: non-penetration function \\
$g_{\bar{N}}$: penetration function \\
$\varphi_{\epsilon}$: approximate deformation field \\
$\varphi$: exact deformation field

\subsubsection{Method}
Non-penetration function:
\begin{align}\label{nonPenFun}
g_{N} = (\mathbf{x}^2 - \mathbf{\bar{x}}^1) \cdot \mathbf{\bar{n}}^1 \geq 0
\end{align}
Constraint condition for penetration function: 
\begin{align}\label{penFun}
g_{\bar{N}} = \begin{cases}
    			(\mathbf{x}^2 - \mathbf{\bar{x}}^1) \cdot \mathbf{\bar{n}}^1, & \text{if } (\mathbf{x}^2 - \mathbf{\bar{x}}^1) \cdot \mathbf{\bar{n}}^1 < 0 \\
    			0, & \text{otherwise}.
  		   \end{cases}
\end{align}
Adding a penalty term to $\Pi$
\begin{align}\label{631}
\begin{split}
\Pi_c^P = \frac{1}{2} \int_{\Gamma_c} \epsilon_N (g_{\bar{N}})^2 dA \quad \epsilon_N > 0 
\end{split}
\end{align}
Take the variation of Eq. \ref{631}, this holds for pure stick 
\begin{align*}
C_c^P &= \int_{\Gamma_c} \big(\epsilon_N g_{\bar{N}} \delta g_{\bar{N}} \big) dA 
\end{align*}
Large number for the penalty parameters leads to an ill-conditioned numerical problem

\subsubsection{Remarks}
Similarly as for the Lagrange multiplier remark, we do not need to distinguish between the normal and tangential directions. 
\begin{align}
\begin{split}
C_c^{stick} &= \int_{\Gamma_c} \epsilon_N g_{\bar{N}} \delta g_{\bar{N}} dA \quad \text{Choose equal penalty parameters for all directions } \epsilon = \epsilon_N = \epsilon_T \\
		  &= \int_{\Gamma_c} \epsilon (g \cdot \delta g) dA \quad \text{Refer to Eq. \ref{629}} \\
		  &= \int_{\Gamma_c} \epsilon (\mathbf{x}^2 - \mathbf{\bar{x}}^1) \cdot (\eta^2 - \bar{\eta}^1) dA
\end{split}
\end{align}
A value of $\epsilon$ has to be chosen to prevent an ill-conditioned numerical problem. This will mean the constraint equation is only fulfilled approximately, resulting in a deformation field, $\varphi_\epsilon$ which differs from the exact field $\varphi$. Show:
\begin{equation*}
\abs{\abs{\varphi - \varphi_\epsilon}} \rightarrow 0 \text{ for } \epsilon \rightarrow \infty
\end{equation*}

% =================LAGRANGE MULTIPLIER METHOD===============================
\subsection{Lagrange multiplier method}
\subsubsection{Definitions}
$\Pi_c$: contact contributions from lagrange multiplier \\
$\lambda_N$: Lagrange multiplier in normal direction, analogous to contact pressure \\
$g_N$: normal gap function \\
$p_N$: contact normal pressure \\
$\mu$: sliding friction coefficient \\

\subsubsection{Method}
Using Lagrange multipliers to add constraints to a weak form:
\begin{align}\label{626}
\Pi_c^{LM} = \int_{\Gamma_c} (\lambda_N g_N + \mathbf{\lambda_T} \cdot \mathbf{g_T}) dA
\end{align}
Take the variation of Eq. \ref{626} where each term gives two variations:
\begin{align}
\begin{split}
C_c^{LM} &= \int_{\Gamma_c} \big( \underbrace{ \delta \lambda_N g_N + \lambda_N \delta g_N}_\text{Associated with normal contact} + \underbrace{\delta \mathbf{\lambda_T} \cdot \mathbf{g_T} + \mathbf{\lambda_T} \cdot \delta \mathbf{g_T}}_\text{Associated with tangent contact (stick)} \big) dA \\
C_c^{LM} &= \underbrace{\int_{\Gamma_c} (\lambda_N \delta g_N + \mathbf{\lambda_T} \cdot \delta \mathbf{g_T}) dA}_\text{Virtual work of lagrange multipliers} + \underbrace{\int_{\Gamma_c} (\delta \lambda_N g_N + \delta \mathbf{\lambda_T} \cdot \mathbf{g_T}) dA}_\text{Enforcement of constraints}
\end{split}
\end{align}

\subsubsection{Remarks}
When only stick occurs on the contact interface, we do not distinguish between the normal and tangential directions. Therefore, the constraint condition is given in terms of the deformation of the slave point:
\begin{equation}\label{629}
\mathbf{x^2} - \mathbf{x^1} (\bar{\mathbf{\xi}}) = \mathbf{x^2} - \mathbf{\bar{x}}^1 = 0 
\end{equation}
\textit{Is this equivalent to gap function?} \\ \\
Therefore, the equation for contact contribution is altered:
\begin{align}\label{630}
\begin{split}
C_c^{stick} &= \int_{\Gamma_c} (\lambda_N \delta g_N + \mathbf{\lambda_T} \cdot \delta \mathbf{g_T}) dA + \int_{\Gamma_c} \delta \lambda_N g_N dA \\
		&= \int_{\Gamma_c} (\lambda \cdot \delta g + \lambda \cdot \delta g) dA + \int_{\Gamma_c} \delta \lambda \cdot g dA \\
		&= \int_{\Gamma_c} 2 ( \lambda \cdot \delta g) dA + \int_{\Gamma_c} \delta \lambda \cdot g dA \\
		&= \int_{\Gamma_c} \lambda \cdot (\eta^2 - \bar{\eta}^1) dA + \int_{\Gamma_c} \delta \lambda \cdot [ \mathbf{x^2} - \mathbf{x^1} (\bar{\mathbf{\xi}}) ] dA \quad \text{Not sure how to account for extra 2}\\
C_c^{stick} &= \int_{\Gamma_c} \lambda \cdot (\eta^2 - \bar{\eta}^1) dA + \int_{\Gamma_c} \delta \lambda \cdot [ \mathbf{x^2} - \mathbf{x^1} ] dA 
\end{split}
\end{align}
\textit{Does this indicate the following relationship for the variational gap?}
\begin{equation}\label{varGap}
\delta g = \eta^2 - \bar{\eta}^1
\end{equation}

% =================Augmented Lagrange Methods=====================
\subsection{Augmented Lagrange Methods}
The main idea of this method is combining the penalty method with Lagrange multiplier methods. 
\subsubsection{Methods}
The augmented Lagrange functional is introduced for normal contact: 
\begin{align}\label{658}
\Pi_{N}^{AM} = 
\begin{dcases}
    \int_{\Gamma_c} (\lambda_N g_N + \frac{1}{2} \epsilon_N g_N^2) d \Gamma & \text{for } \hat{\lambda}_N \leq 0, \\
    \int_{\Gamma_c} -\frac{1}{2 \epsilon_N} |\lambda_N|^2 d \Gamma               & \text{for } \hat{\lambda}_N > 0
\end{dcases}
\end{align}
where $ \hat{\lambda}_N \leq 0$ indicates that the gap is closed and $\hat{\lambda}_N > 0$ means the gap is open. We introduce the following augmented lagrangian term.
\begin{align*}
\hat{\lambda}_N = \lambda_N + \epsilon_N g_N
\end{align*}
Taking the variation of Eq. \ref{658} gives the following, where we recognize the augmented lagrangian 
\begin{align*}
C_{N}^{AM} &= 
\begin{dcases}
    \int_{\Gamma_c} (\delta \lambda_N g_N + \lambda_N \delta g_N + \epsilon_N g_N \delta g_N) d \Gamma & \text{for } \hat{\lambda}_N \leq 0, \\
    \int_{\Gamma_c} -\frac{1}{\epsilon_N} \lambda_N \delta \lambda_N d \Gamma               & \text{for } \hat{\lambda}_N > 0
\end{dcases} \\
&= 
\begin{dcases}
    \int_{\Gamma_c} \bigg( \delta \lambda_N g_N + (\lambda_N + \epsilon_N g_N) \delta g_N \bigg) d \Gamma & \text{for } \hat{\lambda}_N \leq 0, \\
    \int_{\Gamma_c} -\frac{1}{\epsilon_N} \lambda_N \delta \lambda_N d \Gamma               & \text{for } \hat{\lambda}_N > 0
\end{dcases} 
\end{align*}
Final equation for variation: 
\begin{align}\label{659}
C_{N}^{AM} &= 
\begin{dcases}
    \int_{\Gamma_c} \bigg( \delta \lambda_N g_N + \hat{\lambda}_N \delta g_N \bigg) d \Gamma & \text{for } \hat{\lambda}_N \leq 0, \\
    \int_{\Gamma_c} -\frac{1}{\epsilon_N} \lambda_N \delta \lambda_N d \Gamma               & \text{for } \hat{\lambda}_N > 0
\end{dcases}
\end{align}

\subsubsection{Remarks}
Uzawa algorithm formulation

\section{Linearization of the Contact Contribution}
The Newton scheme relies on linearization of the weak form. Here is a general form of the contact contribution where normal and tangential contact terms are distinguished: 
\begin{align}\label{6113}
\begin{split}
C_c^G &= \int_{\Gamma_c} c_N d \Gamma + \int_{\Gamma_c} c_T d \Gamma \\
	&= \int_{\Gamma_c} c_N (\lambda_N, g_N, \delta \lambda_N, \delta g_N) d \Gamma + \int_{\Gamma_c} c_T (\pmb{\lambda}_T, \pmb{g}_T, \delta \pmb{\lambda}_T, \delta \pmb{g}_T; p_N)d \Gamma \\
\end{split}
\end{align}
where $c_N$ and $c_T$ are functions of variations and variables. \\ \\
Due to geometrical nonlinearities of the finite deformation problem, the variations $\delta g_N$ and $\delta \pmb{g_T}$ also contribute to the linearization, often resulting in very complex terms. \\ \\
The linearization of the Lagrange parameters $\lambda_N$ and $\pmb{\lambda}_T$ are zero \\ \\
Variation of the tangential contribution is complicated, because the stick and slip cases have to be distinguished between. 

\subsection{Normal Contact}
The linearization of $c_n$ can be computed: 
\begin{align}\label{6114}
\begin{split}
\pdv{c_N}{\pmb{u}} \Delta \pmb{u} &= \pdv{c_N}{\lambda_N} \Delta \lambda_N + \pdv{c_N}{g_N} \Delta g_N + \pdv{c_N}{\delta \lambda_N} \Delta \delta \lambda_N + \pdv{c_N}{\delta g_N} \Delta \delta g_N \quad \text{Where } \delta \lambda_N = 0 \\
\pdv{c_N}{\pmb{u}} \Delta \pmb{u} &= \pdv{c_N}{\lambda_N} \Delta \lambda_N + \pdv{c_N}{g_N} \Delta g_N + \pdv{c_N}{\delta g_N} \Delta \delta g_N 
\end{split}
\end{align}

As an example, say we use the Lagrange multiplier method where the normal contact term is given by the following equation: 
\begin{align}
\begin{split}
c_N &= \lambda_N \delta g_N + \delta \lambda_N g_N \\
\pdv{c_N}{\lambda_N} &= \delta g_N \quad \pdv{c_N}{g_N} = \lambda_N \\
\pdv{c_N}{\delta \lambda_N} &= g_N \quad \pdv{c_N}{\delta g_N} = \delta \lambda_N
\end{split}
\end{align}
Therefore, the linearization of $c_n$ for the Lagrange multiplier method would be: 
\begin{align}
\begin{split}
\pdv{c_N}{\pmb{u}} \Delta \pmb{u} &= \pdv{c_N}{\lambda_N} \Delta \lambda_N + \pdv{c_N}{g_N} \Delta g_N + \pdv{c_N}{\delta \lambda_N} \Delta \delta \lambda_N + \pdv{c_N}{\delta g_N} \Delta \delta g_N \quad \text{Where } \\
			&= \delta g_N \Delta \lambda_N + \lambda_N \Delta g_N + g_N \Delta \delta \lambda_N + \delta \lambda_N \Delta \delta g_N \quad \text{Where } \delta \lambda_N = 0 \\
			&= \delta g_N \Delta \lambda_N + \lambda_N \Delta g_N
\end{split}
\end{align}
\textbf{Definitions from Chapter 4} \\
Non-penetration function:
\begin{equation}\label{46}
g_N = (\mathbf{x}^2 - \mathbf{\bar{x}}^1) \cdot \bar{\mathbf{n}}^1
\end{equation}
We then have to take into account the projection of point x2 onto the master surface parameterized by the convective coordinates $\xi^1$ and $\xi^2$
\begin{equation}\label{428}
\delta g_N = [ \pmb{\eta}^2 - \pmb{\bar{\eta}}^1 - \mathbf{\bar{x}}^1_{,\alpha} \delta \xi^\alpha ] \cdot \mathbf{\bar{n}}_1 + [\mathbf{x}^2 - \mathbf{\bar{x}}^1] \cdot \delta \mathbf{\bar{n}}^1
\end{equation}
Where $\pmb{\eta}^\alpha = \delta \mathbf{x}^\alpha$
\begin{equation}\label{429}
\delta g_N = [\pmb{\eta}^2 - \pmb{\bar{\eta}}^1] \cdot \mathbf{\bar{n}}^1
\end{equation}
\textbf{Using these definitions}, the linearization of $\Delta g_N$ has the same structure as the variation of $g_N$ (Eq. \ref{429}) where we exchange $\pmb{\eta}^\alpha$ by $\Delta \mathbf{u}^\alpha$
\begin{align}\label{6115}
\begin{split}
\delta g_N &= [\Delta \mathbf{u}^2 - \Delta \mathbf{\bar{u}}^1] \cdot \mathbf{\bar{n}}^1 \\
		&=  [\Delta \mathbf{u}^2 - \Delta \mathbf{\bar{u}}^1 (\bar{\xi}^1, \bar{\xi}^2)] \cdot \mathbf{\bar{n}}^1 (\bar{\xi}^1, \bar{\xi}^2)
\end{split}
\end{align}
The linearization of the gap in the normal direction has to be computed from the full variation, Eq. \ref{428}. 
Start with Eq. \ref{46}
\begin{align}\label{6116}
\begin{split}
g_N &= (\mathbf{x}^2 - \mathbf{\bar{x}}^1) \cdot \bar{\mathbf{n}}^1 \\
g_N \bar{\mathbf{n}}^1 &= \mathbf{x}^2 - \mathbf{\bar{x}}^1 (\bar{\xi}^1, \bar{\xi}^2) 
\end{split}
\end{align}
Use this relationship to obtain $\delta g_N$ when multiplied by $\mathbf{\bar{n}}^1$
\begin{align}
\begin{split}
\delta g_N &= [ \pmb{\eta}^2 - \pmb{\bar{\eta}}^1 - \mathbf{\bar{x}}^1_{,\alpha} \delta \xi^\alpha ] \cdot \mathbf{\bar{n}}_1 + [\mathbf{x}^2 - \mathbf{\bar{x}}^1] \cdot \delta \mathbf{\bar{n}}^1 \\
\delta g_N - [\mathbf{x}^2 - \mathbf{\bar{x}}^1] \cdot \delta \mathbf{\bar{n}}^1 &= [ \pmb{\eta}^2 - \pmb{\bar{\eta}}^1 - \mathbf{\bar{x}}^1_{,\alpha} \delta \xi^\alpha ] \cdot \mathbf{\bar{n}}_1 \quad \text{Use Eq. \ref{6116}}\\
\delta g_N \mathbf{\bar{n}}_1 - g_N \bar{\mathbf{n}}^1 \cdot \delta \mathbf{\bar{n}}^1  \mathbf{\bar{n}}_1 &= \pmb{\eta}^2 - \pmb{\bar{\eta}}^1 - \mathbf{\bar{x}}^1_{,\alpha} \delta \xi^\alpha \\
\delta g_N \mathbf{\bar{n}}_1 - g_N \delta \mathbf{\bar{n}}^1 (\bar{\mathbf{n}}^1 \cdot \mathbf{\bar{n}}_1) &= \pmb{\eta}^2 - \pmb{\bar{\eta}}^1 - \mathbf{\bar{x}}^1_{,\alpha} \delta \xi^\alpha \\
\delta g_N \mathbf{\bar{n}}_1 - g_N \delta \mathbf{\bar{n}}^1 &= \pmb{\eta}^2 - \pmb{\bar{\eta}}^1 - \mathbf{\bar{x}}^1_{,\alpha} \delta \xi^\alpha
\end{split}
\end{align}
$\xi$: Deformation dependent surface coordinates 
$\pmb{\bar{\xi}}$: Values related to projection point denoted by bar \\



\end{document}