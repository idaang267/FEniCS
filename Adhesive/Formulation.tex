\documentclass[12pt,3p]{article}
\usepackage[utf8]{inputenc}
\usepackage[english]{babel}
 \usepackage[margin=0.5in]{geometry}
 \usepackage{amsmath}
\usepackage{mathtools}
\usepackage{enumitem}
\usepackage{physics}
\usepackage[round,numbers]{natbib}
\usepackage[colorlinks = false]{hyperref}

\setcounter{tocdepth}{4} 
\setcounter{secnumdepth}{4}

\numberwithin{equation}{section}
\begin{document}

\title{Neo-Hookean Strain Energy Density Functions \\
	\large{Updated on \_}}
\author{Ida Ang}
\date{\vspace{-5ex}}
\maketitle

\tableofcontents
\newpage

%====================
%====================
%====================
\section{Definitions}
Young's Modulus: E \\
Poisson's Ratio: $\nu$ \\
Shear Modulus or first Lamé Parameter: $\mu$ \\
Second Lamé Parameter: $\lambda$ 
\begin{equation}\label{EqLame}
\mu = \frac{E}{2 (1 + \nu)} \quad \lambda = \frac{E \nu}{(1+ \nu) (1 - 2 \nu)} 
\end{equation}
Bulk Modulus: $\kappa$ \\ 
\begin{equation}\label{EqBulk}
\kappa = \lambda + \frac{2}{3} \mu 
\end{equation}
Deformation Gradient: $\mathbf{F}$ \\
Right Cauchy Green Tensor: $\mathbf{C} = \mathbf{F}^T \mathbf{F}$ \\
1st Piola Kirchoff Stress: $\mathbf{P}$ \\
The third invariant of the deformation gradient 
\begin{equation}
I_3 = J = \det \mathbf{F} 
\end{equation}
The first invariant of the right Cauchy Green Tensor:
\begin{equation}\label{EqInvariants}
I^\mathbf{C}_1 = \trace \mathbf{C}  
\end{equation}
Lastly, we note a relationship between pressure and the bulk modulus
\begin{equation}\label{EqPressure}
p = - \kappa I_3 (I_3 - 1)
\end{equation}
\noindent\rule{19cm}{0.5pt} \\ % Straight line across
Proof of the derivative $\pdv{\trace(\mathbf{F^T} \mathbf{F})}{\mathbf{F}}$ in indicial: 
\begin{align*}
\frac{\partial I_1^{\mathbf{C}}}{\partial \mathbf{F}} &= \pdv{\trace(\mathbf{F^T} \mathbf{F})}{\mathbf{F}} \\
									&= \pdv{F_{kI} F_{kI}}{F_{pQ}} \quad \text{Product Rule} \\ 
									&= \pdv{F_{kI}}{F_{pQ}} F_{kI} + F_{kI} \pdv{F_{kI}}{F_{pQ}} \\
									&= \delta_{kp} \delta_{IQ} F_{kI} + F_{kI} \delta_{kp} \delta_{IQ} \\
									&= F_{pQ} + F_{pQ} = 2 F_{pQ} 
\end{align*}
Therefore, we have the following relationships: 
\begin{align}\label{EqProof1}
\begin{split}
\frac{\partial I_1^{\mathbf{C}}}{\partial \mathbf{F}} &= 2 \mathbf{F} \\
\frac{\partial I_3}{\partial \mathbf{F}} &= I_3 \mathbf{F}^{-T}
\end{split}
\end{align}
\noindent\rule{19cm}{0.5pt} % Straight line across

%====================
%====================
%====================
\section{Compressible Cases}
There are several variations for the strain energy density function of a compressible neo-Hookean material 
\begin{align}\label{EqComp}
\begin{split}
W(\mathbf{F}) &= \frac{\mu}{2} (I_1^{\mathbf{C}} - 3 - 2 \ln I_3) + \frac{\kappa}{2} (I_3 - 1)^2 \\
W(\mathbf{F}) &= \frac{\mu}{2} (I_1^{\mathbf{C}} - 3 - 2 \ln I_3) + \frac{\lambda}{2} (\ln I_3)^2 \\
\end{split}
\end{align}
We can calculate the first PK for the first expression using Eq. \ref{EqProof1}
\begin{align}
\begin{split}
\mathbf{P} &= \frac{\partial W}{\partial \mathbf{F}} \quad \text{using chain rule} \\
		&= \frac{\partial W}{\partial I_1^\mathbf{C}} \frac{\partial I_1^\mathbf{C}}{\partial \mathbf{F}} + \frac{\partial W}{\partial I_3} \frac{\partial I_3}{\partial \mathbf{F}} \\
		&= \frac{\mu}{2} \big( 2 \mathbf{F} \big) + \bigg[ \frac{\mu}{2} \big( -2 \frac{\partial \ln I_3}{\partial I_3}\big) + \frac{\kappa}{2} \frac{\partial (I_3 - 1)^2}{\partial I_3}\bigg] I_3 \mathbf{F}^{-T} \\
		&= \mu \mathbf{F} + \bigg[ - \frac{\mu}{I_3} + \kappa (I_3 - 1) \bigg] I_3 \mathbf{F}^{-T} \\
		&=  \mu \mathbf{F} - \mu \mathbf{F}^{-T} + \kappa I_3 (I_3 - 1) \mathbf{F}^{-T} \\
\mathbf{P} &= \mu ( \mathbf{F} - \mathbf{F}^{-T}) + \kappa I_3 (I_3 - 1) \mathbf{F}^{-T} 
\end{split}
\end{align}
For the second expression
\begin{align}
\begin{split}
\mathbf{P} &= \frac{\partial W}{\partial I_1^\mathbf{C}} \frac{\partial I_1^\mathbf{C}}{\partial \mathbf{F}} + \frac{\partial W}{\partial I_3} \frac{\partial I_3}{\partial \mathbf{F}} \\
		&= \frac{\mu}{2} \big( 2 \mathbf{F} \big) + \bigg[ \frac{\mu}{2} \big( -2 \frac{\partial \ln I_3}{\partial I_3}\big) + \frac{\lambda}{2} \frac{\partial (\ln I_3 )^2}{\partial I_3}\bigg] I_3 \mathbf{F}^{-T} \\
		&= \mu \mathbf{F} + \bigg[ - \frac{\mu}{I_3} + \lambda \ln I_3  \bigg] I_3 \mathbf{F}^{-T} \\
		&=  \mu \mathbf{F} - \mu \mathbf{F}^{-T} + \lambda I_3 (\ln I_3 ) \mathbf{F}^{-T} \\
\mathbf{P} &= \mu ( \mathbf{F} - \mathbf{F}^{-T}) + \lambda I_3 (\ln I_3) \mathbf{F}^{-T} 
\end{split}
\end{align}

%====================
%====================
%====================
\section{Incompressible Cases}
For an incompressible material, $I_1$ = det(\textbf{F}) = 1; therefore, Eq. \ref{EqComp} can be modified, 
\begin{align*}
W(\mathbf{F}) &= \frac{\mu}{2} (I_1^{\mathbf{C}} - 3 - 2 \ln (1)) + \frac{\kappa}{2} (1 - 1)^2  \\
W(\mathbf{F}) &= \frac{\mu}{2} (I_1^{\mathbf{C}} - 3 - 2 \ln (1)) + \frac{\lambda}{2} (\ln (1))^2  
\end{align*}
Therefore, we can write
\begin{equation}\label{EqComp}
W (\mathbf{F})  = \frac{\mu}{2} (I_1^{\mathbf{C}}  - 3) 
\end{equation}
To enforce incompressibility, we can utilize the Lagrange multiplier method where the lagrange multiplier, p, enforces the constraint that J=1: 
\begin{align}\label{EqCompLagrange}
\begin{split}
W(\mathbf{F}) &= \frac{\mu}{2} (I_1^{\mathbf{C}}  - 3)  + \text{p} (I_3-1) \\
W(\mathbf{F}) &= \frac{\mu}{2} (I_1^{\mathbf{C}}  - 3 - 2 \ln I_3 )  + \text{p} (I_3-1) 
\end{split}
\end{align}
The 1st PK of Eq. \ref{EqCompLagrange} the first expression 
\begin{align*}
\begin{split}
\mathbf{P} &= \frac{\partial W}{\partial I_1^\mathbf{C}} \frac{\partial I_1^\mathbf{C}}{\partial \mathbf{F}} + \frac{\partial W}{\partial I_3} \frac{\partial I_3}{\partial \mathbf{F}} \\
		&= \frac{\mu}{2} \big( 2 \mathbf{F} \big) + p I_3 \mathbf{F}^{-T} \\
\mathbf{P} &= \mu \mathbf{F} + p I_3 \mathbf{F}^{-T} 
\end{split}
\end{align*}
The 1st PK of the second expression 
\begin{align}
\begin{split}
\mathbf{P} &= \frac{\partial W}{\partial I_1^\mathbf{C}} \frac{\partial I_1^\mathbf{C}}{\partial \mathbf{F}} + \frac{\partial W}{\partial I_3} \frac{\partial I_3}{\partial \mathbf{F}} \\
		&= \frac{\mu}{2} \big( 2 \mathbf{F} \big) + \bigg[ \frac{\mu}{2} \big( -2 \frac{\partial \ln I_3}{\partial I_3}\big) + p \bigg] I_3 \mathbf{F}^{-T} \\
		&= \mu \mathbf{F} + \bigg[ - \frac{\mu}{I_3} + p \bigg] I_3 \mathbf{F}^{-T} \\
		&=  \mu \mathbf{F} - \mu \mathbf{F}^{-T} +p I_3 \mathbf{F}^{-T} \\
\mathbf{P} &= \mu ( \mathbf{F} - \mathbf{F}^{-T}) + p I_3 \mathbf{F}^{-T} 
\end{split}
\end{align}

%====================
%====================
%====================
\section{Summary}
Compressible neo-Hookean, case 1
\begin{align}
\begin{split}
W(\mathbf{F}) &= \frac{\mu}{2} (I_1^{\mathbf{C}} - 3 - 2 \ln I_3) + \frac{\kappa}{2} (I_3 - 1)^2 \\
\mathbf{P} &= \mu ( \mathbf{F} - \mathbf{F}^{-T}) + \kappa I_3 (I_3 - 1) \mathbf{F}^{-T} 
\end{split}
\end{align}
Compressible neo-Hookean, case 2
\begin{align}
\begin{split}
W(\mathbf{F}) &= \frac{\mu}{2} (I_1^{\mathbf{C}} - 3 - 2 \ln I_3) + \frac{\lambda}{2} (\ln I_3)^2 \\
\mathbf{P} &= \mu ( \mathbf{F} - \mathbf{F}^{-T}) + \lambda I_3 (\ln I_3) \mathbf{F}^{-T} 
\end{split}
\end{align}
Incompressible neo-Hookean, case 1
\begin{align}
\begin{split}
W(\mathbf{F}) &= \frac{\mu}{2} (I_1^{\mathbf{C}}  - 3)  + \text{p} (I_3-1) \\
\mathbf{P} &= \mu \mathbf{F} + p I_3 \mathbf{F}^{-T} 
\end{split}
\end{align}
Incompressible neo-Hookean, case 2 
\begin{align}
\begin{split}
W(\mathbf{F}) &= \frac{\mu}{2} (I_1^{\mathbf{C}}  - 3 - 2 \ln I_3 )  + \text{p} (I_3-1) \\
\mathbf{P} &= \mu ( \mathbf{F} - \mathbf{F}^{-T}) + p I_3 \mathbf{F}^{-T} 
\end{split}
\end{align}


\end{document}