\documentclass[12pt,3p]{article}
\usepackage[T1]{fontenc}
\usepackage[utf8]{inputenc}
\usepackage[english]{babel}
\usepackage[margin=0.75in]{geometry}
\usepackage{amsmath}
\usepackage{mathtools}
\usepackage{enumitem}
\usepackage{physics}

\usepackage[round,numbers]{natbib}
\usepackage[colorlinks = false]{hyperref}

\begin{document}

\title{FEniCS: Mixed formulation for Poisson Equation \\
	\large{From documented demonstration number 12 from Dolfin version 1.4.0}}
\author{Ida Ang (Edited July 1, 2019)}
\date{\vspace{-5ex}}
\maketitle

\section{Problem Definition}
Consider a model for the temperature u in a body occupying a domain $\Omega$ subject to a heat source, f. Let $\sigma = \sigma(x)$ denote heat flux. It follows by conservation of energy that the outflow of energy over the boundary $\delta \Omega$ must be balanced by the energy emitted by the heat source f: 
\begin{align}\label{ProbDef1}
\begin{split}
\int_{\delta \Omega} \sigma \cdot n ds &= - \int_{\Omega}  f dx \quad \text{Use Guassian divergence theorem} \\
\int_{\Omega} \nabla \cdot \sigma dx &= - \int_{\Omega}  f dx \\
\nabla \cdot \sigma &= - f \quad \text{in } \Omega
\end{split}
\end{align}
Assume that the heat flux, $\sigma$, is proportional to the gradient of the negative temperature $u$ (Fourier's law)
\begin{align}\label{ProbDef2}
\begin{split}
\sigma &= - \kappa \nabla u \quad \text{Heat conductivity } \kappa = -1 \\
\sigma &= \nabla u \\
\sigma - \nabla u &= 0 \indent \text{in } \Omega
\end{split}
\end{align}
The same equations arise in connection with flow in porous media, and are referred to as Darcy flow. \\ \\
\textbf{Boundary conditions}: \\
The Dirichlet boundary condition is the natural boundary condition within the variational form. The Neumann boundary condition for flux is now the essential boundary condition which must be enforced in the function space.
\begin{equation}\label{DBound}
\text{Dirichlet Boundary: } u = u_o \indent \text{on } \Gamma_D
\end{equation}
\begin{equation}\label{NBound}
\text{Neumann Boundary: }\sigma \cdot n = g \indent \text{on } \Gamma_N
\end{equation}

\section{Weak Form}
We have two partial differential equations, Eq. \ref{ProbDef1} and \ref{ProbDef2}. This leads to the introduction of two different test functions, $\tau$ and $v$. \\ \\
\textbf{First} \\
Convert Eq. \ref{ProbDef2} to indicial notation: 
\begin{align}\label{Parts}
\begin{split}
\sigma = \nabla u \rightarrow \sigma_{ij} (\mathbf{e_i} \otimes \mathbf{e_j}) &= \pdv{u_i}{x_j} \big( \mathbf{e_i} \otimes \mathbf{e_j} \big) \quad \text{Multiply with test function } \tau \\
\sigma_{ij} (\mathbf{e_i} \otimes \mathbf{e_j}) \cdot \tau_k \mathbf{e_k} &= \pdv{u_i}{x_j} \big( \mathbf{e_i} \otimes \mathbf{e_j} \big) \cdot \tau_k \mathbf{e_k} \\
\sigma_{ij} \tau_k \mathbf{e_i} \delta_{jk} &= \pdv{u_i}{x_j} \tau_k \mathbf{e_i} \delta_{jk} \\
\sigma_{ij} \tau_j \mathbf{e_i} &= \pdv{u_i}{x_j} \tau_j \mathbf{e_i} \quad \text{Integrate over domain} \\
\int_{\Omega} \sigma_{ij} \tau_j \mathbf{e_i} dx &= \int_{\Omega} \pdv{u_i}{x_j} \tau_j \mathbf{e_i} dx
\end{split}
\end{align}
Use integration by parts on the right hand side 
\begin{equation}\label{intParts}
(fg)' = f'g + fg' \rightarrow f'g = (fg)' - fg'
\end{equation}
Where we can substitute $f' = \pdv{u_i}{x_j} \mathbf{e_i} $ and $g = \tau_j$ into Eq. \ref{intParts}
\begin{equation*}
\pdv{u_i}{x_j} \mathbf{e_i} \tau_j = (u_i \mathbf{e_i} \tau_j)_{,j} - u_i \mathbf{e_i} \pdv{\tau_j}{x_j}
\end{equation*}
Substitute into RHS of Eq. \ref{Parts}
\begin{align*}
\int_{\Omega} \sigma_{ij} \tau_j \mathbf{e_i} dx &= \int_{\Omega} (u_i \mathbf{e_i} \tau_j)_{,j} dx - \int_{\Omega} u_i \mathbf{e_i} \pdv{\tau_j}{x_j} dx \quad \text{Use divergence theorem} \\
\int_{\Omega} \sigma_{ij} \tau_j \mathbf{e_i} dx &= \int_{\partial \Omega} u_i \mathbf{e_i} \tau_j n_j ds - \int_{\Omega} u_i \mathbf{e_i} \pdv{\tau_j}{x_j} dx \quad \text{Rearrange} \\
\int_{\Omega} \sigma_{ij} \tau_j \mathbf{e_i} dx + \int_{\Omega} \pdv{\tau_j}{x_j} u_i \mathbf{e_i} dx &= \int_{\partial \Omega} \tau_j n_j u_i \mathbf{e_i} ds 
\end{align*}
Convert this into direct notation: \\
\textit{\textbf{Term 1}}
\begin{equation*}
\sigma \cdot \tau = \sigma_{ij} (\mathbf{e_i} \otimes \mathbf{e_j}) \cdot \tau_k \mathbf{e_k} = \sigma_{ij} \tau_k \mathbf{e_i} (\mathbf{e_j} \cdot \mathbf{e_k}) =  \sigma_{ij} \tau_k \mathbf{e_i} \delta_{jk} = \sigma_{ij} \tau_j \mathbf{e_i}
\end{equation*}
\textit{\textbf{Term 2}}
\begin{equation*}
(\nabla \cdot \tau) u = \big( \pdv{\tau_j}{x_i} \mathbf{e_j} \cdot \mathbf{e_i}) u_k \mathbf{e_k} =  \pdv{\tau_j}{x_i} \delta_{ji} u_k \mathbf{e_k} = \pdv{\tau_j}{x_j} u_k \mathbf{e_k} 
\end{equation*}
\textit{\textbf{Term 3}}
\begin{equation*}
(\tau \cdot n) u = (\tau_j \mathbf{e_j} \cdot n_k \mathbf{e_k}) u_i \mathbf{e_i} = \tau_j n_k \delta_{jk} u_i \mathbf{e_i} = \tau_j n_j u_i \mathbf{e_i}
\end{equation*}
Finally, we have the weak form: 
\begin{equation}\label{weakForm1_b}
\int_{\Omega} \sigma \cdot \tau dx + \int_{\Omega} (\nabla \cdot \tau) u dx = \int_{\Gamma} (\tau \cdot n) u ds \indent \forall \tau \in \Sigma
\end{equation}
\textbf{Second} \\
Write Eq. \ref{ProbDef1} in indicial 
\begin{align*}
\nabla \cdot \sigma = -f \rightarrow \pdv{\sigma_{ij}}{x_j} \mathbf{e_i} &= - f_i \mathbf{e_i} \quad \text{Multiply by test function } v \\
\pdv{\sigma_{ij}}{x_j} \mathbf{e_i} \cdot v_p \mathbf{e_p} &= - f_i \mathbf{e_i} \cdot v_p \mathbf{e_p}  \\ 
\pdv{\sigma_{ij}}{x_j} v_p \delta_{ip} &= - f_i v_p \delta_{ip}  \\
\pdv{\sigma_{ij}}{x_j} v_i &= - f_i v_i \quad \text{Integrate over domain} \\
\int_{\Omega} \pdv{\sigma_{ij}}{x_j} v_i dx &= - \int_{\Omega} f_i v_i dx 
\end{align*}
Change to direct notation, leaving the following variational (weak) formulations. Looking at Eq. \ref{weakForm1_b} we can see that the Dirichlet boundary condition (Eq. \ref{DBound}) is a natural boundary condition, which should be applied to the variational form.
\begin{equation}\label{WeakForm1}
\int_{\Omega} \sigma \cdot \tau dx + \int_{\Omega} (\nabla \cdot \tau) u dx = \int_{\Gamma} (\tau \cdot n) u_o ds \indent \forall \tau \in \Sigma
\end{equation}
\begin{equation}\label{WeakForm2}
\int_{\Omega} (\nabla \cdot \sigma) v dx = - \int_{\Omega} fv dx \indent \forall v \in V
\end{equation}

\section{Bilinear Form and Linear Form}
Inserting the boundary conditions, this variational problem can be phrased in the general form. Recall our general form for the demonstration of the simple Poisson problem was written as follows: 
\begin{equation*}
a(u,v) = L(v)  \indent \forall v \in \hat{V}
\end{equation*}
For this problem, we can modify this general form because we have two test functions
\begin{equation*}
a\big((\sigma,u),(\tau, v) \big) = L\big((\tau, v) \big)  \indent \forall (\tau, v) \in \Sigma_0 \times V 
\end{equation*}
We can rewrite Eq. \ref{WeakForm1} and \ref{WeakForm2} in the bilinear form and linear form:
\begin{equation}\label{bilinear}
a\big((\sigma,u),(\tau, v) \big) = \int_{\Omega} \bigg[ \sigma \cdot \tau  + (\nabla \cdot \tau) u + (\nabla \cdot \sigma) v \bigg] dx
\end{equation}
\begin{equation}\label{linear}
L\big((\tau, v) \big) = - \int_{\Omega} fv dx +  \int_{\Gamma_D} (\tau \cdot n) u_o ds
\end{equation}
\textbf{Note}: The Neumann boundary condition for the flux (Eq. \ref{NBound}) is now an essential boundary condition, which should be enforced in the function space. The Dirichlet boundary condition is the natural boundary condition within the weak form.

\section{FEniCS Implementation}
\subsection{Domain}
The domain is a unit square: 
\[ \Omega = [0,1] \times [0,1] \] 
Create a unit square \\
{\fontfamily{qcr}\selectfont
mesh = UnitSquareMesh(32, 32) \\ \\
}
To discretize the formulation, two discrete function spaces $\Sigma_h \subset \Sigma$ and $V_h \subset V$ are needed to form a mixed function space $\Sigma_h \times V_h$. \\
\begin{equation*}
W = \{ (\tau, v) \text{ such that } \tau \in BDM, v \in DG \}
\end{equation*}
\textbf{Define a mixed function space:}  \\
Brezzi-Douglas-Marini (BDM) is a vector space for stress and Discontinuous Galerkin (DG) is a scalar space for displacement. A stable choice of finite element spaces is to let $\Sigma_h$ be the BDM elements of polynomial order k, 1, and let $V_h$ be DG elements of polynomial order k-1, 0. \\
{\fontfamily{qcr}\selectfont
BDM = FiniteElement("BDM", mesh.ufl\_cell(), 1)  \\
DG  = FiniteElement("DG", mesh.ufl\_cell(), 0) \\ \\
}
\textit{\textbf{Note}}: Depreciated function where {\fontfamily{qcr}\selectfont W = BDM * DG} can't be used \\
{\fontfamily{qcr}\selectfont
W = FunctionSpace(mesh, BDM * DG) \\ \\
}
Next we need to specify the trial functions (unknowns) and the test functions. Note {\fontfamily{qcr}\selectfont TrialFunctions} for more than one {\fontfamily{qcr}\selectfont TrialFunction}. We are finding the stress, $\sigma$, and displacement, $u$ corresponding to the two {\fontfamily{qcr}\selectfont TestFunctions}, $\tau$ and $v$. \\
{\fontfamily{qcr}\selectfont
(sigma, u) = TrialFunctions(W) \\
(tau, v) = TestFunctions(W) \\ \\
}
Define input functions \\
\begin{equation}\label{f}
f = 10 \exp (-((x-0.5)^2 +(y -0.5)^2)/0.02)
\end{equation}
\begin{equation}\label{g}
g = \sin (5x)
\end{equation}
Write the defined input functions: Eq. \ref{f} and \ref{g} in C++ syntax (more efficient). The degree is specified for interpolation on the discretized mesh. \\
{\fontfamily{qcr}\selectfont
f = Expression("10*exp(-(pow(x[0] - 0.5, 2) \\
\indent \indent \indent \indent \indent \indent + pow(x[1] - 0.5, 2)) / 0.02)", degree = 1) \\ 
g = Expression("sin(5.0*x[0])", degree = 1)
}

\subsection{Boundary}
\textbf{Dirichlet boundary} (Natural in this formulation): right and left sides of the unit square
\[ \Gamma_{D} = \{(0,y) \cup (1,y) \in \partial \Omega \} \]
\textbf{Neumann boundary} (Essential in this formulation): bottom and top sides of the unit square
\[ \Gamma_{N} = \{(x,0) \cup (x,1) \in \partial \Omega \} \]
The displacement on the Dirichlet BC is specified as zero ($u_o = 0$), changing the linear form, Eq. \ref{linear}. The bilinear form remains the same. 
\begin{align*}
a\big((\sigma,u),(\tau, v) \big) &= \int_{\Omega} \bigg[ \sigma \cdot \tau  + (\nabla \cdot \tau) u + (\nabla \cdot \sigma) v \bigg] dx \\
L\big((\tau, v) \big) &= - \int_{\Omega} fv dx
\end{align*}
Define variational form \\
{\fontfamily{qcr}\selectfont
a = (dot(sigma, tau) + div(tau)*u + div(sigma)*v)*dx \\ 
L = - f*v*dx \\ \\
}
Prescribe the boundary condition for the flux. Specifying the relevant part of the boundary can be done similarly to the Poisson demo, except we are defining the Neumann bc not the Dirichlet bc: \\
{\fontfamily{qcr}\selectfont
def boundary(x): \\
\indent \indent return x[0] < DOLFIN\_EPS or x[0] > 1.0 - DOLFIN\_EPS \\ \\
} 
\textbf{Subclassing Expression Class} \\
An essential bc is handled by replacing degrees of freedom (DOF) by the DOF evaluated at the given data. The BDM finite element spaces are vector-valued spaces and hence the degrees of freedom act on vector-valued objects. Construct a G:
\[G \cdot n = g \rightarrow G = gn\]
{\fontfamily{qcr}\selectfont
class BoundarySource(UserExpression): \\
\indent def \_\_init\_\_(self, mesh, **kwargs): \\
\indent \indent self.mesh = mesh \\ \\
} 
Overloading the {\fontfamily{qcr}\selectfont eval\_cell} method, instead of the usual {\fontfamily{qcr}\selectfont eval}, allows us to extract more geometry information such as the facet normals (cell.normal()). \\ 
{\fontfamily{qcr}\selectfont
\indent def eval\_cell(self, values, x, ufc\_cell): \\
\indent \indent  cell = Cell(self.mesh, ufc\_cell.index) \\
\indent \indent  n = cell.normal(ufc\_cell.local\_facet) \\ \\
}
Note, how g is defined again within this class as well as within the expression \\
{\fontfamily{qcr}\selectfont
\indent \indent   g = sin(5*x[0]) \\ \\
}
Construct g (G = gn). according to Eq. \ref{g} and index values to return. \\
{\fontfamily{qcr}\selectfont
\indent \indent  values[0] = g*n[0] \\
\indent \indent  values[1] = g*n[1] \\ \\
}
Since this is a vector-valued expression, we also need to overload the {\fontfamily{qcr}\selectfont value\_shape} method. \\
{\fontfamily{qcr}\selectfont
\indent def value\_shape(self): \\
\indent \indent  return (2,) \\ \\
}
Next, call the constructed class {\fontfamily{qcr}\selectfont BoundarySource } within the main code: \\ 
{\fontfamily{qcr}\selectfont
G = BoundarySource(mesh, degree=2) \\ \\
}
We want to apply the boundary condition to the first subspace of the mixed space. Subspaces of a mixed FunctionSpace can be accessed by the method sub. In our case, this reads {\fontfamily{qcr}\selectfont W.sub(0)} \\
{\fontfamily{qcr}\selectfont
bc = DirichletBC(W.sub(0), G, boundary)
}

\subsection{Solution and Output}
We need to create a {\fontfamily{qcr}\selectfont Function} to store the solutions. The (full) solution will be stored in "w", which we initialise using FunctionSpace "W" \\
{\fontfamily{qcr}\selectfont
w = Function(W) \\ \\
}
We equate the bilinear and linear forms and call the boundary condition. The computation is performed by calling {\fontfamily{qcr}\selectfont solve}. \\
{\fontfamily{qcr}\selectfont
solve(a == L, w, bc) \\ \\
}
The separate components {\fontfamily{qcr}\selectfont sigma} and {\fontfamily{qcr}\selectfont u} of the solution can be extracted by the {\fontfamily{qcr}\selectfont split} function. \\
{\fontfamily{qcr}\selectfont
(sigma, u) = w.split() \\ \\
}
Save each component by using .xdmf file format which can save multiple fields. Use {\fontfamily{qcr}\selectfont rename } function so the output won't be saved under a random name. \\
{\fontfamily{qcr}\selectfont
file = XDMFFile("mixed\_poisson.xdmf") \\
sigma.rename("Stress", "sigma") \\
u.rename("Temperature", "u") \\ \\
}
Parameters that will allow the file to be saved even if the script exits prematurely and will allow the functions to share the same mesh. \\
{\fontfamily{qcr}\selectfont
file.parameters["flush\_output"] = True \\
file.parameters["functions\_share\_mesh"] = True \\ \\
}
Save parameters with time point 0.0\\
{\fontfamily{qcr}\selectfont
file.write(sigma, 0.0) \\
file.write(u, 0.0) \\ \\
}

\end{document}
